% To pass options to packages loaded from class file, these commands sust be called before the
% \documentclass command.

% Capitalize references when using \cref.
\PassOptionsToPackage{capitalise}{cleveref}

% Add support for inline lists with enumitem package. 
\PassOptionsToPackage{inline}{enumitem}

% Suppress vertical gap between glossary groups (entries are grouped by their first letter).
\PassOptionsToPackage{nogroupskip}{glossaries}

\documentclass[
    % print,
    % draft,
    % nativefonts,
]{dissertation}

% Large first letter of chapter using lettrine.
\usepackage{lettrine}
\renewcommand{\LettrineTextFont}{\scshape}

% Units formatted using siunitx.
\usepackage{siunitx}
\sisetup{
    per-mode=symbol,
}

% Bibliography managed using BibLaTeX.
\usepackage[
    giveninits=true,     % only given name initials and not the full given name
    maxnames=50,         % max authors in bibliography
    maxcitenames=2,      % max authors when citing
    urldate=long,        % explicit URL date
    backend=biber,       % biber backend
    refsection=chapter,  % start a reference section at every \chapter command
]{biblatex}
\addbibresource{references/articles.bib}
\addbibresource{references/arxiv.bib}
\addbibresource{references/misc.bib}
\addbibresource{references/online.bib}
\addbibresource{references/proceedings.bib}

%%%%%%%%%%%%%%%%%%%%%%%%%%%%%%%%%%%%%%%%%%%%%%%%%%%%%%%%%%%%%%%%%%%%%%%%%%%%%%%%%%%%%%%%%%%%%%%%%%%%

% Not sure why this is defined.
\newcommand{\sparkline}[1]{$\vcenter{\hbox{\includegraphics[scale=0.04]{#1}}}$}

% Not sure why this is defined.
\newcommand*{\origrightarrow}{}
\let\oldarrow\textrightarrow
\renewcommand*{\textrightarrow}{\fontfamily{cmr}\selectfont\origrightarrow}

% Custom command: \note{description}
\newcommand{\note}[1]{{\color{red}[\textbf{NOTE:} #1]}}

% Custom command: \todo{description}
\newcommand{\todo}[1]{{\color{blue}[\textbf{TODO:} #1]}}

% Custom hyphenation
\hyphenation{QNodeOS}  % Do not hyphenate QNodeOS
\hyphenation{QDevice}  % Do not hyphenate QDevice

% Define acronyms.
% Notice the use of \makenoidxglossary instead of \makeglossary to avoid running external tools
% (https://github.com/tectonic-typesetting/tectonic/issues/704).
\makenoidxglossaries
\loadglsentries[main]{glossary}

% Disable hyperlinks from acronyms to glossary entries.
\glsdisablehyper

% Set default style for lists. 
\setlist[enumerate,itemize]{
    topsep=1ex,
    itemsep=1ex,
    parsep=0ex,
    partopsep=0ex,
    leftmargin=*
}

%%%%%%%%%%%%%%%%%%%%%%%%%%%%%%%%%%%%%%%%%%%%%%%%%%%%%%%%%%%%%%%%%%%%%%%%%%%%%%%%%%%%%%%%%%%%%%%%%%%%

\begin{document}

% Specify the title and author of the thesis. This information will be used on the title page (in
% chapters/meta/title.tex) and in the metadata of the final PDF.
\title{Software Abstractions for Programmable Quantum Network Nodes}
\author{Carlo}{Delle Donne}

% Use Roman numerals for the page numbers of the title pages and table of contents.
\frontmatter

\begin{titlepage}

\begin{center}

% Extra whitespace at the top.
\vspace*{2\bigskipamount}

% Print the title.
{\makeatletter
\titlestyle\bfseries\LARGE\@title
\makeatother}

% Print the optional subtitle.
{\makeatletter
\ifx\@subtitle\undefined\else
    \bigskip
    \titlefont\titleshape\Large\@subtitle
\fi
\makeatother}

\end{center}

\cleardoublepage
\thispagestyle{empty}

\begin{center}

% The following lines repeat the previous page exactly.

\vspace*{2\bigskipamount}

% Print the title.
{\makeatletter
\titlestyle\bfseries\LARGE\@title
\makeatother}

% Print the optional subtitle.
{\makeatletter
\ifx\@subtitle\undefined\else
    \bigskip
    \titlefont\titleshape\Large\@subtitle
\fi
\makeatother}

% Uncomment the following lines to insert a vertically centered picture into the title page.
%\vfill
%\includegraphics{title}
\vfill

% Apart from the names and dates, the following text is dictated by the promotieregelement.

{\Large\titlefont\bfseries Dissertation}

\bigskip
\bigskip

for the purpose of obtaining the degree of doctor \\
at Delft University of Technology, \\
by the authority of the Rector Magnificus, prof.~dr.~ir.~T.H.J.J.~van~der~Hagen, \\
chair of the Board for Doctorates, \\
to be defended publicly on \\
Tuesday, 27 June 2023 at 12:30 o'clock

\bigskip
\bigskip

by

\bigskip
\bigskip

% Print the full name of the author.
\makeatletter
{\Large\titlefont\bfseries\@firstname\ \titleshape{\MakeUppercase{\@lastname}}}
\makeatother

\bigskip
\bigskip

Master of Science in Embedded Systems, \\
Delft University of Technology, the Netherlands, \\
born in Potenza, Italy.

% Extra whitespace at the bottom.
\vspace*{2\bigskipamount}

\end{center}

\clearpage
\thispagestyle{empty}

% The following line is dictated by the promotieregelement.
\noindent
This dissertation has been approved by the promotors.

\bigskip\noindent
Composition of the doctoral committee:

% List the committee members, starting with the Rector Magnificus and the promotor(s) and ending
% with the reserve members.
\medskip\noindent
\begin{tabular}{p{4.5cm}l}
    Rector Magnificus, & chairperson \\
    Prof.\ dr.\ S.\ D.\ C.\ Wehner, & promotor \\
    Dr.\ P.\ Pawełczak, & promotor \\
\end{tabular}

\medskip\noindent
\begin{tabular}{p{4.5cm}l}
    \mbox{\emph{Independent members:}} & \\
    Prof.\ dr.\ A.\ van Deursen,    & Delft University of Technology \\
    Dr.\ L.\ Y.\ Chen,              & Delft University of Technology \\
    Prof.\ dr.\ R.\ Van Meter,      & Keio University, Japan \\
    Dr.\ C.\ G.\ Almudéver,         & Technical University of Valencia, Spain \\
    Prof.\ dr.\ K.\ G.\ Langendoen, & Delft University of Technology, reserve member \\
\end{tabular}

% Include the following disclaimer for committee members who have contributed to this dissertation.
% Its formulation is again dictated by the promotieregelement.
% \medskip\noindent
% Prof.\ dr.\ ir.\ R.\ Hanson has contributed to the creation of this thesis.

\vfill

% Here you can include the logos of any institute that contributed financially to this dissertation.
\begin{center}
\noindent
\includegraphics[height=0.5in]{figures/logos/tudelft.pdf}
\hfill
\includegraphics[height=0.5in]{figures/logos/qutech.pdf}
\hfill
\includegraphics[height=0.5in]{figures/logos/kavli.pdf}
\\ \vspace{2\baselineskip}
\includegraphics[height=0.5in]{figures/logos/nwo.pdf}
\hfill
\includegraphics[height=0.5in]{figures/logos/qia.pdf}
\hfill
\includegraphics[height=0.5in]{figures/logos/erc.pdf}
\hfill
\includegraphics[height=0.5in]{figures/logos/casimir.pdf}
\end{center}

\vfill

\noindent
\begin{tabular}{@{}p{0.2\textwidth}@{}p{0.8\textwidth}}
    \textit{Keywords:} & Quantum networks, operating systems \\[\medskipamount]
    \textit{Cover:} & 無音の暴君 (\emph{Silent tyrant}), Mari Nakagawa \\[\medskipamount]
    \textit{Style:} & TU Delft House Style, with modifications by Moritz Beller \\
    & \url{https://github.com/Inventitech/phd-thesis-template} \\[\medskipamount]
    \textit{Printed by:} & Ipskamp Printing \\
    & \url{https://www.ipskampprinting.nl/proefschriften/} \\[\medskipamount]
\end{tabular}

\vspace{\bigskipamount}

% Copyrighting this is questionable, because large parts of the thesis have already been published
% with the copyright resigning with the publisher.
% \noindent
% Copyright \textcopyright\ 2015 by A.~Einstein

% Uncomment the following lines if this dissertation is part of the Casimir PhD Series, or a similar
% research school.
% \medskip\noindent
% Casimir PhD Series, Delft-Leiden 2015-01

\noindent
ISBN XXX-XX-XXXX-XXX-X

\medskip
\noindent An electronic version of this dissertation is available at \\
\url{https://doi.org/00.0000/uuid:xxxxxxxx-xxxx-xxxx-xxxx-xxxxxxxxxxxx}.

\end{titlepage}


% The (optional) dedication can be used to thank someone or display a significant quotation.
\dedication{\epigraph{
    Some quote. \\
}{Firstname Lastname}}

\tableofcontents

\chapter*{Summary}
\addcontentsline{toc}{chapter}{Summary}
\setheader{Summary}

Computer networks have been one of the most revolutionary concepts and technologies of the last
fifty years. Currently, it is arguably impossible to imagine a world without the internet. And yet,
just five decades ago, hardly anybody knew what it even meant. Today, the first quantum computer
networks are starting to take shape, along with the promise of a future quantum internet. Quantum
networking exploits fundamental primitives of quantum mechanics --- most importantly
\emph{entanglement} --- to offer a new paradigm of connectivity, which will enhance communication
networks and bring some new exciting applications into the scene.

Quantum networking has been studied for a few years already. Nevertheless, the current state of the
art of quantum networks is somewhat comparable to that of the classical internet at the end of the
1960s: lots of interesting ideas, some experimental demonstrations, and very few reliable testbeds.
Scaling up to larger networks of quantum computers requires joint efforts of physics, mathematics,
electronics and computer science, at the very least. Bringing these disciplines together is a very
bumpy road, given that we do not yet have standard quantum physical platforms to work with, nor
universal frameworks and testbeds to validate our hypotheses against. One of the missing links
between the highly-complex physical platforms and networks and the high-level descriptions of
quantum networking applications is a framework that bridges that gap between these two, providing
platform-independent abstractions of the underlying physics to programmers and users of a quantum
network.

The goal of this thesis is threefold: discuss the requirements for such a framework of abstractions
--- which we refer to as an \acrlong{os} --- for quantum networks, propose a design for such an
\acrlong{os}, and implement and validate this design on a physical quantum network. Whilst we are
interested in measuring the performance of the \acrlong{os}, we consider our design to be
best-effort, and thus we are primarily aiming at establishing a baseline for future research in this
field. Nevertheless, we are after a fully-functional product that we hope can be used to push the
boundaries of quantum networking demonstrations, and to better understand the challenges of
designing and implementing efficient \acrlongpl{os} for quantum network nodes.

\chapter*{Samenvatting}
\addcontentsline{toc}{chapter}{Samenvatting}
\setheader{Samenvatting}

{\selectlanguage{dutch}

Computer netwerken is een van de meest revolutionaire concepten and technologieën van de afgelopen
vijftig jaar. Tegenwoordig is het zo goed als onmogelijk om je een wereld zonder het internet voor
te stellen. En toch, slechts vijf decennia geleden, wist bijna niemand wat het ook maar betekende.
Nu wordt gewerkt aan de eerste kwantum computer netwerken, met daarbij een belofte op een toekomstig
kwantum internet. Kwantum netwerken maken gebruik van de fundamentele beginselen van de
kwantummechanica --- waarbij vooral het concept van \emph{verstrengeling} van belang is --- om een
nieuw paradigma van connectiviteit aan ge bieden, welke communicatie netwerken zal versterken en
nieuwe, spannende toepassingen zal doen opbrengen.

Kwantum netwerken worden al een aantal jaar bestudeerd. Desondanks is de status van kwantum
netwerken tegenwoordig ongeveer vergelijkbaar met dat van het klassieke internet aan het eind van de
jaren 60: veel interessante ideeën, een aantal experimentele demonstraties, en weinig betrouwbare
testbeds. Opschalen naar grotere netwerken met kwantum computers heeft op zijn minst een collectieve
inspanning van natuurkundigen, wiskundigen, elektrotechnici en informatici nodig. Deze disciplines
samen laten komen is een lastige taak, aangezien we nog geen standaard fysieke platformen hebben,
nog hebben we universele kaders en testbeds om onze hypothese mee te testen. Eén van de ontbrekende
schakels tussen complexe fysieke platformen en netwerken en de abstracte omschrijvingen van kwantum
netwerk toepassingen is een kader dat het gat daartussen overbrugt door platform-onafhankelijke
abstracties van de onderliggende natuurkunde aan te bieden aan programmeurs en gebruikers van het
kwantum netwerk.

Het doel van deze scriptie heeft drie hoofdzaken: de benodigdheden van een dergelijk kader van
abstracties voor kwantum netwerken --- welke we een besturingssysteem noemen --- bediscussiëren, een
ontwerp voor zo'n besturingssysteem aandragen, en dit ontwerpen implementeren en valideren op een
fysiek kwantum netwerk. Waar we wel geïnteresseerd zijn in de prestaties van het besturingssysteem,
noemen we ons ontwerp ``best-effort'', and richten we ons vooral op het neerzetten van een
uitgangspunt voor toekomstig onderzoek in dit onderzoeksgebied. Desondanks zijn we op zoek naar een
volledig functionerend product waarvan we hopen dat het de grenzen van kwantum netwerk demonstraties
kan verleggen, en om een beter begrip te krijgen van de uitdagingen in het efficiënt implementeren
van besturingssystemen voor kwantum netwerk nodes.

}


% Use Arabic numerals for the page numbers of the chapters.
\mainmatter

% Turn on thumb indices.
\thumbtrue

\chapter{Introduction}
\label{chp:intro}

\begin{refsection}

\begin{abstract}
Chapter abstract.
\end{abstract}

\newpage

\noindent
Short chapter with informal overview of challenges and goals. This should mostly answer the
question: \emph{Why does this thesis exist?} I would defer a more technical explanation of
background and challenges to \cref{chp:background}.

\section{Networking and Quantum Networking}

Give a layperson's overview of networking-related computer science and quantum
information/networking, and informally introduce the reader to the challenges of applying the former
to the latter.

\section{Research Goals and Methodology}

Succinctly list main research questions and goals. Also, clarify scope of thesis, stating that we
took an implementation- and experiment-driven approach, rather than a more fundamental one.

\section{Thesis Outline}

List and brief overview of chapters that follow.

\printbibliography[heading=subbibintoc,title={References}]

\end{refsection}

\chapter{Background: Abstraction Layers for Quantum Networking}
\label{chp:background}

\begin{refsection}

\begin{abstract}
Chapter abstract.
\end{abstract}

\newpage

\noindent
Detailed analysis of background and challenges, to answer the question: \emph{How useful and
relevant is this thesis?} This would be an extended version of the sections ``Background'' and
``Design Considerations'' from the QNodeOS paper.

\section{Quantum Networking: Theory and Limitations}

Quantum networking basics, state of the art, ad-hoc experiments, and (lack of) sophisticated control
systems above the physical layer.

\section{Abstraction Layers for Classical Computer Networks}

Recap of what exists for classical networking, including communication protocols, network OSes,
SDNs, and what we can learn from these for quantum networking.

\section{Challenges and Design Considerations}

Limitations of quantum theory and hardware that need addressing when designing abstractions, and
generic requirements for such abstractions.

\printbibliography[heading=subbibintoc,title={References}]

\end{refsection}

\chapter
 [Architecture of an Operating System for a Quantum Network Node]
 {Architecture of an\\Operating System for a\\Quantum Network Node}
\label{chp:arch}

\begin{abstract}

And how can one design control systems that address these challenges? Cardinal design considerations
that should drive the design of an \acrlong{os} for quantum network nodes.
\end{abstract}

\blfootnote{
    This chapter is based on the preprint \citetitle{delledonne_2023_qnodeos} by
    \textcite{delledonne_2023_qnodeos}, submitted for peer review.
}

\newpage

% * Section ``QNodeOS Design'' from the QNodeOS paper.

% \lettrine{A}{...}

\printbibliography[heading=subbibintoc,title={References}]

\chapter{Entanglement Generation With a Quantum Networking Stack}
\label{chp:netstack}

\begin{refsection}

\begin{abstract}
Chapter abstract.
\end{abstract}

% \blfootnote{
%     This chapter is partly based on \emph{C. Delle Donne, Some Paper Title, Venue
%     and Year)}~\cite{somepaper}.
% }

\newpage

\noindent
This would be mostly based on the npj paper~\cite{pompili_experimental_2022}.

\section{Implementing a Quantum Link Layer Protocol}

``Quantum link layer protocol'' and ``Revised protocol'' sections from the npj paper.

\section{Real-Time Control of the Quantum Physical Layer}

``Physical layer control in real-time'' section from the npj paper.

\section{Results and Takeaways}

``Evaluation'' section from the npj paper.

\section{Conclusion}

...

\printbibliography[heading=subbibintoc,title={References}]

\end{refsection}

\chapter{Quantum Networking With an Elementary Operating System}
\label{chp:qnodeos}

\begin{refsection}

\begin{abstract}
Chapter abstract.
\end{abstract}

% \blfootnote{
%     This chapter is partly based on \emph{C. Delle Donne, Some Paper Title, Venue
%     and Year)}~\cite{somepaper}.
% }

\newpage

% \lettrine{N}{...}

\noindent
This would be mostly based on the QNodeOS paper.

\section{Goals of the Operating System}

Recap of the generic requirements from \cref{chp:background}, and list of key components that such
an OS should have.

\section{Design and Implementation}

``Design'' and ``Implementation'' sections from the QNodeOS paper.

\section{Target Applications and Evaluation Goals}

First part of the ``Evaluation'' section from the QNodeOS paper, with a more extensive explanation
of the target applications (including multitasking), and an explanation of how each of the apps,
whilst conceptually simple, test most functionalities of the OS.

\section{Results and Takeaways}

Second part of the ``Evaluation'' section from the QNodeOS paper.

\section{Conclusion}

...

\printbibliography[heading=subbibintoc,title={References}]

\end{refsection}

\chapter{Data Origin Authentication in the Quantum Networking Stack}
\label{chp:doa}

\begin{abstract}
Chapter abstract.
\end{abstract}

\blfootnote{
    This chapter is based on the preprint \citetitle{abrahams_2023_doa} by
    \textcite{abrahams_2023_doa}, submitted for peer review.
}

\newpage

% \lettrine{U}{...}

\noindent
This would be mostly based on the data origin authentication paper written with Joël.

\section{Motivation for Data Origin Authentication}

...

\section{Measuring Authentication Overhead}

...

\section{Estimating the Impact of Authentication on Quantum Link Performance}

...

\section{Conclusion}

...

\printbibliography[heading=subbibintoc,title={References}]

\chapter{Conclusion}
\label{chp:conclusion}

% \lettrine{U}{...}

The conclusions of this thesis.


% Use letters for the chapter numbers of the appendices.
\appendix

\chapter{Introduction --- The Bot's Take}
\label{app:intro}

\begin{abstract}
Writing the actual introduction to this thesis (\cref{chp:intro}) took ...
\end{abstract}

\chapter{Software Architecture of QNodeOS}
\label{app:arch}

% \lettrine{L}{...}

\chapter{NetQASM Application Snippets}
\label{app:apps}

\chapter{Soundtrack}
\label{app:soundtrack}

As a musician (?) and music enthusiast, I always try to find a soundtrack that fits whatever I am
doing or thinking in a certain situation. Very much like in a movie, soundtracks can be mash-ups of
existing songs and organic contextual sounds, or sometimes more though-out compositions. Running
applications on an experimental quantum network using an experimental operating system can be very
frustrating at times: the hardware can break, the software can be buggy, the entities running the
simulated reality~\cite{bostrom_2003_simulation} can be pesky. Therefore, those scenes are typically
backed by impromptu rambling noises --- thudding, stomping, finger-snapping, thigh-smacking. On a
good day, the satisfaction of a successful experiment turns the percussive soundtracks into a more
melodic, climax-resolving tune --- think of something like \citetitle{gorillaz_feelgoodinc} by
\textcite{gorillaz_feelgoodinc}, or perhaps \citetitle{tool_lateralus} by \textcite{tool_lateralus}
if I am feeling more (bare) metal that day.

However, I felt that the experimental acts of this thesis needed their ad-hoc piece, something that
more adequately represented my feelings towards what in retrospect looks like a fantastic journey
through physics and computers. For that matter, I am offering an alternative interpretation of the
results obtained... post-processed through this Python script... and embellished with some
human-made percussion sounds. The outcome of this final experiment is available at...

\printbibliography[heading=subbibintoc,title={References}]


% Turn off thumb indices for unnumbered chapters.
\thumbfalse

% Print glossary.
% Notice the use of \printnoidxglossary instead of \printglossary to avoid running external tools
% (https://github.com/tectonic-typesetting/tectonic/issues/704).
\glsaddall
\printnoidxglossary[type=\acronymtype,title={Glossary}]
\addcontentsline{toc}{chapter}{Glossary}
\setheader{Glossary}

\chapter*{Acknowledgments}
\addcontentsline{toc}{chapter}{Acknowledgments}
\setheader{Acknowledgments}

Acks.

\begin{flushright}
{\itshape
Carlo \\
Delft, May 2023
}
\end{flushright}

\chapter*{Curriculum Vit\ae}
\addcontentsline{toc}{chapter}{Curriculum Vit\ae}
\setheader{Curriculum Vit\ae}

% Print the full name of the author.
\makeatletter
\authors{\@firstname\ {\titleshape\@lastname}}
\makeatother

\noindent
\begin{longtable}{p{.225\textwidth} p{.70\textwidth}}
    1994/05/02 & Date of birth in Potenza, Italy
\end{longtable}

\chapter*{List of Publications}
\addcontentsline{toc}{chapter}{List of Publications}
\setheader{List of Publications}
\label{publications}

% We use the 'etaremune' environment (the reverse of 'enumerate') to get a numbered list of
% publications in reverse chronological order. If the list of authors is long, it might be useful to
% emphasize your own name with \textbf.
\begin{etaremune}
% {\small
    \item[\faFileTextO~~1.] \textbf{Carlo Delle Donne}. Some paper title and related info.
    \item[2.] \textbf{Carlo Delle Donne}. Some other paper title and related info.
% }
\end{etaremune}

\vspace{0.5cm}
\noindent
\faFileTextO~~Included in this thesis.


\end{document}
