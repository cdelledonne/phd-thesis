% To pass options to packages loaded from class file, these commands sust be called before the
% \documentclass command.

% Capitalize references when using \cref.
\PassOptionsToPackage{capitalise}{cleveref}

% Add support for inline lists with enumitem package. 
\PassOptionsToPackage{inline}{enumitem}

% Suppress vertical gap between glossary groups (entries are grouped by their first letter).
% \PassOptionsToPackage{nogroupskip}{glossaries}

% Remove dot after entry description in glossary. 
\PassOptionsToPackage{nopostdot}{glossaries}

%%%%%%%%%%%%%%%%%%%%%%%%%%%%%%%%%%%%%%%%%%%%%%%%%%%%%%%%%%%%%%%%%%%%%%%%%%%%%%%%%%%%%%%%%%%%%%%%%%%%

\documentclass[
    % print,
    % draft,
    % nativefonts,
]{dissertation}

% Large first letter of chapter using lettrine.
\usepackage{lettrine}
\renewcommand{\LettrineTextFont}{\scshape}

% Units formatted using siunitx.
\usepackage{siunitx}
\sisetup{
    group-separator={\ },
    per-mode=symbol,
}

% Braket notation.
\usepackage{braket}

% Theorems with amsthm (need to undefine \openbox to avoid re-definition error).
\let\openbox\relax
\usepackage{amsthm}

% Index-style glossary. 
\usepackage{glossary-mcols}

% Bibliography managed using BibLaTeX.
\usepackage[
    giveninits=true,     % only given name initials and not the full given name
    maxnames=50,         % max authors in bibliography
    maxcitenames=2,      % max authors when citing
    urldate=long,        % explicit URL date
    backend=biber,       % biber backend
    refsection=chapter,  % start a reference section at every \chapter command
]{biblatex}
\addbibresource{references/articles.bib}
\addbibresource{references/arxiv.bib}
\addbibresource{references/misc.bib}
\addbibresource{references/online.bib}
\addbibresource{references/proceedings.bib}

% Define bibliography entries that won't be included in list of references.
\DeclareBibliographyCategory{noprint}
\addtocategory{noprint}{abrahams_2023_doa_noprint}
\addtocategory{noprint}{delledonne_2023_qnodeos_noprint}
\addtocategory{noprint}{pompili_2022_experimental_noprint}

%%%%%%%%%%%%%%%%%%%%%%%%%%%%%%%%%%%%%%%%%%%%%%%%%%%%%%%%%%%%%%%%%%%%%%%%%%%%%%%%%%%%%%%%%%%%%%%%%%%%

% Not sure why this is defined.
\newcommand{\sparkline}[1]{$\vcenter{\hbox{\includegraphics[scale=0.04]{#1}}}$}

% Not sure why this is defined.
\newcommand*{\origrightarrow}{}
\let\oldarrow\textrightarrow
\renewcommand*{\textrightarrow}{\fontfamily{cmr}\selectfont\origrightarrow}

% Custom command: \note{description}
\newcommand{\note}[1]{{\color{red}[\textbf{NOTE:} #1]}}

% Custom command: \todo{description}
\newcommand{\todo}[1]{{\color{blue}[\textbf{TODO:} #1]}}

% Custom hyphenation.
\hyphenation{ADwin}    % Do not hyphenate ADwin
\hyphenation{QNodeOS}  % Do not hyphenate QNodeOS
\hyphenation{QDevice}  % Do not hyphenate QDevice

% Define acronyms.
% Notice the use of \makenoidxglossary instead of \makeglossary to avoid running external tools
% (https://github.com/tectonic-typesetting/tectonic/issues/704).
\makenoidxglossaries
\loadglsentries[main]{glossary}

% Customize style of glossary. 
\setglossarystyle{mcolindex}
\setlength\columnsep{1cm}
\renewcommand*{\glstreeitem}{\par\raggedright}
\renewcommand*{\glsgroupskip}{\medskip}
\renewcommand*{\glsnamefont}[1]{\textbf{#1}}

% Show section numbers next to PDF outline bookmarks.
\hypersetup{bookmarksnumbered=true}

% Disable hyperlinks from acronyms to glossary entries.
\glsdisablehyper

% Set default style for lists. 
\setlist[enumerate,itemize]{
    topsep=1ex,
    itemsep=1ex,
    parsep=0ex,
    partopsep=0ex,
    leftmargin=*
}

% Define style of inline lists.
\newlist{inlinelist}{enumerate*}{1}
\setlist[inlinelist]{label=(\arabic*)}

% Define new type of table column.
\newcolumntype{Y}{>{\raggedleft\arraybackslash}X}

% Define custom theorem environment for examples.
\theoremstyle{definition}
\newtheorem{example}{Example}[section]

% Redefine \paragraph command to reduce vertical and horizontal spacing and to add a period after
% the heading. References:
% - https://tex.stackexchange.com/questions/40943/new-line-and-no-indent-after-paragraph
% - https://tex.stackexchange.com/questions/328355/add-a-period-after-each-paragraph-title
\makeatletter
\renewcommand\paragraph{\@startsection{paragraph}{4}{\z@}%
    {1ex \@plus 1ex \@minus 0.2ex}%
    {-0.5em}%
    {\normalfont\normalsize\bfseries\maybe@addperiod}%
}
\newcommand{\maybe@addperiod}[1]{%
    #1\@addpunct{.}%
}
\makeatother

%%%%%%%%%%%%%%%%%%%%%%%%%%%%%%%%%%%%%%%%%%%%%%%%%%%%%%%%%%%%%%%%%%%%%%%%%%%%%%%%%%%%%%%%%%%%%%%%%%%%

\begin{document}

% Specify the title and author of the thesis. This information will be used on the title page (in
% chapters/meta/title.tex) and in the metadata of the final PDF.
\title{Software Abstractions for Programmable Quantum Network Nodes}
\author{Carlo}{Delle Donne}

% Use Roman numerals for the page numbers of the title pages and table of contents.
\frontmatter

\begin{titlepage}

\begin{center}

% Extra whitespace at the top.
\vspace*{2\bigskipamount}

% Print the title.
{\makeatletter
\titlestyle\bfseries\LARGE\@title
\makeatother}

% Print the optional subtitle.
{\makeatletter
\ifx\@subtitle\undefined\else
    \bigskip
    \titlefont\titleshape\Large\@subtitle
\fi
\makeatother}

\end{center}

\cleardoublepage
\thispagestyle{empty}

\begin{center}

% The following lines repeat the previous page exactly.

\vspace*{2\bigskipamount}

% Print the title.
{\makeatletter
\titlestyle\bfseries\LARGE\@title
\makeatother}

% Print the optional subtitle.
{\makeatletter
\ifx\@subtitle\undefined\else
    \bigskip
    \titlefont\titleshape\Large\@subtitle
\fi
\makeatother}

% Uncomment the following lines to insert a vertically centered picture into the title page.
%\vfill
%\includegraphics{title}
\vfill

% Apart from the names and dates, the following text is dictated by the promotieregelement.

{\Large\titlefont\bfseries Dissertation}

\bigskip
\bigskip

for the purpose of obtaining the degree of doctor \\
at Delft University of Technology, \\
by the authority of the Rector Magnificus, prof.~dr.~ir.~T.H.J.J.~van~der~Hagen, \\
chair of the Board for Doctorates, \\
to be defended publicly on \\
Tuesday, 27 June 2023 at 12:30 o'clock

\bigskip
\bigskip

by

\bigskip
\bigskip

% Print the full name of the author.
\makeatletter
{\Large\titlefont\bfseries\@firstname\ \titleshape{\MakeUppercase{\@lastname}}}
\makeatother

\bigskip
\bigskip

Master of Science in Embedded Systems, \\
Delft University of Technology, the Netherlands, \\
born in Potenza, Italy.

% Extra whitespace at the bottom.
\vspace*{2\bigskipamount}

\end{center}

\clearpage
\thispagestyle{empty}

% The following line is dictated by the promotieregelement.
\noindent
This dissertation has been approved by the promotors.

\bigskip\noindent
Composition of the doctoral committee:

% List the committee members, starting with the Rector Magnificus and the promotor(s) and ending
% with the reserve members.
\medskip\noindent
\begin{tabular}{p{4.5cm}l}
    Rector Magnificus, & chairperson \\
    Prof.\ dr.\ S.\ D.\ C.\ Wehner, & promotor \\
    Dr.\ P.\ Pawełczak, & promotor \\
\end{tabular}

\medskip\noindent
\begin{tabular}{p{4.5cm}l}
    \mbox{\emph{Independent members:}} & \\
    Prof.\ dr.\ A.\ van Deursen,    & Delft University of Technology \\
    Dr.\ L.\ Y.\ Chen,              & Delft University of Technology \\
    Prof.\ dr.\ R.\ Van Meter,      & Keio University, Japan \\
    Dr.\ C.\ G.\ Almudéver,         & Technical University of Valencia, Spain \\
    Prof.\ dr.\ K.\ G.\ Langendoen, & Delft University of Technology, reserve member \\
\end{tabular}

% Include the following disclaimer for committee members who have contributed to this dissertation.
% Its formulation is again dictated by the promotieregelement.
% \medskip\noindent
% Prof.\ dr.\ ir.\ R.\ Hanson has contributed to the creation of this thesis.

\vfill

% Here you can include the logos of any institute that contributed financially to this dissertation.
\begin{center}
\noindent
\includegraphics[height=0.5in]{figures/logos/tudelft.pdf}
\hfill
\includegraphics[height=0.5in]{figures/logos/qutech.pdf}
\hfill
\includegraphics[height=0.5in]{figures/logos/kavli.pdf}
\\ \vspace{2\baselineskip}
\includegraphics[height=0.5in]{figures/logos/nwo.pdf}
\hfill
\includegraphics[height=0.5in]{figures/logos/qia.pdf}
\hfill
\includegraphics[height=0.5in]{figures/logos/erc.pdf}
\hfill
\includegraphics[height=0.5in]{figures/logos/casimir.pdf}
\end{center}

\vfill

\noindent
\begin{tabular}{@{}p{0.2\textwidth}@{}p{0.8\textwidth}}
    \textit{Keywords:} & Quantum networks, operating systems \\[\medskipamount]
    \textit{Cover:} & 無音の暴君 (\emph{Silent tyrant}), Mari Nakagawa \\[\medskipamount]
    \textit{Style:} & TU Delft House Style, with modifications by Moritz Beller \\
    & \url{https://github.com/Inventitech/phd-thesis-template} \\[\medskipamount]
    \textit{Printed by:} & Ipskamp Printing \\
    & \url{https://www.ipskampprinting.nl/proefschriften/} \\[\medskipamount]
\end{tabular}

\vspace{\bigskipamount}

% Copyrighting this is questionable, because large parts of the thesis have already been published
% with the copyright resigning with the publisher.
% \noindent
% Copyright \textcopyright\ 2015 by A.~Einstein

% Uncomment the following lines if this dissertation is part of the Casimir PhD Series, or a similar
% research school.
% \medskip\noindent
% Casimir PhD Series, Delft-Leiden 2015-01

\noindent
ISBN XXX-XX-XXXX-XXX-X

\medskip
\noindent An electronic version of this dissertation is available at \\
\url{https://doi.org/00.0000/uuid:xxxxxxxx-xxxx-xxxx-xxxx-xxxxxxxxxxxx}.

\end{titlepage}


% The (optional) dedication can be used to thank someone or display a significant quotation.
\dedication{\epigraph{
    Some quote. \\
}{Firstname Lastname}}

\tableofcontents

\chapter*{Summary}
\addcontentsline{toc}{chapter}{Summary}
\setheader{Summary}

Computer networks have been one of the most revolutionary concepts and technologies of the last
fifty years. Currently, it is arguably impossible to imagine a world without the internet. And yet,
just five decades ago, hardly anybody knew what it even meant. Today, the first quantum computer
networks are starting to take shape, along with the promise of a future quantum internet. Quantum
networking exploits fundamental primitives of quantum mechanics --- most importantly
\emph{entanglement} --- to offer a new paradigm of connectivity, which will enhance communication
networks and bring some new exciting applications into the scene.

Quantum networking has been studied for a few years already. Nevertheless, the current state of the
art of quantum networks is somewhat comparable to that of the classical internet at the end of the
1960s: lots of interesting ideas, some experimental demonstrations, and very few reliable testbeds.
Scaling up to larger networks of quantum computers requires joint efforts of physics, mathematics,
electronics and computer science, at the very least. Bringing these disciplines together is a very
bumpy road, given that we do not yet have standard quantum physical platforms to work with, nor
universal frameworks and testbeds to validate our hypotheses against. One of the missing links
between the highly-complex physical platforms and networks and the high-level descriptions of
quantum networking applications is a framework that bridges that gap between these two, providing
platform-independent abstractions of the underlying physics to programmers and users of a quantum
network.

The goal of this thesis is threefold: discuss the requirements for such a framework of abstractions
--- which we refer to as an \acrlong{os} --- for quantum networks, propose a design for such an
\acrlong{os}, and implement and validate this design on a physical quantum network. Whilst we are
interested in measuring the performance of the \acrlong{os}, we consider our design to be
best-effort, and thus we are primarily aiming at establishing a baseline for future research in this
field. Nevertheless, we are after a fully-functional product that we hope can be used to push the
boundaries of quantum networking demonstrations, and to better understand the challenges of
designing and implementing efficient \acrlongpl{os} for quantum network nodes.

\chapter*{Samenvatting}
\addcontentsline{toc}{chapter}{Samenvatting}
\setheader{Samenvatting}

{\selectlanguage{dutch}

Computer netwerken is een van de meest revolutionaire concepten and technologieën van de afgelopen
vijftig jaar. Tegenwoordig is het zo goed als onmogelijk om je een wereld zonder het internet voor
te stellen. En toch, slechts vijf decennia geleden, wist bijna niemand wat het ook maar betekende.
Nu wordt gewerkt aan de eerste kwantum computer netwerken, met daarbij een belofte op een toekomstig
kwantum internet. Kwantum netwerken maken gebruik van de fundamentele beginselen van de
kwantummechanica --- waarbij vooral het concept van \emph{verstrengeling} van belang is --- om een
nieuw paradigma van connectiviteit aan ge bieden, welke communicatie netwerken zal versterken en
nieuwe, spannende toepassingen zal doen opbrengen.

Kwantum netwerken worden al een aantal jaar bestudeerd. Desondanks is de status van kwantum
netwerken tegenwoordig ongeveer vergelijkbaar met dat van het klassieke internet aan het eind van de
jaren 60: veel interessante ideeën, een aantal experimentele demonstraties, en weinig betrouwbare
testbeds. Opschalen naar grotere netwerken met kwantum computers heeft op zijn minst een collectieve
inspanning van natuurkundigen, wiskundigen, elektrotechnici en informatici nodig. Deze disciplines
samen laten komen is een lastige taak, aangezien we nog geen standaard fysieke platformen hebben,
nog hebben we universele kaders en testbeds om onze hypothese mee te testen. Eén van de ontbrekende
schakels tussen complexe fysieke platformen en netwerken en de abstracte omschrijvingen van kwantum
netwerk toepassingen is een kader dat het gat daartussen overbrugt door platform-onafhankelijke
abstracties van de onderliggende natuurkunde aan te bieden aan programmeurs en gebruikers van het
kwantum netwerk.

Het doel van deze scriptie heeft drie hoofdzaken: de benodigdheden van een dergelijk kader van
abstracties voor kwantum netwerken --- welke we een besturingssysteem noemen --- bediscussiëren, een
ontwerp voor zo'n besturingssysteem aandragen, en dit ontwerpen implementeren en valideren op een
fysiek kwantum netwerk. Waar we wel geïnteresseerd zijn in de prestaties van het besturingssysteem,
noemen we ons ontwerp ``best-effort'', and richten we ons vooral op het neerzetten van een
uitgangspunt voor toekomstig onderzoek in dit onderzoeksgebied. Desondanks zijn we op zoek naar een
volledig functionerend product waarvan we hopen dat het de grenzen van kwantum netwerk demonstraties
kan verleggen, en om een beter begrip te krijgen van de uitdagingen in het efficiënt implementeren
van besturingssystemen voor kwantum netwerk nodes.

}


% Use Arabic numerals for the page numbers of the chapters.
\mainmatter

% Turn on thumb indices.
\thumbtrue

\chapter{Introduction}
\label{chp:intro}

\lettrine{Q}{uite} possibly, the word ``quantum'' is one of the most trending, and perhaps one of
the most abused, of the last decade, at least in the context of science and technology. Attaching it
to the name of tech products makes them sound more advanced. Even just a capital ``Q'' in a brand's
name suggests superiority. When I embarked on my Ph.D. at the end of 2018, I started to comprehend
how challenging it is to develop actual quantum technology. Nevertheless, then next thing I learned
is that Mozilla had already dropped a quantum (?) browser~\cite{firefox_quantum}.

However, quantum physics is not just an otherworldly theory from science fiction books, nor just a
catchy name for 21st-century consumer electronics. Since the formulation of the theory of quantum
mechanics in the early decades of the 20th century, researchers and enthusiasts have been looking
for how to make use of these physical properties, particularly in the fields of electronics,
information processing and telecommunications. Today, we know of a handful of applications that
exploit the axioms of quantum information theory to achieve something that was though to be very
hard, sometimes even impossible. Some of the most well-known use cases include fast resolution of
computational problems --- like integer factorization~\cite{shor_1994_algorithms} and database
searching~\cite{grover_1996_search} --- and efficient and secure communication schemes --- for
instance quantum key distribution~\cite{bennett_2014_bb84, ekert_1991_e91} and superdense
coding~\cite{bennett_1992_communication}.

\emph{Quantum networking}, a new paradigm of telecommunications, seeks to enhance --- not replace
--- our current internet technology to provide new functionalities that are impossible to attain
with purely classical communications. Novel applications include security-enhancing communication
schemes such as \acrfull{qkd}~\cite{bennett_2014_bb84, ekert_1991_e91}, advanced clock
synchronization routines~\cite{komar_2014_clocks}, distributed consensus
protocols~\cite{benor_2005_byzantine}, distributed sensing~\cite{gottesman_2012_telescope}, and
secure cloud quantum computing~\cite{broadbent_2009_ubqc, childs_2005_secure_qc}. Even though
quantum communications are an established reality, and their potential applications have garnered
attention from industry and research institutes, the general public is still rather puzzled, on
average, when someone tries to pitch their research on quantum networking. The above list of
applications does not necessarily appeal to the masses, which are still sometimes stuck with the
hope that quantum teleportation will instantly get them to the Bahamas~\cite{xkcd_teleportation}.
Nonetheless, the community of quantum networking researchers is not discouraged by this mismatch in
expectations, as it hopes that more applications will be devised once the technology becomes more
widespread and available to more ``consumers''. After all, we were also not aware of all the
possible uses of the classical internet when it was first developed. Yet, today the internet means
instantaneous access to low-cost clothes, scenes of hilarious felines, and --- for some --- tapes of
bare bodies engaging in intimate action on camera.

One of the most frequently asked questions about quantum technology is: \emph{When can we use it?}
If more people had access to quantum networks, they would perhaps come up with more ideas for useful
quantum networking applications. Thus, what is missing before we can deploy quantum networks
consisting of a useful number of nodes? What are the main limitations we are facing? How can we
overcome them? Not surprisingly, the answers to these questions are complicated. There is a cauldron
of fundamental and theoretical limitations, technological hardware obstacles, and computer science
puzzles that hinder the success of quantum networking. Fortunately, though, there is an increasing
community of passionate researchers trying to study these limitations, build better hardware, and
solve these puzzles. In this thesis, we will address some of the challenges of managing the activity
and the resources of a quantum network node, from an operating system's perspective. We thus aim to
answer a tiny fraction of the research questions in the field of quantum networking, and to lay
another brick in the construction of \emph{scalable}, \emph{controllable}, and \emph{configurable}
quantum networks.

The remainder of this chapter walks the reader through basic networking concepts and quantum
networking challenges, lists the research questions we aim to address, and provides an outline of
the rest of the thesis.

\section{Networking and Quantum Networking}

Digital communication networks have come a long way from the early days of the \textsc{arpanet} ---
one of the most important precursors of today's internet. Back in 1969, one of the most advanced
networks of computers consisted of just \emph{four} nodes. In 1981, as the global network grew
larger, networking researchers standardized the \acrfull{ipv4}, which allowed up to \num{4} billion
devices to have their own address on the public internet~\cite{rfc_791}. Soon after, it became clear
that the pool of available \acrshort{ipv4} addresses was going to be depleted sooner than later ---
which happened in 2011~\cite{icann_2011}. In 2023, there are \num{3.6} devices connected to the
internet per capita, as estimated by \citeauthor{cisco_2020} in 2020~\cite{cisco_2020}.

Scaling up from a four-node experiment to the massive networks of the 21st century was no easy feat
of course. This was made possible by advancements in various fields, including networking hardware,
traffic engineering, and network programmability. You would most likely not be able to download a
digital copy of this thesis in a fraction of a second if it were not for \unit{\tera\bit\per\second}
network switches, a diverse spectrum of routing protocols, and \acrlong{sdn}. Nevertheless,
classical networking was already appealing in its infant stages, for the simple reason that even a
small network of nodes can accomplish tasks that would not be attainable without it --- in the case
of the \textsc{arpanet}, sending simple pieces of text over large distances almost instantaneously.
The applications of quantum networking are not dissimilar to their classical networking counterpart,
in that some of them can be useful and effective on small quantum networks already, whilst other use
cases require more powerful nodes and more complex networks~\cite{wehner_2018_stages}.

Even though hardware and software advancements were essential to the betterment of the internet,
there are a few basic design ingredients that have been there since the dawn of classical networking
and that have made these technological leaps even possible. For instance, it was immediately clear
that data was to be grouped and transmitted according to the \emph{packet switching} model, to
maximize network utilization and to allow for dynamic routing decisions. As another example,
networking protocols have been organized in \emph{abstraction layers} since the very beginning of
computer networks, so as to encapsulate network functionalities in services of increasing
user-friendliness. When designing quantum networks and networking services, one should draw
inspiration from classical networking principles and avoid monolithic designs and software
architectures that would render the system hard to scale.

Then, how similar are quantum networks and classical networks? Which are the challenges that they
have in common, and which ones are exclusive to quantum networking? How much can we capitalize on
the vast body of classical networking literature? In principle, quantum networks are just networks
with a special physical layer, which, albeit more technologically complex, could be abstracted away
and encapsulated into an ad-hoc networking protocol. This simplification would, however, disregard
some fundamental limitations that are inherent to quantum information, as well as the imperfect
nature of near-term quantum hardware. Standard networking routines like signal amplification,
classical error correction, and data retransmission would not work in quantum networks --- one
fundamental theorem of quantum mechanics states that \emph{an arbitrary quantum state cannot be
cloned}~\cite{wootters_1982_nocloning, dieks_1982_communication}. Moreover, quantum states are
subject to \emph{decoherence}, a physical process whereby the quality of the stored information
degrades over time and due to external interferences. Thus, not only do computation and networking
delays affect throughput and latency, but they also exact a toll on the quality of the service,
effectively determining whether a certain application produced meaningful results or not. Whilst
decoherence can be worked around with more sophisticated quantum hardware and control algorithms,
the no-cloning theorem is a hard limit on what one can do with quantum information, and thus we
cannot design quantum networking protocols assuming quantum information can be freely copied and
stored indefinitely.

Without being too speculative, we can argue that we cannot just encapsulate the requirements of
quantum networking into a specialized physical layer. Reusing classical networking techniques and
protocols as they are would fall short of the aforementioned challenges posed by quantum mechanics.
Nonetheless, many of the questions that drove the classical networking research community can be of
inspiration for analyzing requirements and limitations of quantum networks. Examples of such
questions are: \emph{Can we organize quantum information into packets? Do we need to resort to path
reservation for quantum communications? In which situations can we tolerate local and network
latency? How do we organize and layer quantum networking protocols and services? What metrics do we
look at when designing routing protocols? Can we improve performance and quality of service with the
employment of a software-defined control plane?}

\section{Research Goals}

Perhaps disappointingly, but unsurprisingly too, this thesis will not try to answer all the research
questions from the previous section. The good news is that there are many ongoing efforts from
various scientists that are looking into these questions. Most of the times, however, researchers
have to validate their designs on a simulated quantum network, given that we do not yet have access
to mid-scale testbeds consisting of more than a handful of nodes --- the most advanced of which
features three interconnected devices~\cite{pompili_2021_multinode}. On the other hand, we need a
framework to evaluate quantum networking protocols and applications on real quantum networks, to
help us verify our assumptions and simulation results even in the early stages of this research
field. In this thesis, we will discuss design considerations for such a framework, design and
implement a rudimentary instance of it, and evaluate our design and implementation on a quantum
network.

The goal of this thesis is to provide a framework that facilitates the experimental investigation of
quantum networking-related research questions, and that can help researchers learn about the
behavior of quantum communication applications without having to delve into the complexity of
running and managing the underlying network and the interaction of the nodes with it. We refer to
this framework as an \emph{\acrlong{os}} (\acrshort{os}), as its goal is to abstract and manage
quantum physical processes and resources to provide a user-friendly interface to the application.
More specifically, we will address the following questions:

\begin{enumerate}[label={Q\arabic*.}]
    \item \emph{What goals should an \acrshort{os} for quantum network nodes achieve?} We explore
          challenges, requirements and goals that one should consider when designing such an
          \acrshort{os}, whose overarching objective is to bridge the gap between high-level user
          applications and low-level quantum networking hardware.
    \item \emph{What does an architecture for such an \acrshort{os} look like?} We propose the first
          proof-of-principle architecture for an \acrshort{os} for quantum network nodes. The
          architecture ensures the \acrshort{os} can be deployed on various quantum platforms,
          provides means to manage resources and schedule operations, and allows running multiple
          quantum networking applications concurrently.
    \item \emph{What is the performance of the \acrshort{os}'s quantum network stack?} We revise and
          implement state-of-the-art quantum network protocols to be integrated in the proposed
          \acrshort{os}, and evaluate their performance for a basic networking feature: entanglement
          delivery.
    \item \emph{What is the performance of the whole \acrshort{os}?} We implement the proposed
          architecture, and evaluate its functioning and performance for some basic quantum
          networking applications, including scenarios of concurrent execution of multiple
          applications.
    \item \emph{How would \acrlong{doa} affect the performance of a quantum link?} We evaluate, this
          time in simulation, what penalty would be incurred in the performance of entanglement
          generation if the quantum network stack would communicate over an authenticated classical
          channels, as opposed to exchanging non-authenticated classical messages.
\end{enumerate}

This work is aimed at designing, implementing and evaluating a ``product'' that, although
experimental and not production-ready, is a fully-functional research tool, and as such is ready to
be reused and adapted with little effort by anyone who is interested in experimenting with quantum
networking protocols and applications on real networks. Our implementation-driven research does not
intend to produce the best protocols and algorithms for the control of quantum network nodes ---
rather, it wishes to establish a baseline for such a system, and a framework to study and test more
advanced versions of its components. To demonstrate the applicability of our tool, we evaluate the
\acrshort{os} on a small state-of-the-art quantum network based on \acrlong{nv} centers in
diamond~\cite{pompili_2021_multinode}, deployed in a laboratory environment.

\begin{table}[t]
    \centering
    \begin{tabularx}{\linewidth}{Xl}
        \toprule
        \textbf{Object}                                  & \textbf{Location}                                                                            \\
        \midrule
        Datasets for \cref{chp:netstack}                 & \href{https://doi.org/10.4121/16912522}{\textsc{doi}: \texttt{10.4121/16912522}}             \\
        Datasets for \cref{chp:qnodeos}                  & \href{https://doi.org/xx.xxxx/xxxxxxxx}{\textsc{doi}: \texttt{TBD}}                          \\
        Datasets for \cref{chp:doa}                      & \href{https://doi.org/yy.yyyy/yyyyyyyy}{\textsc{doi}: \texttt{TBD}}                          \\
        Data analysis for \cref{chp:netstack}            & With data (\href{https://doi.org/10.4121/16912522}{\textsc{doi}: \texttt{10.4121/16912522}}) \\
        Data analysis for \cref{chp:qnodeos}             & With data (\href{https://doi.org/xx.xxxx/xxxxxxxx}{\textsc{doi}: \texttt{TBD}})              \\
        Data analysis for \cref{chp:doa}                 & With data (\href{https://doi.org/yy.yyyy/yyyyyyyy}{\textsc{doi}: \texttt{TBD}})              \\
        Application \acrshort{sdk} (NetQASM)             & \url{https://github.com/QuTech-Delft/netqasm}                                                \\
        Applications for \cref{chp:netstack,chp:qnodeos} & \url{https://gitlab.tudelft.nl/TBD/}                                                         \\
        \bottomrule
    \end{tabularx}
    \caption{
        Location of experimental data and software supporting this thesis.
        \note{Add missing references}
    }
    \label{tab:data-and-soft}
\end{table}

\section{Data and Software Availability}

The datasets that support this thesis are made public and available in the online data repository
4TU.ResearchData. The software to analyze such data is also made public. The application
\acrfull{sdk} is open-sourced on GitHub. The application scripts themselves are also open-sourced.
Finally, the software for the \acrlong{os} implemented in this thesis is part of a larger project
that is still under development, and this not currently available for external use. Refer to
\cref{tab:data-and-soft} for pointers to data and software.

\section{Thesis Outline}

The rest of this thesis provides some useful preliminary knowledge and then tackles the research
questions listed above. \Cref{chp:background} offers some background on quantum information, quantum
networking and classical networking, and recaps the main challenges involved. \Cref{chp:arch}
outlines the most important design considerations that serve as the basis for the design of our
\acrshort{os} (question Q1), and describes our proposal for the architecture of a quantum network
node's \acrshort{os} (question Q2). \Cref{chp:netstack} evaluates the performance of the quantum
network stack integrated in the \acrshort{os} for entanglement generation (question Q3).
\Cref{chp:qnodeos} validates and benchmarks the full \acrshort{os} against a set of simple quantum
networking applications, also when running multiple of these concurrently (question Q4).
\Cref{chp:doa} quantifies, this time in simulation, the effect of authenticating classical messages
in the quantum network stack on the performance of a quantum link (question Q5). Finally,
\cref{chp:conclusion} concludes this thesis and reflects upon future steps.

\printbibliography[heading=subbibintoc,title={References},notcategory=noprint]

\chapter
 [Quantum Networking: Background and Challenges]
 {Quantum Networking:\\Background and Challenges}
\label{chp:background}

\begin{abstract}
Quantum networks are a fundamentally new paradigm of telecommunications, destined to enhance our
classical networking primitives to achieve unprecedented tasks in various areas of
communications and sensing. But how do quantum networks differ from their classical counterpart?
What makes them special, and at the same time challenging to manage? This chapter provides useful
background knowledge on quantum networking and recaps the primary challenges thereof.
\end{abstract}

\blfootnote{
    This chapter is based on the preprint: \fullcite{delledonne_2023_qnodeos_noprint}. \note{Add
    link to arXiv when submitted}
}

\newpage

\lettrine{U}{biquitous} internet connectivity has already unlocked a plethora of applications that
were not even conceived just years ago. Similarly, a future quantum
internet~\cite{kimble_2008_quantum, wehner_2018_stages} aims to connect quantum devices --- the
\emph{end nodes} --- over large distances, in order provide new internet functionality that is
impossible to achieve using solely classical communication. Examples of applications running on end
nodes include security-enhancing protocols such as \acrfull{qkd}~\cite{ekert_1991_e91,
bennett_2014_bb84}, improved clock synchronization~\cite{komar_2014_clocks}, support for distributed
sensing~\cite{gottesman_2012_telescope} and distributed systems~\cite{benor_2005_byzantine}, as well
as secure quantum computing in the cloud~\cite{broadbent_2009_ubqc, childs_2005_secure_qc}.

To run a general quantum networking application, the end nodes' hardware-software system needs to be
capable of performing certain actions, as summarized in \cref{fig:quantum-internet}. First, nodes
must be able to establish a quantum connection by generating quantum \emph{entanglement} between
them. Entanglement is a special property of at least two quantum bits --- or \emph{qubits} --- one
held by each end node. The entangled qubits are often measured directly by the application, or may
be used to transmit data qubits from one end node to the other through
teleportation~\cite{bennett_1993_teleportation}. Alongside entanglement, end nodes must be capable
of executing local quantum operations on the qubits held by an end node, that is, quantum
\emph{gates} and quantum \emph{measurements}. For simple quantum applications such as secure
communication~\cite{ekert_1991_e91, bennett_2014_bb84} it is sufficient to produce entanglement and
then perform a local measurement at each end node. However, for more complex quantum applications
--- enabled at higher stages of quantum internet development~\cite{wehner_2018_stages} --- local
operations can include the execution of quantum gates and in fact full quantum computation on a
quantum processor. Finally, next to such quantum actions, most quantum applications known to date
require local classical processing, as well as classical communication between the end nodes.

\begin{figure}[b]
    \centering
    \includegraphics[width=0.6\linewidth]{figures/quantum-internet.pdf}
    \caption{
        A quantum networking application consists of separate programs running at two or more end
        nodes that communicate via classical message passing and quantum entanglement. Local
        operations include quantum operations (gates and measurements) as well as classical
        processing.
    }
    \label{fig:quantum-internet}
\end{figure}

Abstractly, a quantum networking application consists of multiple programs, each running on one of
the end nodes. The distinct programs only interact with one another by means of entanglement
generation and classical communication. This allows a programmer to realize security-sensitive
applications just as in the classical domain, but prohibits a global orchestration of the quantum
execution as one might do in quantum computing. The case of secure quantum computing in the
cloud~\cite{broadbent_2009_ubqc, childs_2005_secure_qc} is an example of a quantum networking
application, schematically depicted in \cref{fig:app-struct}. In blind quantum computing, a client
node wants to perform a computation on a remote server node, the latter being a powerful quantum
computer, without the server learning anything about the computation. Blind quantum computing
illustrates the need for a continuing interaction between the classical and quantum parts of the
execution, such as waiting for a message from a remote client before continuing the quantum
execution at the server. It also highlights the need for both classical and quantum state to be kept
alive, for example such that future quantum instructions can be executed depending on messages from
remote end nodes. This is in sharp contrast to quantum computing applications, where one can process
the entire quantum execution in a single batch.

\begin{figure}[t]
    \centering
    \includegraphics[width=\linewidth]{figures/app-struct.pdf}
    \caption{
        Structure of a typical quantum network application (blind quantum
        computation~\cite{broadbent_2009_ubqc, childs_2005_secure_qc}), which consists of
        interleaved quantum processing blocks and classical processing blocks. Quantum processing
        blocks include local quantum operations (gates and measurements, yellow boxes) and network
        operations (entanglement generation, blue boxes). The execution of some classical and
        quantum blocks might be conditional on classical and quantum data coming from previous
        blocks. Qubit states in quantum blocks may have to persist (``Persist qubits'') to be used
        in later quantum blocks (``Use qubits''), e.g. following the reception of classical messages
        from the remote node.
    }
    \label{fig:app-struct}
\end{figure}

Up to now, demonstrations of quantum networking beyond \acrshort{qkd} focused on hardware
realizations. Different types of end node quantum hardware have been realized, ranging from simple
photonic devices on which the only operation is a measurement~\cite{vallone_2015_satellite,
yin_2017_satellite}, to fully-fledged quantum processors with a network
interface~\cite{bernien_2013_heralded, humphreys_2018_delivery, pompili_2021_multinode,
moehring_2007_ion_traps, reiserer_2015_neutral_atoms}. The largest quantum network linking quantum
processors to date connects three nodes~\cite{pompili_2021_multinode} based on nitrogen-vacancy
centers in diamond (\acrshort{nv} centers) at the physical layer. Demonstrations of applications
beyond \acrshort{qkd} have been performed using several photonic
devices~\cite{barz_2012_demonstration, thalacker_2019_anonymous, bozzio_2020_coin,
ng_2012_noisystorage}. Central to all these demonstrations is that the software to control the
hardware was specific to the experiment setup, written to perform one single task (the experiment
itself) and programmed into low-level control devices. In fact, often applications were not even
actually fully realized towards a user, and instead they were meant to show that the hardware is in
principle good enough for that specific application~\cite{zhang_2022_diqkd,
liu_2022_photonic_diqkd}.

In order to advance quantum networks from a physics experiment to fully-fledged systems, we need a
combined software-hardware system that is built as a series of abstraction layers. These
abstractions should expose a simple interface for the user to write applications in high-level,
platform-independent software, and be able to interact with a variety of candidate platforms for
future quantum network hardware. When designing such a system, many challenges arise (refer to
\cref{sec:background:challenges} for details), which can be roughly classified into three areas.
First, there exist \emph{fundamental differences} between classical and quantum communication. A
example of this is the concept of heralded entanglement generation, which requires coordinated
actions by both nodes involved --- that is, the operations to produce entanglement need to be
scheduled at both nodes at the same time. Second, the \emph{technological limitations} of near-term
quantum devices impose stringent demands on the performance of such a system. One example is that
the same quantum device is used for processing as well as networking, which implies that local
operations cannot be scheduled independently of network operations --- which in turn depend on the
remote node. Finally, we remark that, unlike in the study of classical operating systems, which take
advantage of the existence of advanced computer architectures defining a specific interaction of
software and hardware, there exists \emph{no general low-level quantum processor architecture}.
Ideally, our system should be able to operate under the stringent constraints imposed by current
technological limitations, but should also not be tailored to near-term quantum devices only.

\section{Background}
\label{sec:background:background}

Here, we define some basic concepts of quantum networking hardware and performance metrics,
necessary to understand the remainder of this chapter and this thesis. We refer the reader to the
book \citetitle{nielsen_chuang_2002} by \textcite{nielsen_chuang_2002} for a general introduction to
quantum information, and to the article \citetitle{dahlberg_2019_egp} by
\textcite{dahlberg_2019_egp} for more information on quantum networking.

\paragraph{Quantum network nodes}

Generally speaking, a quantum network node is a quantum processor (or device) with an optical
interface for external communication (entanglement). The processor can perform operations on one or
more qubits. Local quantum operations range from simple qubit
measurements~\cite{vallone_2015_satellite, yin_2017_satellite} to universal quantum
computation~\cite{bernien_2013_heralded, humphreys_2018_delivery, pompili_2021_multinode,
moehring_2007_ion_traps, reiserer_2015_neutral_atoms}. Quantum networking operations allow certain
qubits to produce entanglement with a remote node. In practice, only some types of qubits are suited
for entanglement generation --- we refer to such qubits as \emph{communication qubits} --- and only
these qubits have an optical interface to the outside world. Other types of qubits --- referred to
as \emph{storage qubits} --- are instead more suited for storing quantum states for longer times (up
to seconds in some cases~\cite{abobeih_2018_one_sec, bradley_2019_one_min}). Often, storage qubits
can also be used to process quantum information directly. On some quantum devices instead, for
instance \acrlong{nv} centers in diamond (\acrshort{nv} centers), communication qubits are also the
main gateway for local quantum gates, meaning that most processing operations need to step though
these qubits, and thus their usage needs to be shared between local quantum computation and
entanglement generation. What operations can be performed on what types of qubits depends on the
specific quantum device. We remark that, at this stage of technological development, qubits, as well
as any quantum operations applied on them, are not perfect. In fact, their quality even depends on
the specific device sample being used.

\paragraph{Timing constraints}

Quantum devices are generally controlled using a variety of classical signal generators, depending
on the quantum device itself. For instance, in \acrshort{nv} centers, the quantum device is
controlled using microwave as well as laser pulses. The device-level control must satisfy hard
real-time constraints and timing precision --- nanosecond precision with sub-nanosecond jitter ---
and is realized using waveform generators, lasers, and custom electronics assisted by a dedicated
microcontroller. Entanglement generation between two nodes connected by an optical fiber also
requires the same scale of timing synchronization between the two devices~\cite{dahlberg_2019_egp,
pompili_2022_experimental}. The low-level control of other quantum platforms is realized
similarly~\cite{moehring_2007_ion_traps, reiserer_2015_neutral_atoms}. On top of those constraints,
qubits have a limited lifetime --- the states they hold must be processed before they become
invalid. On some devices, qubit lifetimes have been shown to exceed one
second~\cite{abobeih_2018_one_sec}. Nevertheless, the quality of a qubit state is not constant
throughout its lifetime, but it \emph{decoheres} (becomes worse) at a certain rate. Qubit lifetimes
and decoherence, however, are technological limitations, rather than fundamental ones, and are
expected to become more tractable in the future.

\paragraph{Performance metrics}

Next to standard classical performance metrics such as latency and throughput, the performance of
quantum networking applications hinges on the quality of the quantum execution too. In the quantum
networking domain, it is not generally an objective to eliminate all errors towards the application
level~\cite{dahlberg_2019_egp, vardoyan_2022_netarch}, and hence the performance of any operating
system for quantum network nodes would be measured by the execution quality, and by the trade-offs
with classical performance metrics~\cite{dahlberg_2019_egp, vardoyan_2022_netarch}. This quantum
quality is generally measured by the quantum \emph{fidelity} $F \in [0,1]$, where a higher value
corresponds to higher quality. For a quantum state, $F$ measures the quality with respect to an
ideal state. For a quantum gate or measurement, it measures the quality of execution, averaged over
all possible states that it could be applied to. For a specific application, $F$ can be translated
into its quantum performance.

\section{Challenges}
\label{sec:background:challenges}

Whilst the high-level goals for an operating system for quantum nodes mimic those of a classical
operating system, we face a number of general challenges inherent to (near-term) quantum network
nodes and network applications:
%
\begin{inlinelist}
    \item limited available qubits, which imposes strict limits on the processing, networking, and
          storage capabilities of networking nodes;
    \item limited qubit lifetimes, which imposes strict deadlines on how fast the data must be
          processed before it becomes useless;
    \item noisy operations, which implies that applying operations on qubits degrades the quality of
          the qubit states themselves;
    \item cross-node scheduling dependencies, meaning that the operations on one node cannot be
          scheduled independently of other nodes;
    \item interaction between classical and quantum parts of a program, which requires keeping
          quantum data alive in memory while waiting for an event to occur (e.g. a message from a
          remote network node).
\end{inlinelist}

The first three challenges are technological, that is, we expect the situation to improve with
further progress in quantum hardware development. Furthermore, to a certain extent, these issues are
common to quantum computing too, from which we can draw some inspiration for their solutions. The
fourth challenge is inherent to the nature of entanglement generation and to the physics of the
devices currently in use. Successful entanglement generation requires both nodes to execute a
network operation at the same moment in time. Moreover, at this stage of development, the same
quantum device functions as the processing unit and as the network device, and consequently local
quantum operations (such as measurements and gates) and network operations cannot be performed
simultaneously~\cite{vardoyan_2019_performance}. This limitation, however, could be mitigated by a
new quantum hardware architecture separating the devices~\cite{vardoyan_2022_netarch}. Finally, the
last challenge is a fundamental issue, as it applies to quantum network applications regardless of
technological progress. It is the last two points that fundamentally differentiate a quantum
networking node from a quantum computing system, and are the key driver for a \emph{networked}
quantum node operating system.

\begin{xstretch}
\printbibliography[heading=subbibintoc,title={References},notcategory=noprint]
\end{xstretch}

\chapter
 [Architecture of an Operating System for a Quantum Network Node]
 {Architecture of an\\Operating System for a\\Quantum Network Node}
\label{chp:arch}

\begin{abstract}
The end goal of an \acrfull{os} for quantum network nodes is to bridge the gap between user
applications --- written in high-level and platform-independent software --- and the underlying
quantum hardware, to which the user is agnostic. How can one design a control system that adheres to
this objective, while addressing the challenges that come with quantum networking? And what does an
example architecture of such a system look like? This chapter explores the cardinal design
considerations that should drive the design of an \acrshort{os} for quantum network nodes, and
proposes a proof-of-principle architecture for such an \acrshort{os}.
\end{abstract}

\note{To be precise, this chapter is extracted from sections 3 (Design Considerations, but excluding
    3.1) and 4 (QNodeOS Design) from the QNodeOS paper. There aren't any major additions.}

\blfootnote{
    This chapter is based on the preprint \citetitle{delledonne_2023_qnodeos} by
    \textcite{delledonne_2023_qnodeos}, submitted for peer review.
}

\newpage

\lettrine{A}{n} \acrfull{os} is usually the cornerstone of a system's control software: it manages
and marshals access to physical resources, abstracts low-level hardware functionalities into
user-friendly services, and provides an interface to users to program and run applications on the
system. Our goal here is to apply basic principles from classical \acrshort{os} design literature to
our novel use case of programmable and scalable quantum networking nodes, and to hopefully create a
framework where the challenges outlined in \cref{chp:background} can be studied and addressed. We
thus investigate the general requirements that such a system should satisfy, illustrated with the
example of a quantum processor based on \acrfull{nv} centers in diamond. This provides a guideline
for future systems of this form. We also propose the first proof-of-principle architecture for an
\acrlong{os} for quantum network nodes, which we call \acrshort{qnodeos}. Our system's capabilities
include quantum memory management, scheduling different types of quantum operations on the device,
as well as an interface to different drivers addressing several possible quantum hardware
architectures. On the quantum networking front, \acrshort{qnodeos} adopts the quantum network stack
and protocols proposed by \textcite{dahlberg_2019_egp} and by \textcite{kozlowski_2020_qnp}.

\section{General Design Considerations}
\label{sec:arch:considerations}

We assume that the operating system builds upon a quantum hardware system capable of the execution
of \emph{physical instructions} addressing specific qubits on the quantum chip. These physical
instructions may be dependent on the type of quantum hardware (e.g. NV in Diamond, or Ion Traps),
and include instructions for initializing and measuring qubits on the chip, moving the state of a
qubit to another location in the quantum memory, performing quantum gates, as well as to make
attempts at entanglement generation at the physical layer~\cite{pompili_2022_experimental}. The
quantum hardware furthermore exposes the capabilities of the quantum chip: (1) the number of qubits
(2) the type of each qubit (3) the memory lifetime of the qubits (4) the physical instructions that
can be performed on on the qubit(s) and (5) the average quality of these instructions
(\cref{chp:background}). We emphasize that, unlike in classical computing, there is currently no
established low-level microarchitecture that defines the line between (quantum) hardware and
software upon which such an operating system would be built. We nevertheless expect that almost all
of the below would be functions taken on by any operating system, some of which could possibly be
shifted to control hardware in the future.

Each node in the network runs its own independent quantum network operating system. Nodes may
interact with each other using both classical message passing as well as entanglement generation.
The goal of the combined system is to execute quantum network applications, which themselves consist
of separate programs running on the operating system(s) of two (or more) network nodes. Such
programs generally also communicate via classical message passing and entanglement generation. Each
program itself consists of both classical and quantum blocks of code, where the quantum blocks of
code may contain low-level classical logic (specifically, branching on classical variables and
loops). Classical blocks of code may depend on quantum ones via classical variables generated during
the quantum execution (measurement results, notification of entanglement generation, and information
on the state of the quantum system such as the availability of qubits). Similarly, quantum blocks
may depend on variables set by the classical blocks, such as messages received from remote network
nodes. Finally, quantum blocks may themselves depend on other quantum blocks via qubits in the
quantum memory. It is the responsibility of the programmer or compiler to identify what is a
classical and what is a quantum block. Similarly, we assume that (potentially fine-grained)
deadlines or priorities in the execution are determined by the programmer (or compiler) using the
knowledge of the exposed capabilities of the quantum hardware system (e.g. memory lifetimes).
Determining precise deadlines (e.g. when too much time has elapsed for the qubits to be useful) is
in general a computationally expensive procedure, sometimes estimated in practice by a repeated
simulation of the execution. We remark that there is no way in quantum mechanics to measure the
current quality of a qubit or operation during the ongoing execution, and such qualities are
determined by performing estimates independently of the program execution itself. Of course, the
operating system could itself engage in such estimates when idling in order to update its knowledge
of the capabilities of the quantum hardware.

\section{Key OS Components}
\label{sec:arch:components}

We now describe the essential components we envision any operating system for quantum network nodes
to have.

\subsection{Memory Management Unit}

Executing quantum network applications demands a continuing interaction between the classical and
quantum parts of the execution, including keeping qubits alive in memory to take further actions
depending on messages from remote network nodes. We thus require persistent memory management
capabilities. This may be taken up by a quantum memory management unit (QMMU). A QMMU has knowledge
of the physical qubits available on the underlying quantum hardware, and may keep any other
information about said qubits, such as the qubit type (communication or storage qubit) and qubit
lifetime. A QMMU allows physical qubits to be assigned to different applications or to the operating
system itself, and may allow a transfer of ownership of the qubits from one owner to another. A QMMU
may also provide abstractions familiar to classical computing such as a virtual address space, where
the applications refer to virtual qubit addresses that are then translated to physical qubit
addresses. This avoids the situation in which physical qubit addresses must be bound at compile
time, particularly limiting when allowing multiple applications to concurrently run on the same
node. Advanced forms of a QMMU may also cater to the limitations of near term quantum devices, by
matching memory lifetime requirements specified by the application code to the capabilities of the
underlying qubits, as well their topology (i.e. taking into account which two qubits allow two-qubit
gates to be performed on them directly). While one cannot measure the decoherence of a qubit during
a general program execution on the quantum level, the QMMU could also take into account additional
information from the classical control system to signal to the application that a qubit has become
invalid.

\subsection{Quantum Network Stack}

The OS should include the capability for quantum communication with remote nodes in the network,
typically the generation of entanglement. We thus assume that the OS realizes a quantum network
stack that can be relied upon to enable entanglement generation, where we refer to
Ref.~\cite{dahlberg_2019_egp} for design considerations of quantum network stacks themselves. The
OS's network stack allows an application to request a certain number or rate of entangled pairs to
be produced with remote nodes with a specified quality (i.e. fidelity) of entanglement. The stack is
responsible for ensuring the delivery of the entanglement. One possible quantum network stack can be
found in Ref.~\cite{dahlberg_2019_egp} including the first link layer protocol now realized on
quantum hardware~\cite{pompili_2022_experimental}, and a network layer protocol (as proposed in
Ref.~\cite{kozlowski_2020_qnp}).

To successfully produce entanglement, the network stack needs access to a communication qubit,
resulting in two requirements for the rest of the system:
\begin{enumerate*}[label=(\arabic*)]
    \item A scheduler (see below) should take into account that generating entanglement at the
          physical layer between two nodes directly connected by a physical communication medium
          requires that the two nodes apply a series of physical operations with very precise timing
          synchronization between them (nanosecond precision with sub-nanosecond jitter). Therefore,
          entanglement generation across a link with an adjacent node must always be scheduled in a
          synchronized manner between the two adjacent neighbors. Similarly, due to limited memory
          lifetimes, generating entanglement with the help of an intermediary node at the network
          layer~\cite{kozlowski_2020_qnp} requires specific operations (entanglement swapping) to be
          scheduled at all three nodes within a time window allowed by the memory lifetimes.
    \item On some quantum hardware systems (e.g. NV in diamond), the communication qubit is in
          general needed to enable the execution of quantum gates on and in-between storage qubits.
          This has implications both on the scheduler (local instructions cannot be scheduled
          concurrently to networked ones), as well as on the QMMU, which needs to allow qubit
          ownership transfer between applications and the network stack. A typical use case of such
          ownership transfer would occur when the network stack claims the communication qubit for
          entanglement generation, and then yields it to an application.
\end{enumerate*}

\subsection{Scheduler}

In order to maximize the usage of resources, we envision the OS to include a scheduler. This may be
a single scheduler, or more likely several schedulers that address scheduling at different levels.
In general, we may consider scheduling at the level of applications, at the level of blocks of
quantum code, and at the level of instructions, each level not being independent of one another.

\paragraph{General considerations}

A scheduler for quantum network nodes should be capable of managing the limited physical resources
to achieve the desired performance. The performance of any form of scheduling method in the quantum
domain is assessed not only by existing classical metrics --- like throughput and latency --- but
also by quantum metrics (see \cref{chp:background}). At the level of the application, the latter can
be measured in terms of the success probability of the quantum network application. At the level of
an operation (or a block of operations) it may be measured by the quantum qualities (fidelity) of
the states and operations performed. We remind that due to limited memory lifetimes, delays have
always a direct impact on the quantum metrics of performance, resulting in general in trade-offs
between classical and quantum performance metrics when assessing any scheduling procedure.

Practically, the scheduler in question should allocate the underlying physical resources --- most
importantly, the available qubits --- based on a set of well-defined constraints, the fundamental
ones being:
%
\begin{enumerate*}[label=(\arabic*)]
    \item \emph{Synchronized network schedule}: due to the bilateral nature of entanglement, each
          node will have its quantum networking activity synchronized with its neighbors, thus
          missing a synchronization window on one node results in a waste of resources on remote
          nodes too.
    \item \emph{Local quantum computation}: in addition to quantum networking, a node's resources
          must also be reserved for local quantum gates, which are integral parts of quantum network
          applications.
    \item \emph{Inter-block dependencies}: quantum and classical processing blocks of an application
          may depend on results originating from other blocks, and thus cannot be scheduled
          independently.
    \item \emph{Multitasking}: for a node to be shared by multiple users, the scheduler should not
          allocate all the available resources to a single application indefinitely, and instead it
          should be aware of the presence of multiple applications.
\end{enumerate*}

Additionally, scheduling at any level could optionally process another set of input variables, where
we generally assume that the programmer or compiler provide aggregate advice based on these input
variables to the OS:
%
\begin{enumerate*}[label=(\arabic*)]
    \item \emph{Duration of operations}: local quantum operations typically take a fixed amount of
          time and always succeed. Entanglement, on the other hand, is a probabilistic process, and
          generating an entangled pair can take an indefinite (and large) number of attempts.
          Scheduling decisions may factor this in to yield better performance.
    \item \emph{Decoherence}: as already stated, the fidelity of a quantum state stored in a qubit
          is not constant, and it also degrades due to physical noise induced by other qubits and by
          operations applied on such qubits. An advanced scheduler could use knowledge of qubit
          lifetimes and elapsed time to dynamically re-prioritize application demands based on the
          advice of the compiler.
\end{enumerate*}

\paragraph{Scheduling of applications}

In an OS allowing the execution of multiple concurrent quantum network applications, the task of an
application-level scheduler would be to decide which application to schedule next. We remark that a
programmer (or compiler) aware of the underlying capabilities of the hardware system (e.g. memory
lifetimes) can provide advice in the form of a deadline by which the network application must have
completed in order to be successful. To allow for potentially time-consuming classical pre- and
post-processing, it is natural to apply such deadlines not for the entirety of the application, but
for the period between initializing the qubits and terminating the quantum part of the execution.
This suggests in general using real-time schedulers for quantum network applications, taking
inspiration from the extensive work on this topic in classical systems (see e.g.
Ref.~\cite{liu_1973_scheduling}). While outside the scope of this work, we remark that this type of
scheduling offers to inspire interesting new work in a form of ``quantum soft-real time''
scheduling, where deadlines may occasionally be missed at the expense of reduced application
performance (success probability), to maximize the overall performance of the system in which
applications are typically executed repeatedly. A benchmark for the quantum performance of any
application level scheduler is the quality of the quantum execution when the entire system (all
nodes) are reserved for only one application at the time.

\paragraph{Scheduling of quantum blocks}

Scheduling can also (additionally) be performed on the level of quantum blocks of code. This can in
principle also take the form of a (soft) real-time scheduler that schedules blocks of the currently
running application, or schedule blocks of several applications (potentially independently of any
application level scheduling) depending on the availability of resources on the quantum hardware
system. This form of scheduling may be appealing for efficiency reasons, depending on where what
parts of the operating system are executed, where some parts are closer to the underlying hardware
system than others (see e.g. \cref{sec:arch:design}).

\paragraph{Scheduling of operations}

Finally, scheduling can be performed at two levels of operations: First, one can consider the
problem of scheduling local versus networked instructions, where one simple way of realizing a
schedule that respects the constraints inherent in such a schedule (see above) is presented in
\cref{sec:arch:design}. Second, one can consider scheduling any form of operation on the underlying
quantum processor. While our current realization of QNodeOS now achieves this by populating an
instruction queue in software, we envision that this form of scheduling would later be moved from
QNodeOS to control hardware in a microarchitecture for quantum nodes as e.g. in the work by
\textcite{fu_2017_microarch}.

\section{QNodeOS Design}
\label{sec:arch:design}

\acrshort{qnodeos} is an operating system for quantum network nodes, designed to address the
challenges described in \cref{chp:background}. It includes all the identified key components, plus
some additional convenience abstraction layers. The current design of QNodeOS is considered
\emph{best-effort} --- it is meant to explore the main design aspects of an operating system for
quantum networks, and to provide a minimum working system.

\subsection{Full Stack of a Quantum Network Node}

As described in \cref{chp:background} and illustrated in \cref{fig:app-struct}, a quantum network
application consists of programs running on different end nodes, composed of blocks of quantum code
and blocks of fully-classical code. In fact, quantum code blocks may also contain lightweight
classical logic --- like simple arithmetic and branching instructions --- used for flow control.
These quantum code blocks do not have any dependencies on data originating from other nodes.
Fully-classical code blocks --- which include local processing and communication with other end
nodes --- mainly produce input data for the next quantum code blocks. That is, a classical code
block typically precedes a quantum code block whose instructions depend on external data coming from
a remote end node. In our system, quantum code blocks expressed in
\emph{NetQASM}~\cite{dahlberg_2022_netqasm}. NetQASM is an open-source software development kit
(SDK) and instruction set for quantum applications~\cite{netqasm_sdk}. The NetQASM SDK compiles a
quantum network application, written in Python, into a series of classical and quantum code blocks.
The instruction set used for the quantum code blocks is similar to other QASM
languages~\cite{cross_2017_qasm, khammassi_2018_cqasm, fu_2019_eqasm}, but it is extended to include
instructions for quantum networking. NetQASM is not a strict requirement of QNodeOS, but it does
impose certain conventions (described in Ref.~\cite{dahlberg_2022_netqasm}) on a particular
implementation of the system.

In principle, classical and quantum code blocks can be run on a single system, provided that this
has a connection to the quantum device to execute the actual quantum instructions. However, in the
interest of a simpler implementation, where each system has a scoped responsibility, we opted to map
classical and quantum blocks onto two distinct environments. Classical blocks are run on a system
that features a fully-fledged OS (like Linux), with access to high level programming languages (like
C++ and Python) and libraries. Quantum blocks are delegated to the \emph{quantum network processing
unit} (QNPU), which is a system capable of interpreting quantum code blocks and managing the
resources of a quantum device. In our design, a quantum network application starts on the
general-purpose OS --- that we call the \emph{host} --- which runs classical code blocks internally,
and offloads quantum code blocks to the QNPU. QNodeOS is the topmost component of the QNPU. It runs
the quantum code blocks, relying on the underlying quantum device --- denoted as \emph{QDevice} ---
to execute the actual quantum operations.

The architecture of a quantum network node is depicted in \cref{fig:quantum-node}. An alternative
architecture of the end node of a quantum network could merge host and QNodeOS into the same system,
potentially enabling some performance optimizations, at the cost of a higher system complexity. We
also note that the host, QNodeOS and the classical control modules of the QDevice can be deployed on
distinct physical devices, or combined in some way. For simplicity, we implemented each component
(host, QNodeOS, QDevice) on a separate device (refer to \cref{sec:implementation} for details).

\subsection{Processes}

A quantum network application starts on the host --- there, the host environment compiles it into
classical and quantum code blocks, and creates a new process associated with the application. The
host then registers the application with QNodeOS (through QNodeOS's end node API), which, in turn,
creates its own process associated with the registered application. The process on the host is a
standard OS process, which executes the classical code blocks and interacts with the counterpart
process on QNodeOS (by means of a shared memory, as defined in
NetQASM~\cite{dahlberg_2022_netqasm}). On QNodeOS, a process encapsulates the execution of quantum
code blocks of an application with associated context information, such as process owner, ID,
process state and priority.

\begin{figure}
    \centering
    \includegraphics[width=0.6\linewidth]{figures/quantum-node.pdf}
    \caption{
        Full-stack architecture of a quantum network node. User applications start in the host
        environment, which runs classical code blocks and offloads quantum code blocks to the QNPU.
        QNodeOS, lying at the top of the QNPU, processes quantum code blocks and invokes the quantum
        device (QDevice) to run the actual quantum instructions. QNodeOS consists of an end node API
        handler, a quantum network stack (Q. net. stack), a process manager (Proc. manager), a quantum
        memory management unit (QMMU), a scheduler, a processor, and a QDevice driver to communicate
        with the QDevice itself. The host shares a classical communication channel with other end nodes'
        hosts for application data. QNodeOS shares a classical channel with its neighbors. The QDevice
        shares a classical channel for coordination and a quantum channel for entanglement with its
        neighbors.}
    \label{fig:quantum-node}
\end{figure}

The execution time of an application is typically dominated by that of quantum blocks, as
entanglement generation is a time-consuming operation, and its duration grows exponentially with the
distance between the nodes. For this reason, in this paper we focus on the scheduling of quantum
blocks only, and thus we only discuss QNodeOS processes from this point onward. Again, this does not
exclude that, in a future iteration of the design, host and QNodeOS could be merged into one system,
and therefore classical and quantum blocks would be scheduled jointly.

\paragraph{QNodeOS user processes}

QNodeOS allocates a new \emph{user process} for each quantum network application registered by the
host. A user process becomes active (ready to be scheduled) as soon as QNodeOS receives a quantum
code block from the host. Multiple user processes --- relative to different host applications ---
can be concurrently active on QNodeOS, but only one can be running at any time. A running user
process executes its quantum code block instructions directly, except for entanglement requests,
which are instead submitted to the quantum network stack and executed asynchronously.

\paragraph{QNodeOS network process}

QNodeOS also defines \emph{kernel processes}, which are similar to user processes, but are created
by default (on boot) and have different priority values. Currently, the only existing kernel process
is the \emph{network process}. The network process, owned by the quantum network stack, handles
entanglement requests submitted by user processes, coordinates entanglement generation with the rest
of the network, and eventually returns entangled qubits to user processes. The activation of the
network process is dictated by a network-wide entanglement generation schedule. Such a schedule
defines when a particular entanglement generation request can be processed, and therefore it has
intersecting entries on adjacent nodes (given that entanglement is a two-party process). The
schedule can be computed by a centralized network controller~\cite{skrzypczyk_2021_arch} or by a
distributed protocol~\cite{dahlberg_2019_egp}. In our design, the network process follows a
\emph{time-division multiple access schedule}, computed by a centralized network controller (as
originally proposed by \textcite{skrzypczyk_2021_arch}) and installed on each QNodeOS node.

\paragraph{QNodeOS process states}

A QNodeOS process can be in any of the following states:
%
\begin{enumerate*}[label=(\arabic*)]
    \item \emph{idle}: when it exists but it is not active;
    \item \emph{ready}: when it is active and ready to issue instructions;
    \item \emph{running}: when it is running on QNodeOS;
    \item \emph{waiting}: when it is waiting for some event to occur.
\end{enumerate*}
User processes enter the waiting state when they need one or more entangled pairs to proceed, and
then become ready again once all the requested pairs are delivered by the network process.

\paragraph{Inter-process communication}

At the moment, QNodeOS does not allow for any explicit inter-process communication. The only
indirect primitive available to processes to interact with one another is \emph{qubit ownership
transfer}, used when a process produces a qubit state which is to be consumed by another process. In
particular, the network process transfers ownership of the entangled qubits that it produces to the
process which requested them.

\paragraph{Process concurrency}

The strict separation between local quantum processing and quantum networking is a key design
decision in QNodeOS, as it helps us address the scheduling challenge described in
\cref{chp:background}. A user process can continue executing local instructions even after it has
requested entanglement. Conversely, networking instructions can execute asynchronously of local
quantum instructions. This is rather important in a quantum network, since all entanglement
generation must be synchronized with the neighboring node (and possibly the rest of the
network~\cite{skrzypczyk_2021_arch}). Additionally, separating user applications into user processes
also allows QNodeOS to schedule several applications \emph{concurrently}.

\begin{figure}
    \centering
    \includegraphics[width=0.6\linewidth]{figures/process-flow.pdf}
    \caption{Flow of execution between a user process requesting entanglement and the network process
        responsible for generating entanglement. The user process starts by asynchronously issuing an
        entanglement request. Once issued, it is free to continue with other local operations or
        classical processing. Once it reaches a point in its execution where entanglement is required
        the process enters the waiting state. The network process is scheduled once the appropriate time
        bin (as determined by the network schedule) starts. Once running, it attempts entanglement
        generation until entanglement success (or until a set timeout). The entangled qubit is then
        transferred to the user process. This unblocks the process which consumes the entanglement and
        releases the qubit.}
    \label{fig:process-flow}
\end{figure}

\paragraph{Process flow}

\cref{fig:process-flow} illustrates the typical control flow between a user process and the network
process. User processes are free to execute any non-networked instructions independently of the
network process and other user processes. Once the application reaches a point in its execution
where an entangled qubit is required, the process enters the waiting state and is flagged as waiting
for entanglement. When the network process is scheduled, it issues network instructions and
generates entanglement as requested by the user process. Once an entangled pair is generated by the
network process, the qubit is handed over to the waiting user process. When all the entangled pairs
that the user process was waiting for are delivered, the user process becomes ready and can start
running again.

\subsection{Process Scheduling}

As previously mentioned, we focus our attention on the quantum blocks of an application, and thus we
only discuss the scheduling of QNodeOS processes. At present, the QNodeOS scheduler does not give
any guarantees on when a process is scheduled --- for that, one would need to define concrete
real-time constraints to feed to the scheduler. Instead, the current version of QNodeOS implements a
best-effort scheduler, which selects processes on the basis of their priority, and does not allow
preemption.

\paragraph{Priority scheduling}

QNodeOS schedules ready processes using a \emph{priority-based} algorithm. In particular, the
network process is assigned the highest priority, and is given precedence whenever the network
schedule activates it~\cite{skrzypczyk_2021_arch}. Prioritizing entanglement generation over local
operations is key for a node to be able to fulfill its networking duty, and to avoid peer nodes to
waste their resources.

\paragraph{No preemption}

To avoid context switching overhead, potentially leading to degraded fidelity, the QNodeOS scheduler
is \emph{cooperative}. That is, once a process is scheduled, it gets to run until it either
completes all of its instructions or it blocks waiting for entanglement. Allowing process preemption
would need a definition of critical section and could potentially impact the quality of the affected
qubit states.

\subsection{QNodeOS Architecture}

\Cref{fig:quantum-node} illustrates the internals of QNodeOS, and outlines the interactions with the
rest of the components of a quantum network node. At its top, QNodeOS implements an \emph{end node
    API handler} to process requests from the host. Internally, QNodeOS features a \emph{quantum network
    stack}, a \emph{process manager}, a \emph{process scheduler}, a \emph{quantum memory management
    unit} (QMMU), and an \emph{instruction processor}. Actual quantum instructions are offloaded to the
underlying quantum device (QDevice) through the \emph{QDevice driver}.

We note that QNodeOS itself is an entirely classical system that interacts with the quantum hardware
(the QDevice). At the moment, our implementation of QNodeOS is fully software, including the
instruction processor. In general, the system may be implemented entirely in software running on a
classical CPU, or parts of its functionality may be implemented in classical hardware (e.g.~FPGA or
ASIC).

\paragraph{End node API}

Each user application is registered on QNodeOS by the host through the end node API. Using the same
API, the host can then send quantum code blocks and receive their results (like measurement outcomes
and entanglement generation information). Upon registration of an application, QNodeOS allocates a
new user process. Upon reception of a quantum code block, the related user process is activated and
made eligible for scheduling.

\paragraph{Process manager, scheduler, processor}

The QNodeOS process manager keeps track of existing user and kernel processes and their execution
context. Upon activation, processes are added to a scheduling queue. When selected by the scheduler,
a process is assigned to the QNodeOS processor, which
%
\begin{enumerate*}[label=(\arabic*)]
    \item executes classical control-flow instructions directly,
    \item offloads local quantum computation to the QDevice, and
    \item registers entanglement requests with the quantum network stack.
\end{enumerate*}

\paragraph{Quantum network stack}

The role of the quantum network stack in QNodeOS is to abstract the unreliable entanglement attempts
that the QDevice offers into a robust, multi-node network service. The network stack can handle
entanglement generation requests which specify a number of parameters --- including source and
destination, desired fidelity, and number of entangled pairs --- and returns the entangled pair(s)
described by an identifier and the generated Bell state. The quantum network stack owns the network
process, whose activation is dictated by a network-wide entanglement schedule. The quantum network
stack in QNodeOS is based on the model outlined by \textcite{dahlberg_2019_egp}, and features a
\emph{link layer protocol} --- presented in the same work, and recently evaluated on
hardware~\cite{pompili_2022_experimental} --- and a \emph{network layer protocol} --- as designed by
\textcite{kozlowski_2020_qnp}.

\paragraph{Quantum memory management unit}

QNodeOS's QMMU implements basic memory management functionality: \emph{virtual address spaces} and
\emph{qubit ownership transfer}. A virtual quantum memory address space is akin to a classical
virtual address space, but it isolates the qubit address spaces of QNodeOS processes. Ownership
transfer is an indirect type of inter-process communication (IPC) mechanism for passing quantum data
between processes. Since quantum states cannot be copied due to the no-cloning theorem, this is the
only valid IPC for passing quantum data between address spaces. Ownership transfer is only logical
--- only the data's owner is updated --- rather than it being a physical move of quantum data in the
memory. This way we can avoid issuing a QDevice instruction which would cause degradation in
fidelity due to hardware imperfections and additional processing time. The current QMMU is rather
simple due to the fact that our current quantum nodes have only have a few qubits each. Features
like decoherence tracking and topology-based allocation can be part of a later version of the QMMU.

\printbibliography[heading=subbibintoc,title={References}]

\chapter{Entanglement Generation With a Quantum Networking Stack}
\label{chp:netstack}

\begin{abstract}
Chapter abstract.
\end{abstract}

\blfootnote{
    This chapter is based on the paper \citetitle{pompili_2022_experimental} by
    \textcite{pompili_2022_experimental}.
}

\newpage

% \lettrine{N}{...}

\section{Implementing a Quantum Link Layer Protocol}

``Quantum link layer protocol'' and ``Revised protocol'' sections from the npj paper.

\section{Real-Time Control of the Quantum Physical Layer}

``Physical layer control in real-time'' section from the npj paper.

\section{Results and Takeaways}

``Evaluation'' section from the npj paper.

\section{Conclusion}

...

\printbibliography[heading=subbibintoc,title={References}]

\chapter{Quantum Networking With an Elementary Operating System}
\label{chp:qnodeos}

\begin{abstract}
An \acrfull{os} for quantum network nodes should provide more than just networking functionalities.
Ultimately, it should enable quantum networking applications to be written in high-level,
platform-independent software, and should be able to manage the resources of the underlying
device when deployed in a multi-node and multi-user quantum network. This chapter briefly
discusses our implementation of \acrshort{qnodeos}, an \acrshort{os} for quantum network nodes,
which includes a quantum network stack for entanglement generation, as well as resource
management and scheduling features that allow the concurrent execution of multiple applications.
We also validate our implementation on state-of-the-art quantum network hardware based on
\acrlong{nv} centers in diamond, and discuss general design considerations for any such quantum
network \acrshort{os}. Our work sets a baseline for running applications on quantum networks
of the future, and serves as a hands-on study to understand and define computer-science
challenges in building quantum network \acrshortpl{os}.
\end{abstract}

\noindent
\note{To be precise, this chapter is extracted from sections 5 (Implementation), 6 (Evaluation), 7
(Related Work) and 8 (Discussion) from the \acrshort{qnodeos} paper. There aren't any major
additions.}

\blfootnote{
    This chapter is based on the preprint: \fullcite{delledonne_2023_qnodeos_noprint}. \note{Add
    proper link to arXiv when submitted}
}

\newpage

\lettrine{T}{he} preliminary experiments conducted in \cref{chp:netstack} showcased elementary
quantum networking functionalities through a platform-independent control system --- mainly, the
quantum networking stack embedded in \acrshort{qnodeos}. Nevertheless, entanglement generation is
just one of the blocks constituting fully-fledged quantum communications applications, which also
include local quantum processing and classical communication and processing, as shown in
\cref{fig:app-struct}. With \acrshort{qnodeos}, we aim to take the state of the art of quantum
networking experiments one step further, and demonstrate the execution of complete applications,
some of which comprise quantum and classical processing and communication. We also include one case
study that serves as a proof-of-concept demonstration of the usefulness of a multitasking-ready
\acrshort{os}, to be expanded on when investigating multi-user quantum networks more in depth. In
this chapter we report on and discuss the results of these experiments. Prior to that, we also give
a brief overview of the implementation of \acrshort{qnodeos} and the underlying \acrshort{qdevice}.
Refer to \cref{app:qnodeos} for additional details on the implementation of the components of
\acrshort{qnodeos} and their interfaces, and to \cref{app:qdevice} for the specification of the
interface to the \acrshort{qdevice}.

\section{Implementation}
\label{sec:qnodeos:implementation}

\Cref{fig:node-deployment} outlines software and hardware implementation of \acrshort{qnodeos} and
the whole node system. \acrshort{qnodeos} is implemented in C++ on top of FreeRTOS~\cite{freertos},
a tiny operating system for microcontrollers. The stack runs on a dedicated MicroZed~\cite{microzed}
--- an off-the-shelf platform based on the Zynq-7000 SoC, which hosts two ARM Cortex-A9 processing
cores, of which only one is used, clocked at \qty{667}{\MHz}. \acrshort{qnodeos} connects to peer
\acrshort{qnodeos} systems via \acrshort{tcp} over a Gigabit Ethernet interface. For the
\acrshort{qdevice}, we replicated the setup used for \cref{chp:netstack}, which mainly consists of:
%
\begin{inlinelist}
    \item an ADwin-Pro II~\cite{adwin} acting as the main orchestrator of the setup;
    \item a series of subordinate devices responsible for qubit control, including laser pulse
          generators and optical readout circuits;
    \item the quantum physical device, based on \acrshort{nv} centers, counting one single
          (communication) qubit for each node.
\end{inlinelist}
The \acrshort{qdevice} is where the time-critical qubit control lies. \acrshort{qnodeos} interfaces
with the \acrshort{qdevice}'s ADwin-Pro II through a \qty{12.5}{\MHz} \acrshort{spi} interface, used
to exchange 4-byte control messages at a rate of \qty{50}{kHz}. Finally, the host layer is a Python
runtime executing on a general-purpose 4-core desktop machine running Linux. The host machine
connects to \acrshort{qnodeos} via \acrshort{tcp} over the same Gigabit Ethernet interface that
\acrshort{qnodeos} uses to connect to its peers (average ping \acrshort{rtt} of \qty{0.1}{\ms}), and
sends application registration requests and quantum code blocks over this interface (\num{10} to
\num{1000} bytes, depending on the length of the block).

We implemented \acrshort{qnodeos} on top of FreeRTOS to avoid re-implementing standard \acrshort{os}
primitives like threads and network communication. FreeRTOS provides basic \acrshort{os}
abstractions like tasks, inter-task message passing, and the \acrshort{tcpip} stack. The FreeRTOS
kernel --- like any other standard \acrshort{os} --- cannot however directly manage the quantum
resources (qubits, entanglement requests and entangled pairs), and hence its task scheduler cannot
take decisions based on such resources. \acrshort{qnodeos} adds these capabilities and takes care of
the scheduling of quantum code blocks based on the status of the quantum resources.

\begin{figure}[t]
    \centering
    \includegraphics[width=0.6\linewidth]{figures/node-deployment.pdf}
    \caption{
        Node deployment overview. Our quantum network node consists of a desktop machine for the
        host runtime, a Zynq-7000 SoC for \acrshort{qnodeos}, and a series of digital and analog
        controllers for the \acrshort{qdevice}.
    }
    \label{fig:node-deployment}
\end{figure}

\section{Evaluation}
\label{sec:qnodeos:evaluation}

We verify the functioning of \acrshort{qnodeos} by means of four case studies, aimed at validating
\begin{inlinelist}
    \item single-node execution, including qubit initialization, gates, and measurements,
    \item entanglement generation,
    \item delegated quantum computation, and
    \item multitasking.
\end{inlinelist}

\subsection{Local Qubit State Tomography}

Our first case study is a simple local application where a certain qubit state is prepared and then
immediately measured. This translates to one or more single-qubit gates and one qubit measurement.
The application is run several times to assess the quality of the prepared state, and various qubit
states are analyzed. This local qubit state tomography is the simplest application
\acrshort{qnodeos} can run --- there is a single user process running, and the network process is
not even activated, given that entanglement is never requested.

\paragraph{Results}

This application prepares the six cardinal states $\ket{+X}$, $\ket{+Y}$, $\ket{+Z}$, $\ket{-X}$,
$\ket{-Y}$, and $\ket{-Z}$, in sequence. Each state is measured in all six cardinal bases. This
application is run \num{1000} times for each combination of target state and readout basis. The
measured state fidelity, reported in \cref{fig:local-tomo}, is in line with what the quantum
hardware is capable of delivering, showing that basic interactions among the components of
\acrshort{qnodeos} and with the \acrshort{qdevice} work.

\begin{figure}[t]
    \centering
    \includegraphics[width=0.6\linewidth]{figures/local-tomography.pdf}
    \caption{
        Average fidelity of the prepared cardinal states $\ket{+X}$, $\ket{+Y}$, $\ket{+Z}$,
        $\ket{-X}$, $\ket{-Y}$, and $\ket{-Z}$ in the local qubit state tomography application.
    }
    \label{fig:local-tomo}
\end{figure}

\subsection{Entanglement Generation}

The second test case is an application that generates an entangled pair between the two nodes and
measures the generated state right away. This is a distributed application, where both nodes are
active --- they engage in entanglement generation, and they both measure their end of the entangled
pair. As the user can specify the requested fidelity of the entangled pairs, this application is run
for various target fidelities. This time, all \acrshort{qnodeos} components are at work. Since
entanglement is requested, the quantum network stack is triggered, and thus the network process
becomes active, competing for resources with the user process. The \acrshort{qmmu} is also invoked
by the network process to transfer ownership of the entangled qubit to the user process (the
inter-process communication primitive of \acrshort{qnodeos}).

\paragraph{Results}

Entanglement generation is run for a range or target fidelities \numlist{0.50; 0.55; 0.60; 0.65;
0.70; 0.75; 0.80}, and entangled pairs are read out in various bases to measure their correlators
$\braket{\mathrm{XX}}$, $\braket{\mathrm{YY}}$ and $\braket{\mathrm{ZZ}}$ (and their $\pm$
variations, for a total of $12$ correlators). The application is run \num{125} times for each
combination of target fidelity and correlator, for a total of \num{10500} entangled pairs. The
results for measured fidelity versus requested fidelity are shown in \cref{fig:ent-gen}. The
measured fidelities are --- within measurement uncertainty --- always matching or exceeding the
requested minimum ones (the dashed gray line in \cref{fig:ent-gen} is the $y=x$ diagonal). It is to
be noted that measurement outcomes are post-processed to eliminate tomography errors and events in
which the physical devices were in the incorrect charge state.
\note{Retake this experiment, fidelities have gotten worse on setup. Also report on fidelities
without correction.}

\begin{figure}[t]
    \centering
    \includegraphics[width=0.6\linewidth]{figures/ent-gen.pdf}
    \caption{
        Measured fidelity of the entangled states generated and read out via \acrshort{qnodeos}.
        Measurements are corrected to eliminate tomography errors and events in which the physical
        devices were in the incorrect charge state. Error bars represent 1 s.d. The dashed gray line
        is the $y=x$ diagonal.
    }
    \label{fig:ent-gen}
\end{figure}

\subsection{Delegated Computation}

With this case study we aim to showcase a more complex quantum network application, schematically
depicted in \cref{fig:del-comp}. Here, one node acts as the client, and the other as the server. The
client's goal is to delegate a certain quantum computation on some data qubit to the server, while
keeping the server agnostic to the computation. To perform the desired computation, described by a
parameter $\alpha$, the two nodes follow these steps:
%
\begin{inlinelist}
    \item the two node establish an entangled state,
    \item the client ``encodes'' its qubit by means of a series of local gates, described by a
          parameter $\theta$,
    \item the client measures its end of the entangled pair and stores the classical outcome
          $m_\text{c}$
    \item the client communicates the delegated computation parameter, which is a function of
          $\alpha$, $\theta$ and $m_\text{c}$,
    \item the server performs the computation,
    \item the server measures its end of the entangled pair and sends the classical outcome
          $m_\text{s}$ back to the client.
\end{inlinelist}
In this scheme, the client application consists of a single quantum code block and an additional
classical code block that communicates the computation parameter to the server. More interestingly,
the server application comprises two quantum code blocks --- the first is the establishment of the
entangled pair, and the second is the delegated computation --- interleaved by a classical code
block that receives the computation parameter from the client. This is the first example of an
interactive application, where the execution (on the server) spans more than one quantum code block,
and thus the quantum state generated in one block has to persist and remain valid for the other
block too.

\begin{figure}[t]
    \centering
    \includegraphics[width=0.6\linewidth]{figures/del-comp.pdf}
    \caption{
        Schematic of delegated computation application. The client wishes to have the server perform
        a quantum computation on a certain data qubit. To do so, the two nodes establish
        entanglement, then the client processes and measures its end of the entangled pair, sends
        the computation parameter to the server, which finally executes the delegated computation.
        Blue boxes represent quantum code blocks. The client's application is composed of a single
        block, while the server's consists of one block for entanglement and one block for the
        quantum computation, interleaved by a classical code block (the reception of the computation
        parameter).
    }
    \label{fig:del-comp}
\end{figure}

\paragraph{Results}

\note{Refine when data is available.}
The delegated computation is run for various values of $\alpha$ ($\pi$, $\pi/2$) and $\theta$
($\pi$, $\pi/2$, $\pi/4$). The application is run \num{500} times for each combination of $\alpha$
and $\theta$. We report the measured fidelity of the qubit state after the delegated computation for
each of the computation values. We also report a breakdown of the average application latency, to
give an indication of where time is spent during execution. \note{Plot and discuss data} The
abstractions provided by \acrshort{qnodeos} allow for the execution of more complex quantum network
applications. Distributed applications may result in idle time on some nodes, which
\acrshort{qnodeos} can allocate to other pending applications.

\subsection{Multitasking}

One of the core features of modern \acrshortpl{os} is the ability to run several applications
concurrently, a key aspect in multi-user networks and nodes. \acrshort{qnodeos} is designed with
multitasking capabilities --- not only can it multiplex a user process and the network process, but
it also allows for multiple user processes to run at the same time. This means that multiple users
can submit their applications simultaneously, and \acrshort{qnodeos} will service all pending user
processes based on resource availability, in order to increase the utilization of the
\acrshort{qdevice} and to limit idle time and average application latency. One drawback of
multitasking on a quantum network node is the trade-off between concurrency and fidelity:
applications that have active data in the quantum memory, and that are waiting to be scheduled while
other applications are in progress, may experience lower-quality qubit states, given that such
quality degrades due to the passing of time and to noise induces by operations on other qubits.

We aim to demonstrate the multitasking capabilities of \acrshort{qnodeos} by having multiple users
run independent applications concurrently. In our case study, a pool of users runs the delegated
computation application, while the rest of the users runs the local qubit state tomography
application. We evaluate multitasking on the client node, while the server just runs its part of the
delegated computation application. The idle time resulting from running the delegated computation
application on the client is a perfect candidate for scheduling other pending applications. We
evaluate the multitasking performance of \acrshort{qnodeos} under various system load conditions,
which essentially depend on the number of users utilizing \acrshort{qnodeos} at the same time. We
note that, even though a higher degree of concurrency should in principle results in better device
utilization, this is limited by the scarce physical resources available on the underlying
\acrshort{qdevice}.

\paragraph{Results}

\note{Refine when data is available.}
We measure device utilization and average application latency on the client, for various
configurations of users and applications: $N$ users, with $N \in \{2, 3, 5, 10\}$, half of which
running the local tomography application, and the remaining half running the delegated computation
application. To measure the performance benefit of multitasking, we also run the same set of
applications with multitasking disabled --- on \acrshort{qnodeos}, this means that a user process
can only be scheduled if no other user processes are either running or waiting for entanglement
generation. We report device utilization and average application latency, for the various usage
patters, resulting from a continuous execution of each usage pattern over \qty{\approx 30}{min}.
\note{Plot and discuss data} The scheduler of \acrshort{qnodeos}, in combination with the
\acrshort{qmmu}, can dynamically re-prioritize outstanding processes based on resource availability.
\acrshort{qnodeos} can thus support multi-user quantum network nodes, and can take advantage of
network idle times to improve device utilization and average application latency. Impact of
multitasking on fidelity not visible with a single qubit.

\section{Discussion}

\noindent
\note{Copy over from \acrshort{qnodeos} paper after experiments have been performed}

\begin{xstretch}
\printbibliography[heading=subbibintoc,title={References},notcategory=noprint]
\end{xstretch}

\chapter{Data Origin Authentication in the Quantum Networking Stack}
\label{chp:doa}

\begin{abstract}
Quantum networking protocols make use of classical communications to coordinate entanglement
generation and other tasks. In this chapter, we discuss the need for authentication of
classical messages exchanged at the quantum network stack level, with focus on concrete protocol
proposals. We then experimentally measure the overhead incurred by sending authenticated
classical messages through an authentication system that uses key material supplied by an
\acrshort{mdiqkd} system. We use this information to simulate the performance of a quantum link
whose protocol stack uses an authenticated classical channel, and compare that to existing
simulations of the same protocol stack. We find that message authentication overhead is not
detrimental to entanglement generation and to the fidelity of the entangled pairs.
\end{abstract}

\blfootnote{
This chapter is based on the article in preparation: \fullcite{abrahams_2023_doa_noprint}.

Contributions: J. S. Abrahams, \textbf{C. Delle Donne}, J. A. Slater and S. Wehner designed the
research. J. S. Abrahams researched and implemented \acrshortpl{mac}. P. Brussee, T. Middelburg and
R. C. Berrevoets prepared the authentication proxy. J. S. Abrahams and \textbf{C. Delle Donne}
prepared the framework to measure \acrshort{rtt} delays. J. S. Abrahams and \textbf{C. Delle Donne}
ran the simulations. J. S. Abrahams, \textbf{C. Delle Donne}, J. A. Slater and S. Wehner wrote the
manuscript with input from all authors. J. A. Slater and S. Wehner supervised the research. J. S.
Abrahams and \textbf{C. Delle Donne} contributed equally to this work.
}

\newpage

\lettrine{U}{p} until now, several proposals have been put forward as to how quantum networks should
be structured~\cite{van_meter_2013_repeaters, schoute_2016_shortcuts, joshi_2020_trusted}.
Researchers are also investigating how to abstract the complex physics of quantum network devices so
as to provide platform-independent services to the end user~\cite{dahlberg_2019_egp,
pirker_2019_quantum, illiano_2022_quantum} --- one very basic service would be the distribution of
entanglement between network nodes, that is, \emph{entanglement generation}. To coordinate this, and
other activities, proposals for a \emph{quantum network stack} have provided outlines of the layers
and the separation of responsibilities, as well as protocols to populate these
layers~\cite{dahlberg_2019_egp, kozlowski_2020_qnp}. Within the proposed stack, each layer makes use
of the service exposed by the layer below, and provides a higher-level service to the layer
above~\cite{dahlberg_2019_egp}. This stack is inspired by classical architectures like the
well-known TCP/IP network stack or the more generic \acrfull{osi} model, and is illustrated in
\cref{fig:functional-allocation}. Each layer and protocol within the quantum network stack makes use
of classical control messages, exchanged over a classical network, to coordinate the quantum
communication activities. These messages from the \emph{control protocols} form what we refer to as
the \emph{classical data plane} --- which lives alongside the \emph{quantum data plane}, in which
information encoded in quantum systems is transmitted.

\begin{figure}[b]
    \centering
    \includegraphics[width=0.6\linewidth]{figures/functional-allocation.pdf}
    \caption{
        Functional allocation of layers in a quantum network stack, adapted from
        Ref.~\cite{dahlberg_2019_egp}. The physical layer is quantum platform-dependent. The link
        layer provides platform-independent robust entanglement generation. All subsequent layers
        are therefore also platform independent, including the network and optionally transport
        layers which facilitate end-to-end entanglement between non-adjacent nodes. The application
        layer uses the services offered by the stack to perform quantum networking tasks.
    }
    \label{fig:functional-allocation}
\end{figure}

When designing control protocols for quantum networks, one should carefully estimate various key
performance indicators, such as their impact on the end-to-end latency of quantum services, such as
entanglement generation between network nodes. One obvious reason to assess the impact on latency,
is the fundamental aspect of quantum information which happens to impose strict constraints on
end-to-end latency: \emph{decoherence}. Storing quantum data reliably for extended periods of time
is non-trivial. Qubits have relatively short lifetimes, usually of the order of milliseconds, or at
best of a few seconds~\cite{abobeih_2018_one_sec, bradley_2019_one_min}. Therefore, not only can
high end-to-end latency affect the quality of the service offered by the network, but in some cases
it may result in no service at all. Classical processing and communication overhead must, thus, be
kept to a minimum, such that the generated entangled qubits can be used as quickly as possible.

On top of that, control messages must be transmitted in a secure manner for a quantum network to
function reliably. \textcite{satoh_2020_attacking} list forged classical messages as a general
concern for quantum networks. To prevent such forgeries, quantum network nodes may employ
\acrfull{doa} cryptography suites, to distinguish between genuine and fraudulent control messages.
\acrshort{doa} is performed using a secret that is shared between two parties. Then, a \acrfull{mac}
algorithm can produce a tag for each control message both at the sending and at the receiving end,
to verify that the message contents
%
\begin{inlinelist}
    \item were not altered and
    \item were produced by a party which owns the shared secret.
\end{inlinelist}
This process --- known as \emph{authentication} --- is illustrated in \cref{fig:mac-structure}.

\begin{figure}[t]
    \centering
    \includegraphics[width=0.6\linewidth]{figures/mac-structure.pdf}
    \caption{
        General structure of a \acrfull{mac}, where sender and receiver have a shared key. The
        receiver checks the output of the \acrshort{mac} to verify that the message was not modified
        in transit.
    }
    \label{fig:mac-structure}
\end{figure}

Inevitably, performing \acrshort{doa} on control messages would incur some computation and
communication overhead. Such latency is typically neglected when modeling, simulating, or
experimentally validating network protocols for quantum communications. In this work, we examine the
effects of latency caused by classical \acrshort{doa} on a simulated quantum network link requested
to generate entanglement between two nodes (henceforth referred to as the simulated quantum link).
First, we experimentally measure the latency incurred by a \acrshort{doa} system using
\acrshort{qkd}-powered authentication algorithms. We then analyze the behavior of the simulated
quantum link --- using a simulator for quantum networks~\cite{coopmans_2021_netsquid} --- where the
model of the control channel of the network includes the measured latencies incurred by the
\acrshort{qkd}-powered \acrshort{doa}. The contributions of this work are as follows:

\begin{enumerate}
    \item We provide a concise motivation for why \acrshort{doa} is a necessary component to uphold
          the availability of system-level protocols of the quantum network stack, and to help
          maintain the integrity of quantum data.
    \item We experimentally measure the added latencies incurred by \acrshort{doa}, performed using
          a \acrshort{qkd}-based authentication system, when run on a classical communication link.
    \item We offer a quantitative analysis of the impact of \acrshort{doa}, using the latencies
          measured as per the previous point, when applied to the quantum network protocols of a
          simulated quantum link (such link was first studied by \textcite{dahlberg_2019_egp}).
\end{enumerate}

\section{Related Work}
\label{sec:doa:relwork}

Conventional networking technology is an essential component of quantum networks and quantum
networking applications. \citeauthor{kozlowski_2019_towards} mention that the security of classical
communications is of concern when designing a quantum network~\cite{kozlowski_2019_towards}.
\citeauthor{satoh_2020_attacking} present a general motivation for authenticating the control
channel of a quantum link~\cite{satoh_2020_attacking}. They model attack vectors on quantum
communications through the lens of confidentiality, integrity, and availability. Without any
security measures in place, an attacker may:

\begin{itemize}
    \item Disrupt the network in any number of ways, affecting its \emph{availability}.
    \item Interfere with data sent via the network, hampering the \emph{integrity} of quantum data.
    \item Read out quantum data through the accompanying classical control data, affecting the
          \emph{confidentiality} of quantum data.
\end{itemize}

All identified proposals of quantum network designs and quantum network protocols use some form of
conventional communication to control and coordinate quantum
communication~\cite{van_meter_2013_repeaters, schoute_2016_shortcuts, joshi_2020_trusted,
pirker_2019_quantum, kozlowski_2019_towards, dahlberg_2019_egp, kozlowski_2020_qnp}. In this work,
we investigate one such proposal for a quantum network stack: the one put forth by
\textcite{dahlberg_2019_egp}, which has been evaluated in simulation, as well as on
hardware~\cite{pompili_2022_experimental}, and extended by \textcite{kozlowski_2020_qnp}.

The proposed protocol stack for quantum networks includes physical, link, network, transport, and
application layers, as illustrated in \cref{fig:functional-allocation}. The physical layer protocol,
called \acrfull{mhp}~\cite{dahlberg_2019_egp}, performs heralded entanglement generation attempts.
At the link layer, the \acrfull{qegp}~\cite{dahlberg_2019_egp} protocol has an internal retry
mechanism and performs coordination between adjacent nodes to provide more robust entanglement
generation. \acrshort{qegp} accepts two types of requests from the layer above:
%
\begin{inlinelist}
    \item \acrlong{ck} (\acrshort{ck}), to create an entangled pair and store it in memory;
    \item \acrlong{md} (\acrshort{md}), to create entangled pair, but measure it immediately, and
          report its outcome.
\end{inlinelist}
At the network layer, the \acrfull{qnp}~\cite{kozlowski_2020_qnp} protocol coordinates
entanglement generation and swap operations on a chain of nodes between two non-adjacent end nodes.

In this work, we simulate the three service-level protocols \acrshort{mhp}, \acrshort{qegp}, and
\acrshort{qnp}. In the next section, we explain why \acrfull{doa} is important for these protocols
to function, mostly focusing on \emph{availability} of the network and \emph{integrity} of quantum
data.

\section{Why Data Origin Authentication}
\label{sec:doa:why}

We investigate the applicability of \acrlong{doa} to three system-level protocols for quantum
network stacks: \acrshort{mhp}, \acrshort{qegp} and \acrshort{qnp}~\cite{dahlberg_2019_egp,
kozlowski_2020_qnp}. Quantum applications and their protocols themselves lie outside the scope of
this work. Here, we provide a non-exhaustive example list of actions that a malicious actor may
perform if they were able to forge or modify classical messages exchanged at the control protocol
level. We mention whether each action affects the \emph{availability} of the link or network, or the
\emph{integrity} of the quantum data sent via the network.

\paragraph{Physical layer}

\acrshort{mhp} operates at the physical layer. Hardware vulnerabilities of the physical entanglement
generation process are outside the scope of this work.

\begin{example}
\textit{Availability.}
Change a successful heralding signal to an error code such Alice and Bob falsely conclude that
entanglement has failed.
\end{example}

\begin{example}
\textit{Integrity.}
Modify a signal from a heralding station by changing the state announcement such that Alice or Bob
apply the wrong Pauli corrections to their qubits.
\end{example}

\begin{example}
\textit{Integrity.}
Interfere with mismatch verification~\cite{dahlberg_2019_egp, pompili_2022_experimental} such that
Alice and Bob falsely conclude that a \acrshort{mhp} request belongs to the same \acrshort{qegp}
request, thus hampering the integrity of quantum data due to cross-process interference.
\end{example}

\paragraph{Link layer}

\acrshort{qegp} operates at the link layer~\cite{dahlberg_2019_egp}. As proposed in
Ref.~\cite{dahlberg_2019_egp}, it is used to synchronize entanglement requests, and to communicate
the number of available memory qubits and the expiration of requests.

\begin{example}
\textit{Availability.}
Continually send requests for entanglement, exhausting the resources of nodes receiving
them~\cite{kozlowski_2019_towards}.
\end{example}

\begin{example}
\textit{Availability.}
When advertising the number of available communication qubits or storage qubits, set either to 0.
The receiving node then assumes that there are no communication or storage qubits available on the
sending node.
\end{example}

\begin{example}
\textit{Integrity.}
Change the qubit identifier of an entanglement generation request such that Alice and Bob entangle
the wrong data qubits, causing cross-path or process interference.
\end{example}

\paragraph{Network layer}

\acrshort{qnp} operates at the network layer. Conceptually, it allows non-adjacent nodes in a
quantum network to coordinate entanglement generation. This is akin to the \acrfull{ip} in the
classical stack. It makes use of \texttt{FORWARD} and \texttt{TRACK} messages to track entanglement
generation~\cite[Figure 6]{kozlowski_2020_qnp}. \texttt{FORWARD} messages are used to communicate
entanglement requests, and \texttt{TRACK} messages contain classical correction information for both
end-nodes in the network.

\begin{example}
\textit{Availability.}
Modify \texttt{FORWARD} message so that intermediary nodes do not receive entanglement swap
instructions.
\end{example}

\begin{example}
\textit{Integrity.}
Modify \texttt{TRACK} messages such that the end-nodes of a path do not receive the correct
entanglement swap outcome, and thus apply the wrong corrections to their qubits.
\end{example}

\begin{figure}[t]
    \centering
    \includegraphics[width=0.6\linewidth]{figures/doa-examples.pdf}
    \caption{
        A summary of example effects of tampering with control messages at each layer of the quantum
        network stack through the lens of confidentiality, integrity, and availability (CIA). We
        take as examples concrete implementations of quantum network protocols --- including
        \acrshort{qnp}, \acrshort{qegp} and \acrshort{mhp}~\cite{kozlowski_2020_qnp,
        dahlberg_2019_egp} --- to highlight the potential vulnerabilities at each layer of the
        stack. In this work, we do not focus on confidentiality issues, as identified by
        \textcite{satoh_2020_attacking}.
    }
    \label{fig:doa-examples}
\end{figure}

We illustrate the concerns of availability and data integrity in a quantum network stack in
\cref{fig:doa-examples}. Such security threats are addressed in part by using \acrlong{doa}, which
can prevent modification and forgery of classical control messages. It should be noted, however,
that availability may still be hampered by an adversary capable of halting transmission of classical
control messages outright. Furthermore, if a malicious actor were to gain access to a node itself
then this might circumvent many or all security mechanisms in place, including \acrshort{doa}, and
should therefore also be prevented.

In general, we also note that classical control messages might reveal information about who is using
the network and the types of operations being performed. Therefore, in a more mature network, it
will likely be worthwhile to also encrypt the contents of control messages.
\textcite{satoh_2020_attacking} mention tracking as a general concern for inter-node classical
communication within the quantum stack. Encryption combined with transmission of random noise could
address tracking concerns.

\section{Experimental Methodology}
\label{sec:doa:meth}

We aim to examine the effects of \acrlong{doa}, and the total latency incurred by the classical
control messages, on the performance of a hypothetical quantum link. To limit the scope of our
investigation, we focus on the performance of link and physical layer protocols as proposed by
\textcite{dahlberg_2019_egp}, and we extend the simulation therein such that it fully captures
%
\begin{inlinelist}
    \item classical transmission overhead, roughly defined as the latency of classical messages
          through the network and all networking equipment, and
    \item \acrshort{doa} overhead, roughly defined as the processing time required by the
          \acrshort{doa} systems.
\end{inlinelist}

Instead of modeling transmission and authentication overhead analytically, we measure it
experimentally over a real-world, physically deployed network, consisting of multiple datacenter
locations, sample quantum node controllers running FreeRTOS~\cite{freertos} on a MicroZed
board~\cite{microzed}, and complete \acrshort{doa} servers. The \acrshort{doa} servers exist at both
ends of our real-world network, and run message authentication protocols. From a networking
perspective, these servers act like transparent proxy servers, and are thus generally unknown and
unseen to the quantum controllers, but do authenticate control messages between the two ends of our
real-world deployed network. The \acrshort{doa} servers authenticate (and validate) control messages
using key material obtained from a physically-deployed \acrshort{qkd} system, also running between
the datacenter locations. The \acrshort{qkd} devices form a \acrfull{mdiqkd} network, based on the
work by \textcite{berrevoets_2022_deployed}. The \acrshort{qkd} devices run next to the
\acrshort{doa} servers, and deliver \acrshort{qkd}-key material via standard interface protocols.

We recorded \acrfull{rtt} statistics of sample control messages sent between the quantum controllers
--- and thus through the message authentication protocols of the \acrshort{qkd}-powered
\acrshort{doa} servers. We then injected these \acrshort{rtt} data, along with other simulation
parameters described below, to extract metrics, as done in Ref.~\cite{dahlberg_2019_egp}, and assess
the performance of a simulated quantum entanglement generating link The measured control
communication overheads is reported in \cref{sec:doa:latency}, while the simulation results are
presented in \cref{sec:doa:results}.

\paragraph{\acrshort{doa} servers}

We collect control communication latency measurements under three different configurations of the
\acrshort{doa} servers:

\begin{enumerate}
    \item Bypass servers: at first, we bypass the \acrshort{doa} servers altogether. This is useful
          to measure baseline communication latency, excluding all computation delays that would be
          introduced by the servers.
    \item No \acrshort{mac}: this time, the \acrshort{doa} servers are configured to skip the
          authentication (and verification) step, but packets do go through the servers' processing
          pipelines. Results for this configuration give us insights into the overhead of processing
          packets on the \acrshort{doa} servers, excluding the computation latency incurred by the
          \acrshort{mac} authentication (and verification) step.
    \item \acrshort{poly}: finally, we configure the \acrshort{doa} servers to use
          \acrshort{poly}~\cite{bernstein_2005_poly1305}, which is a popular, fast, and
          computationally secure \acrshort{mac}. In this configuration, a \acrshort{doa} server
          attaches a \num{128}-bit mac to each outgoing message.
\end{enumerate}

We do not report on a configuration where the \acrshort{doa} servers use an
information-theoretically secure \acrshort{mac}, as such an authentication scheme would consume
considerably more key material than the \acrshort{qkd} system could deliver. We did verify this
assessment by configuring the \acrshort{doa} servers to use
\acrshort{vmac}~\cite{krovetz_2007_message} --- an information-theoretically secure \acrshort{mac}
--- during which the \acrshort{doa} servers attempted to use considerably more key material than
available.

\begin{figure}[t]
    \centering
    \includegraphics[width=0.6\linewidth]{figures/mac-setup-qkd-locations-resized.png}
    \caption{
        Physical layout of Alice and Bob nodes. Alice is located in Groenekan, the Netherlands,
        while Bob is in Nieuwegein, the Netherlands. The optical fiber connecting Alice to the
        center node is \qty{20}{\km} in length, whereas the connection between Bob and the center
        node is \qty{22}{\km}.
    }
    \label{fig:mac-setup-qkd-locations}
\end{figure}

\paragraph{Network topology}

In our network deployment, we name the two end locations Alice and Bob. As is normal in
\acrshort{mdiqkd}, each end location is connected to a center hub. In the real-world network, the
link connecting Alice to the center hub runs for \qty{20}{\km}, whereas the link between Bob and the
center hub is \qty{22}{\km} in length. Thus, the \acrshort{rtt} we measure is for a link of a total
length of \qty{42}{\km}. The topology of Alice, Bob, and the center hub of the \acrshort{mdiqkd}
system is illustrated in \cref{fig:mac-setup-qkd-locations}. The control communication overhead
measured in \cref{sec:doa:latency} will be scaled to simulate quantum links of various lengths
(\qtyrange{2.5}{75}{\km}) in \cref{sec:doa:results}.

\begin{figure}[t]
    \centering
    \includegraphics[width=0.6\linewidth]{figures/mac-setup-diagram.pdf}
    \caption{
        Experimental setup used to measure the end-to-end delays of transmitting an
        \emph{authenticated classical message} from Alice to Bob with \qty{42}{\km} of fiber-optic
        cables between them. Classical messages go through a \acrshort{doa} server that tags
        messages using \acrshort{mdiqkd}-generated key material.
    }
    \label{fig:mac-setup-diagram}
\end{figure}

\section{Classical Communication Latency}
\label{sec:doa:latency}

In order to obtain an estimate of the expected latency incurred by classical control messages in a
quantum network stack, we measure the \acrfull{rtt} of a sample of messages when sent through an
authenticated classical channel in the field. The messages are \acrshort{icmp} (ping) packets, sized
to mimic \acrshort{mhp} and \acrshort{qegp} packets. The authentication mechanism is external to the
controllers exchanging ping messages, and runs on transparent \acrshort{doa} proxy servers between
them. The \acrshort{doa} server next to the sender calculates a \acrshort{mac} tag over the sent
message, forwards the message and the tag to the receiver, whose \acrshort{doa} servers verifies the
tag before delivering the message to the destination controller. The experimental setup is depicted
in \cref{fig:mac-setup-diagram}.

Ping messages are exchanged between two real-time classical network nodes similar to those used in
the experimental validation of \acrshort{qegp}~\cite{pompili_2022_experimental} --- MicroZed
boards~\cite{microzed} running a FreeRTOS~\cite{freertos} application --- connected to the
\acrshort{doa} servers via Gigabit Ethernet interfaces. The sending node records the \acrshort{rtt}
of the message, and computes various statistics on these timestamps, most notably average and
standard deviation. We use these statistics to extrapolate the expected control communication
latency for the various quantum link simulation scenarios presented in \cref{sec:doa:results}.

The \acrfull{mac} on both ends retrieves authentication keys from a local \acrshort{qkd} node
running a \acrshort{qkd} key server, which contains key material identical to its counterpart at the
other end. Key material is produced by an \acrshort{mdiqkd} implementation deployed in the Utrecht
area, in the Netherlands. The \acrshort{mdiqkd} system is very similar to the one implemented in
Ref.~\cite{berrevoets_2022_deployed}.

\paragraph{Rate and size of messages}

When using our particular \acrshort{doa} servers, one cannot transmit a classical message more than
once every \qty{10}{\ms} without experiencing detrimental packet loss. The reason for this is that
the \acrshort{doa} servers run prototype software designed for research purposes, and have not been
optimized for performance. We thus transmit ping messages at a rate of \qty{100}{\Hz}. Moreover, the
employed \acrshort{doa} servers can only authenticate packets with a payload that is small enough to
be accommodated (together with all protocol headers and the \acrshort{mac} tag) in a single Ethernet
frame. Therefore, we send ping messages with payloads of
%
\begin{inlinelist}
    \item \num{12}~bytes, which is the size of the smallest \acrshort{qegp} message, and
    \item \num{1200}~bytes, which is close to the maximum allowed by the \acrshort{doa} server.
\end{inlinelist}
The second configuration is useful to examine how much \acrshortpl{rtt} depend on the size of the
message.

\paragraph{Results}

We record the \acrlong{rtt} of 360,000 ping messages per configuration (\acrshort{doa} server
configuration and size of message). Mean and standard deviation of the measured \acrshortpl{rtt} are
reported in \cref{tab:rtt}, whilst the raw distributions of \acrshort{rtt} measurements are
presented in \cref{app:rtt}.

When bypassing the \acrshort{doa} servers altogether, \acrshort{rtt} delays follow a simple Gaussian
distribution, with a mean of less than \qty{4}{\ms} and a standard deviation of around
\qty{20}{\us}. When messages go through the \acrshort{doa} servers (configurations ``No
\acrshort{mac}'' and ``\acrshort{poly}''), \acrshort{rtt} measurements can be approximated by
bimodal Gaussian distributions (see \cref{app:rtt}). This bimodal distribution is likely the result
of some caching behavior exhibited by the \acrshort{doa} servers. For both configurations, one of
the two modes strongly dominates over the other, with the mean of this mode (around \qty{14}{\ms}
for all configurations and sizes) being approximately equal to the overall mean of the entire
distribution ($\pm\qty{2}{\percent}$ difference at most). \Cref{tab:rtt} reports mean and standard
deviation of the strongest modes of each distribution, which are deemed to be the most
representative statistics to use in our simulations. Refer to \cref{app:rtt} for a more in-depth
analysis of the \acrshort{rtt} distributions.

The noticeable gap between ``Bypass servers'' and the other two configurations is an indicator of
the suboptimal packet processing performance of the \acrshort{doa} server, which is merely a
soft-processing packet pipeline implemented in Python. The computational overhead introduced by the
actual \acrshort{mac} is overshadowed by the baseline latency of the \acrshort{doa} servers, as
observed in the ``\acrshort{poly}'' configuration. Interestingly, the size of messages does not
appear to be a noticeable factor in the mean end-to-end latency.

\begin{table}
    \centering
    \begin{tabularx}{0.75\linewidth}{@{} lYYYY @{}}
        \toprule
        \textbf{Server}                    & \textbf{Payload} & \multicolumn{2}{c}{\textbf{\acrshort{rtt}} [\unit{\us}]}                \\
        \cmidrule(l){3-4}
        \textbf{configuration}             & [Bytes]          & \textbf{mean}                                            & \textbf{std} \\
        \midrule
        \multirow{2}{*}{Bypass servers}    & \num{12}         & \num{3657}                                               & \num{23}     \\
                                           & \num{1200}       & \num{3881}                                               & \num{22}     \\
        \midrule
        \multirow{2}{*}{No \acrshort{mac}} & \num{12}         & \num{14036}                                              & \num{797}    \\
                                           & \num{1200}       & \num{14089}                                              & \num{767}    \\
        \midrule
        \multirow{2}{*}{\acrshort{poly}}   & \num{12}         & \num{14099}                                              & \num{861}    \\
                                           & \num{1200}       & \num{14015}                                              & \num{729}    \\
        \bottomrule
    \end{tabularx}
    \caption{
        Mean and standard deviation of \acrfull{rtt} for different configurations of the
        \acrshort{doa} servers and for different message sizes. Measurements for the ``Bypass
        servers'' configuration follow a simple Gaussian distribution. Measurements for the other
        two configurations can be approximated by bimodal Gaussian distributions, with one of the
        two modes strongly dominating over the other. This table reports mean and standard deviation
        of the strongest modes of each distribution. In the ``\acrshort{poly}'' configuration, the
        \acrshort{doa} servers attaches a \num{128}-bit key to each message.
    }
    \label{tab:rtt}
\end{table}

We can therefore conclude that latency is dominated by two main factors:
%
\begin{inlinelist}
    \item the performance of any classical networking hardware and the length of the classical link,
          which depend on the network technology and topology, and
    \item the packet processing performance of the \acrshort{doa} servers.
\end{inlinelist}
However, in a real production network, it is fair to assume that packets will be processed at a much
faster rate, and thus result in an end-to-end latency that is much more similar to that of the
``Bypass servers'' configuration. Considering also that authentication does not incur any noticeable
overhead --- as shown by the difference between the ``No \acrshort{mac}'' and ``\acrshort{poly}''
configurations --- these results support the case for both the use of \acrshort{doa} in quantum
networks, as well as the use of \acrshort{doa} supported by \acrshort{qkd} in present-day,
real-world production networks.

\section{Simulation Results}
\label{sec:doa:results}

We quantify the effects of using an authenticated classical channel for control messages exchanged
at the quantum network stack level. In particular, we augment the model used by
\textcite{dahlberg_2019_egp} to simulate the performance of physical (\acrshort{mhp}) and link
(\acrshort{qegp}) layer control protocols within the quantum network stack. As opposed to the
original work, we model the classical control communication latency to also include transmission and
authentication overhead as experimentally measured in our real-world network, and presented in
\cref{sec:doa:latency}. We report the \emph{throughput} of the quantum link, expressed as number of
entangled pairs generated per second.

We use \acrshort{rtt} statistics from \cref{tab:rtt} for our simulations. As described in
\cref{sec:doa:latency}, \acrshort{rtt} measurements for the ``Bypass servers'' configuration can be
approximated by a Gaussian distribution. \acrshortpl{rtt} for the ``No \acrshort{mac}'' and
``\acrshort{poly}'' configurations instead follow bimodal Gaussian distributions, with one of the
modes dominating over the other. For these two configurations, we use mean and standard deviation of
the strongest mode.

\paragraph{Configurations}

We run our simulations for two types of configurations:
%
\begin{inlinelist}
    \item To begin with, we analyze the quantum link throughput for several node-to-node distances
          (\qtyrange{2.5}{75}{\km}), at a fixed requested fidelity ($F_\text{min}=0.65$). For the
          various distances, we scale the classical delays measured in \cref{sec:doa:latency}
          accordingly. For this configuration, we compare our augmented model with the original
          baseline, in which transmission and authentication delays were not
          modeled~\cite{dahlberg_2019_egp}.
    \item Furthermore, we analyze the quantum link throughput at a fixed node-to-node distance
          (\qty{42}{\km} between the two nodes, same as in the real-world network in
          \cref{sec:doa:latency}), but this time varying the requested fidelity at the
          \acrshort{qegp} layer (fidelity \numrange{0.50}{0.85}). This time, we compare three
          models, corresponding to the three configurations of the \acrshort{doa} servers as per
          \cref{sec:doa:latency}: ``Bypass servers'', ``No \acrshort{mac}'', and
          ``\acrshort{poly}''.
\end{inlinelist}

For each of the above configurations, we perform \num{20} simulation runs, each consisting of
\num{20} minutes of continuous use of the quantum link. We then calculate the average throughput
over all runs for each configuration with a confidence interval of \qty{95}{\percent}.

\paragraph{Results}

\begin{figure}[t]
    \centering
    \includegraphics[width=0.6\linewidth]{figures/throughput-vs-distance.pdf}
    \begin{tabularx}{0.6\linewidth}{@{} lYYYYYYYY @{}}
        \toprule%
                        & \multicolumn{8}{c}{\textbf{Breakdown of distance A -- B [\unit{\km}]}}                                   \\%
        \midrule%
        \textbf{A -- M} & 1.5                                                                    & 3 & 6  & 9  & 15 & 22 & 30 & 45 \\%
        \textbf{M -- B} & 1                                                                      & 2 & 4  & 6  & 10 & 20 & 20 & 30 \\%
        \midrule%
        \textbf{Total}  & 2.5                                                                    & 5 & 10 & 15 & 25 & 42 & 50 & 75 \\%
        \bottomrule                             %
    \end{tabularx}%
    \caption{
        Throughput (rate) of entangled pair generation for multiple distances between node A and B,
        with a single midpoint station in between and for a requested fidelity of
        $F_\text{min}=0.65$. Classical communication delays were modeled using latency measurements
        collected as described in \cref{sec:doa:latency} --- configurations ``Bypass servers'' and
        ``\acrshort{poly}'' --- as well as replicated from the original simulations of \acrshort{qegp}
        and \acrshort{mhp}~\cite{dahlberg_2019_egp}. The simulated \acrshort{rtt} configuration
        represents the best-case scenario, given that only propagation delays are modeled in this
        one. The configuration ``\acrshort{poly}'' is the worst-case scenario instead, given that it
        models delays measured on the field, including the overhead incurred by the slow packet
        processing pipeline. For \acrshort{md} type requests, only the best case and the worst case
        are plotted, since they practically overlap, and thus any other scenario in between best and
        worst would not result in any significant difference. The table below the plot shows how
        distance is distributed between A, B, and the midpoint station M. The \qty{25}{\km} data
        point is equivalent to the QL2020 hypothetical setup simulated in
        Ref.~\cite{dahlberg_2019_egp}.
    }
    \label{fig:results-distance}
\end{figure}

The results for throughput versus distance are illustrated in \cref{fig:results-distance}. For
\acrfull{md} type requests (for which entanglement can be measured directly, as described in
earlier), mean throughput is approximately equal across the two models of added latency. This is to
be expected, given that these types of requests can be effectively pipelined --- that is, the next
entanglement request can be initiated right after the previous one without the need to wait for any
acknowledgment messages from a heralding station --- and thus throughput is mostly dominated by the
physical entanglement generation procedure, and not as much by classical communication latency. On
the other hand, \acrfull{ck} requests (for which entanglement must be stored while waiting for
acknowledgment messages, as described earlier) show a variation in throughput that depends on the
delay model. The best-case scenario --- corresponding to the original simulations of \acrshort{mhp}
and \acrshort{qegp}~\cite{dahlberg_2019_egp} (``Simulated RTT'') --- results in the highest
throughput, which peaks at around \num{6.47} pairs per second at the shortest distance. This is
followed by the scenario with ``Bypass servers'' delays, where classical communication latency has a
higher overhead than the previous case, but is more realistic. In this case, the shortest link can
deliver around \num{1.42} entangled pairs per second. Finally, the worst-case scenario of
``\acrshort{poly}'', where classical communication delays also include the overhead of the slow
packet processing pipeline of the \acrshort{doa} servers, can yield a little less than one pair per
second. This decrease in throughput is expected, but fortunately, it is also not detrimental to the
point that the quantum link becomes non-functional altogether. Naturally, final numbers depend
strongly on the quantum properties of the quantum memories (quantum coherence time). Conceivably, if
the coherence time of the quantum memories was longer than the added latency of \acrshort{doa}
servers, the decrease in performance would be less significant.

The results for throughput versus fidelity are shown in \cref{fig:results-fidelity}. Again, the
outcome matches our expectations. \acrshort{md} requests are not significantly affected by classical
communication, and their throughput only decreases when higher-fidelity entangled pairs are
requested. \acrshort{ck} requests, instead, are more affected by classical communication latency,
and their throughput is a lot lower than that of \acrshort{md} requests. Additionally, throughput
also decreases when classical messages go through the \acrshort{doa} servers (``Bypass servers''
versus ``No \acrshort{mac}''). As expected from the results in \cref{tab:rtt}, the extra latency
incurred by the actual \acrshort{mac} computation are negligible, and its effect on throughput not
noticeable (``No \acrshort{mac}'' versus ``\acrshort{poly}'').

\begin{figure}[t]
    \centering
    \includegraphics[width=0.6\linewidth]{figures/throughput-vs-fidelity.pdf}
    \caption{
        Throughput (rate) of entangled pair generation for multiple requested fidelities, with
        \qty{20}{\km} between node A and the midpoint station, and \qty{22}{\km} between the
        midpoint station and node B, and for three configurations of the proxy as per
        \cref{sec:doa:latency}: Bypass servers, No \acrshort{mac}, and \acrshort{poly}. The solid line
        represents mean throughput, the colored area around it depicts the \qty{95}{\percent}
        confidence interval. \Acrfull{md} type requests are not as heavily affected as these are
        pipelined. \Acrfull{ck} type requests are performed sequentially, and thus increases in
        classical delays have a more profound impact on throughput.
    }
    \label{fig:results-fidelity}
\end{figure}

To summarize, our experiments and simulations demonstrate that
%
\begin{inlinelist}
    \item in real-world quantum links, throughput is likely to be worse than what results from
          simulations that do not fully model transmission delays of messages stemming from
          latencies of conventional communication networks,
    \item the overhead incurred by classical communications is sizeable, but does not outright
          disrupt the operation of quantum links, and
    \item delays incurred by \acrlong{doa} are negligible.
\end{inlinelist}

\section{Discussion}

We have shown how the classical data plane of a quantum network stack presents a significant attack
surface for confidentiality, integrity, and availability of the quantum link and data therein. To
address these concerns one must employ, among other things, \acrlong{doa} on the classical control
messages exchanged at the quantum network stack level. We conclude that \acrlong{doa} is necessary
to both uphold the integrity of quantum data and the availability of the quantum network itself.

We have also simulated the performance of a hypothetical quantum link under the assumption that
control messages are authenticated by conventional \acrshort{doa} techniques. Here, we modeled
classical communication latency, including authentication overhead and transmission delay, using
metrics collected from a real-world communication link, authenticated using \acrshort{doa} servers,
using key material sourced from a \acrshort{qkd} system. If we disregard the large packet processing
delays incurred by the \acrshort{doa} servers itself --- the delays of the servers packet pipeline,
and not the delays of the authentication software running on the server --- we observe that an
authenticated classical control channel introduces a negligible amount of extra classical overhead.

However, we have also seen that even just transmission delays have a noticeable effect on the
performance of a quantum link, whether or not \acrlong{doa} is applied --- entanglement generation
throughput drops when classical communication delays are larger. Nevertheless, we have observed in
our simulations that even when the entangled qubits must be stored in quantum memories with limited
lifetimes, the quantum link remains operational.

\begin{xstretch}
\printbibliography[heading=subbibintoc,title={References},notcategory=noprint]
\end{xstretch}

\chapter{Conclusion}
\label{chp:conclusion}

\lettrine{M}{ost} interesting conclusions.

The conclusions of this thesis.

\printbibliography[heading=subbibintoc,title={References},notcategory=noprint]


% Use letters for the chapter numbers of the appendices.
\appendix

\chapter{\acrshort{qnodeos} Components and Interfaces} \label{app:qnodeos}

We provide here additional details on the components of the \acrshort{qnodeos} architecture and
their intefaces. \Cref{fig:qnodeos-core} gives an overview of all the components of
\acrshort{qnodeos}. The \emph{process manager} marshals accesses to all user and kernel processes.
The \emph{scheduler} assigns ready processes to the \emph{processor}, which runs quantum
instructions through the underlying \acrshort{qdevice}, processes classical QASM instructions
locally, and registers entanglement requests with the \emph{\acrlong{emu}} (\acrshort{emu}). The
\acrshort{emu} maintains a list of EPR sockets and entanglement requests, forwards the latter to the
\emph{quantum network stack}, which, in turn, registers available entangled qubits with the
\acrshort{emu}. Finally, the \emph{\acrlong{qmmu}} (\acrshort{qmmu}) keeps track of used qubits, and
transfers qubit ownership across processes when requested.

\section{Process Manager}

The process manager owns \acrshort{qnodeos} processes and marshals accesses to those. Creating a
process, adding a routine to it and accessing the process's data must be done through the process
manager. Additionally, the process manager is used by other components to notify \emph{events} that
occur inside \acrshort{qnodeos}, upon which the state of one of more processes is updated. Process
state updates result in a notification to the scheduler.

\paragraph{Interfaces}

The process manager exposes interfaces for three services:
\begin{itemize}
    \item Process management (interface~\circled{1} in \cref{fig:qnodeos-core}): to create and
          remove processes, and to add routines to them. When the user registers an application, the
          \acrshort{qnodeos} \acrshort{api} Handler uses the process manager to create a
          \acrshort{qnodeos} user process. The returned process ID can be later used to add a
          routine to that process, or to remove the process once all its routines are fully
          processed.
    \item Event notification (interface~\circled{2} in \cref{fig:qnodeos-core}): to notify an event
          occurred inside \acrshort{qnodeos}, including the addition of a routine, the completion of
          a routine, the scheduling of the process, the hitting of a Wait condition, and the
          generation of an entangled qubit destined to the process. Some events trigger follow-up
          actions --- for instance, when a process that was waiting for an event becomes ready, it
          gets added to the queue of ready processes maintained by the scheduler.
    \item Process data access (interface~\circled{3} in \cref{fig:qnodeos-core}): to access a
          process's routines and its classical memory space, mostly used while running the process
          (through the processor).
\end{itemize}

\begin{figure}[t]
    \centering
    \includegraphics[width=\linewidth]{figures/qnodeos-core.pdf}
    \caption[]{
        \acrshort{qnodeos} core components and internal interfaces. The core layer includes:
        \begin{enumerate*}[label=(\arabic*)]
            \item a \emph{process manager} (ProcMgr), which owns and manages access to
                  \acrshort{qnodeos} processes;
            \item a \emph{scheduler}, responsible for selecting the next process to be run;
            \item a \emph{processor}, which processes routines' instructions;
            \item an \emph{\acrlong{emu}} (\acrshort{emu}), which keeps a list of entanglement
                  requests and available entangled qubits;
            \item a \emph{quantum network stack} (QNetStack), whose responsibility is to coordinate
                  with peer nodes to schedule quantum networking instructions;
            \item a \emph{\acrlong{qmmu}} (\acrshort{qmmu}), which keeps a record of allocated
                  qubits.
        \end{enumerate*}
    }
    \label{fig:qnodeos-core}
\end{figure}

\section{Scheduler}

The \acrshort{qnodeos} scheduler registers processes that are ready to be scheduled, and assigns
them to the \acrshort{qnodeos} processor when the latter is available. Ready processes are stored in
a \emph{prioritized ready queue}, and processes of the same priority are scheduled with a
first-come-first-served policy.

\paragraph{Interfaces}

The scheduler only exposes one interface for process state notifications (interface~\circled{4} in
\cref{fig:qnodeos-core}), used by the process manager to signal when a process transitions to a new
state. When a \acrshort{qnodeos} process transitions to the ready state, it is directly added to the
scheduler's prioritized ready queue. When a process becomes idle, or is waiting for an event to
happen, the scheduler simply registers that the processor has become available.

\section{Processor}

The \acrshort{qnodeos} processor handles the execution of \acrshort{qnodeos} user and kernel
processes, by running classical instructions locally and issuing quantum instructions to the
\acrshort{qdevice} driver. While executing a process, the processor reads its routines and accesses
(reads and writes) its classical memory. The processor implements a specific instruction set
architecture dictated by the QASM language of choice.

\paragraph{Interfaces}

The processor exposes one interface for processor assignment (interface~\circled{5} in
\cref{fig:qnodeos-core}), used by the \acrshort{qnodeos} scheduler to activate the processor, when
it is idling, and assign it to a \acrshort{qnodeos} process.

\section{Entanglement Management Unit}

The \acrfull{emu} maintains a list of open \emph{EPR sockets} and a list of \emph{entanglement
requests}, and keeps track of the \emph{entangled qubits} produced by the quantum network stack.
Received entanglement requests are considered valid only if an EPR socket associated to such
requests exists. Valid requests are forwarded to the quantum network stack. Entangled qubit
generations are notified as events to the process manager.

\paragraph{Interfaces}

The \acrshort{emu} exposes interfaces for three services:
\begin{itemize}
    \item EPR socket registration (interface~\circled{6} in \cref{fig:qnodeos-core}): to register
          and open EPR sockets belonging to an application, and to set up internal classical network
          tables and to establish classical network connection.
    \item Entanglement request registration (interface~\circled{7} in \cref{fig:qnodeos-core}): to
          add entanglement requests to the list of existing ones, to be used when matching produced
          entangled qubits with a process that requested them.
    \item Entanglement notification (interface~\circled{8} in \cref{fig:qnodeos-core}): to register
          the availability of an entangled qubit, produced by the quantum network stack, and to link
          it to an existing entanglement request.
\end{itemize}

\section{Quantum Network Stack}

The quantum network stack on \acrshort{qnodeos} follows the model presented in
Ref.~\cite{dahlberg_2019_egp} which is based on the classical \acrshort{osi} network stack model for
separation of responsibilities. In particular, \emph{data link layer} and \emph{network layer}
protocols are part of the quantum network stack on \acrshort{qnodeos}. The \emph{physical layer} is
implemented on the \acrshort{qdevice}, the \emph{application layer} is part of the Host, and all
remaining layers are not currently part of the stack.

The quantum network stack component has an associated \emph{\acrshort{qnodeos} kernel process},
created statically on \acrshort{qnodeos}. However, this process's routine is dynamic: the
instructions to be executed on the processor depend on the outstanding entanglement generation
requests received from \acrshort{emu} and network peers.

\paragraph{Interfaces}

The quantum network stack exposes interfaces for two services:
\begin{itemize}
    \item Entanglement request registration (interface~\circled{9} in \cref{fig:qnodeos-core}): to
          add entanglement requests coming from the \acrshort{emu} to the list of existing ones,
          which are used to fill in the quantum network stack process's routine with the correct
          instructions to execute.
    \item Entanglement request synchronization (interface~\circled{10} in \cref{fig:qnodeos-core}):
          similar to the entanglement request registration interface, but to be used to synchronize
          (send and receive) requests with \acrshort{qnodeos} network peers.
\end{itemize}

\section{Quantum Memory Management Unit}

The \acrfull{qmmu} receives requests for \emph{qubit allocations} from \acrshort{qnodeos} processes,
and manages the subsequent usage of those. It also translates QASM \emph{virtual qubit addresses}
into physical addresses for the \acrshort{qdevice}, and keeps track of which process is using which
qubit at a given time. In general, a \acrshort{qmmu} should take into account that the topology of a
quantum memory determines what operations can be performed on which qubits, and thus allow processes
to allocate qubits of a specific type upon request. An advanced \acrshort{qmmu} could also feature
algorithms to move qubits in the background --- that is, without an explicit instruction from a
process's routine --- to accommodate an application's topology requirements while not trashing the
qubits being used by other \acrshort{qnodeos} processes. Such a feature could prove crucial to
increase the number of processes that can be using the quantum memory at the same time, and to
enhance multitasking performances.

\paragraph{Interfaces}

The \acrshort{qmmu} exposes interfaces for three services:
\begin{itemize}
    \item Qubit allocation and deallocation (interface~\circled{11} in \cref{fig:qnodeos-core}): a
          running process can ask for one or more qubits, which, if available, are allocated by the
          \acrshort{qmmu}, and their physical addresses are mapped to the virtual addresses provided
          by the requesting process.
    \item Virtual address translation (interface~\circled{12} in \cref{fig:qnodeos-core}): before
          sending quantum instructions to the \acrshort{qdevice} driver, the processor uses virtual
          qubit addresses specified in QASM to retrieve physical addresses from the \acrshort{qmmu},
          and then replaces virtual addresses with physical addresses in the instructions for the
          \acrshort{qdevice} driver.
    \item Qubit ownership transfer (interface~\circled{13} in \cref{fig:qnodeos-core}): qubits are
          only visible to the process that allocates them. However, in some cases, a process may
          wish to transfer some if its qubits to another one. A notable example is the quantum
          network process transferring an entangled qubit to the process that will use it.
\end{itemize}

\printbibliography[heading=subbibintoc,title={References}]

\chapter{\acrshort{qdevice} Interface} \label{app:qdevice}

The implementation of a \acrshort{qdevice} depends on a number of factors. Most importantly, the
physical signals that are fed to the quantum processing and networking device, and those that are
output from the device, are specific to the nature of the device itself. Different qubit
realizations require different digital and analog control. For instance, manipulating the state of a
spin-based qubit (e.g., in a nitrogen-vacancy center processor) and that of an ultracold atom qubit
(e.g., in a trapped ion processor) are two physical processes that vastly differ in a number of
complicated ways.

For \acrshort{qnodeos} to be portable to a diverse set of quantum physical platforms, there needs to
be a common \emph{\acrshort{qdevice} interface} that \acrshort{qnodeos} can rely on, and that each
\acrshort{qdevice} instance can implement as it is most convenient for the underlying quantum
device. This interface need be quite general, to be able to express all quantum operations that
different quantum devices might be capable of performing, and rather abstract, so that two different
implementations of a well-defined qubit manipulation operation can be expressed with the same
instruction on \acrshort{qnodeos}. Nevertheless, an interface that is too general could result in a
high implementation complexity on the \acrshort{qdevice}, as it might have to transform high-level
instructions in a series of native operations on the fly. Other than complexity of implementation, a
very high-level set of \acrshort{qdevice} instructions might compromise the compiler's ability to
optimize an application for a certain physical platform, as reported by
\textcite{murali_2019_fullstack}.

Defining a set of instructions to express abstract quantum operations as close as possible to what
different quantum physical platforms can natively perform is, to some extent, an open problem. While
this is outside the scope of this work, we have made an effort to specify an interface which is a
good compromise between generality and expressiveness. The \acrshort{qdevice} interface is
essentially a set of instructions that \acrshort{qnodeos} expects a \acrshort{qdevice} to implement.
To be precise, a \acrshort{qdevice} might implement a subset of the interface, according to what
native physical operations it can perform. The Host compiler must then have knowledge about the set
of instructions implemented by the underlying \acrshort{qdevice}, so that it can decompose
instructions that are not natively supported.

Even though this interface does not impose any formal timing constraints, it is important to note
that a \acrshort{qdevice} implementation that tries to guarantee more or less deterministic
instruction processing latencies can prove more beneficial to the real-time requirements of
\acrshort{qnodeos}. Particularly, it would be advisable to time-bound the processing time of
operations whose duration is by nature probabilistic --- most notably, those involving entanglement
generation. Creating an entangled pair may involve a varying number of attempts. Sometimes, if the
remote node becomes unresponsive for a period of time, the number of necessary attempts can increase
by a large amount. Capping the number of attempts could, for instance, provide a more deterministic
maximum processing latency for entanglement instructions, which in turn might help
\acrshort{qnodeos} react more timely to temporary failures or downtime periods of remote nodes. Not
to mention that unbounded entanglement attempts affect the state of other qubits in memory, because
of both passive decoherence and cross-qubit noise.

\begin{table}[t]
    \centering
    \begin{tabularx}{0.75\linewidth}{>{\ttfamily}l l}
        \toprule
        \normalfont{Instruction} & Description                                  \\
        \midrule
        INI                      & Initialize a qubit to default state          \\
        SQG                      & Perform a single-qubit gate                  \\
        TQG                      & Perform a two-qubit gate                     \\
        AQG                      & Perform a gate on all qubits                 \\
        MSR                      & Measure a qubit in a specified basis         \\
        ENT                      & Attempt entanglement generation              \\
        ENM                      & Attempt entanglement and measure qubit       \\
        MOV                      & Move qubit state to another qubit            \\
        SWP                      & Swap the state of two qubits                 \\
        ESW                      & Swap qubits belonging to two entangled pairs \\
        PMG                      & Set pre-measurement gates                    \\
        \bottomrule
    \end{tabularx}
    \caption{
        Summary of \acrshort{qdevice} instructions defined in the \acrshort{qdevice} interface. A
        specific \acrshort{qdevice} might implement a subset of these, depending on the underlying
        quantum physical device and on other design constraints.
    }
    \label{tab:qdevice-instructions}
\end{table}

\Cref{tab:qdevice-instructions} lists the complete set of instructions defined in the
\acrshort{qdevice} interface. Instructions can have operands, whose range of valid values depends on
the underlying \acrshort{qdevice}. For instance, an operand that specifies which qubit to apply an
operation to can only have as many valid values as there are physical qubits in memory. Details for
each instruction and its operands are given below.

\paragraph{Qubit initialization (\texttt{INI})}

The \texttt{INI} instruction brings a qubit to the $\ket{0}$ state. On some physical platforms,
single-qubit initialization is not possible, thus this instruction initializes all qubits to the
$\ket{0}$ state.

\smallskip\noindent
\begin{tabularx}{\linewidth}{>{\ttfamily}l X}
    \toprule
    \normalfont{Operand} & Description                                                                                                         \\
    \midrule
    qubit                & Physical address of the qubit to initialize, ignored on platforms where single-qubit initialization is not possible \\
    \bottomrule
\end{tabularx}
\medskip

\paragraph{Single-qubit gate (\texttt{SQG})}

The \texttt{SQG} instruction manipulates the state of one qubit. The gate is expressed as a rotation
in the Bloch sphere.

\smallskip\noindent
\begin{tabularx}{\linewidth}{>{\ttfamily}l X}
    \toprule
    \normalfont{Operand} & Description                                                                  \\
    \midrule
    qubit                & Physical address of the qubit to manipulate                                  \\
    axis                 & Rotation axis, can be X, Y, Z or H (support is \acrshort{qdevice}-dependent) \\
    angle                & Rotation angle (granularity and range are \acrshort{qdevice}-dependent)      \\
    \bottomrule
\end{tabularx}
\medskip

\paragraph{Two-qubit gate (\texttt{TQG})}

The \texttt{TQG} instruction manipulates the state of two qubits. The gate is expressed as a
controlled rotation, with one qubit being the control and the other one being the target.

\smallskip\noindent
\begin{tabularx}{\linewidth}{>{\ttfamily}l X}
    \toprule
    \normalfont{Operand} & Description                                                                  \\
    \midrule
    qub\_c               & Physical address of the control qubit                                        \\
    qub\_t               & Physical address of the target qubit                                         \\
    axis                 & Rotation axis, can be X, Y, Z or H (support is \acrshort{qdevice}-dependent) \\
    angle                & Rotation angle (granularity and range are \acrshort{qdevice}-dependent)      \\
    \bottomrule
\end{tabularx}
\medskip

\paragraph{All-qubit gate (\texttt{AQG})}

The \texttt{AQG} instruction manipulates the state of all available qubits. The gate is expressed as
a rotation in the Bloch sphere.

\smallskip\noindent
\begin{tabularx}{\linewidth}{>{\ttfamily}l X}
    \toprule
    \normalfont{Operand} & Description                                                                  \\
    \midrule
    axis                 & Rotation axis, can be X, Y, Z or H (support is \acrshort{qdevice}-dependent) \\
    angle                & Rotation angle (granularity and range are \acrshort{qdevice}-dependent)      \\
    \bottomrule
\end{tabularx}
\medskip

\paragraph{Qubit measurement (\texttt{MSR})}

The \texttt{MSR} instruction measures the state of one qubit in a specified basis. A qubit
measurement is destructive --- that is --- the qubit has to be reinitialized before it can be used
again.

\smallskip\noindent
\begin{tabularx}{\linewidth}{>{\ttfamily}l X}
    \toprule
    \normalfont{Operand} & Description                                                                    \\
    \midrule
    qubit                & Physical address of the qubit to measure                                       \\
    basis                & Measurement basis, can be X, Y, Z, H (support is \acrshort{qdevice}-dependent) \\
    \bottomrule
\end{tabularx}
\medskip

\paragraph{Entanglement generation (\texttt{ENT})}

The \texttt{ENT} instruction performs a series of entanglement generation attempts, until one
succeeds, or until a maximum number of attempts is reached (the behavior is
\acrshort{qdevice}-dependent).

\smallskip\noindent
\begin{tabularx}{\linewidth}{>{\ttfamily}l X}
    \toprule
    \normalfont{Operand} & Description                                                                                               \\
    \midrule
    nghbr                & Neighboring node to attempt entanglement with, if the local \acrshort{qdevice} has multiple quantum links \\
    fid                  & Target entanglement fidelity (granularity and range are \acrshort{qdevice}-dependent)                     \\
    \bottomrule
\end{tabularx}
\medskip

\paragraph{Entanglement generation with qubit measurement (\texttt{ENM})}

The \texttt{ENM} instruction performs a series of entanglement generation attempts followed by an
immediate measurement of the local qubit, until one succeeds, or until a maximum number of attempts
is reached (the behavior is \acrshort{qdevice}-dependent).

\smallskip\noindent
\begin{tabularx}{\linewidth}{>{\ttfamily}l X}
    \toprule
    \normalfont{Operand} & Description                                                                                               \\
    \midrule
    nghbr                & Neighboring node to attempt entanglement with, if the local \acrshort{qdevice} has multiple quantum links \\
    fid                  & Target entanglement fidelity (granularity and range are \acrshort{qdevice}-dependent)                     \\
    basis                & Measurement basis, can be X, Y, Z, H (support is \acrshort{qdevice}-dependent)                            \\
    \bottomrule
\end{tabularx}
\medskip

\paragraph{Qubit move (\texttt{MOV})}

The \texttt{MOV} instruction moves the state of one qubit to another qubit. A qubit move renders the
state of the source qubit undefined, and the qubit has to be reinitialized before it can be used
again.

\smallskip\noindent
\begin{tabularx}{\linewidth}{>{\ttfamily}l X}
    \toprule
    \normalfont{Operand} & Description                               \\
    \midrule
    qub\_s               & Physical address of the source qubit      \\
    qub\_d               & Physical address of the destination qubit \\
    \bottomrule
\end{tabularx}
\medskip

\paragraph{Qubit swap (\texttt{SWP})}

The \texttt{SWP} instruction swaps the state of two qubits.

\smallskip\noindent
\begin{tabularx}{\linewidth}{>{\ttfamily}l X}
    \toprule
    \normalfont{Operand} & Description                          \\
    \midrule
    qub\_1               & Physical address of the first qubit  \\
    qub\_2               & Physical address of the second qubit \\
    \bottomrule
\end{tabularx}
\medskip

\paragraph{Entanglement swap (\texttt{ESW})}

The \texttt{ESW} instruction results in two qubits belonging to two entangled pairs to have their
roles swapped.

\smallskip\noindent
\begin{tabularx}{\linewidth}{>{\ttfamily}l X}
    \toprule
    \normalfont{Operand} & Description                          \\
    \midrule
    qub\_1               & Physical address of the first qubit  \\
    qub\_2               & Physical address of the second qubit \\
    \bottomrule
\end{tabularx}
\medskip

\paragraph{Pre-measurement gates setting (\texttt{PMG})}

The \texttt{PMG} instruction allows for a set of (up to) 3 rotations to be performed before a qubit
measurement (\texttt{MSR} or \texttt{ENM}). If the axis of the second rotation is orthogonal to the
axis of the first and the third rotation, these gates can be used to perform a qubit measurement in
an arbitrary basis, given that most likely a \acrshort{qdevice} can natively measure in a limited
set of bases.

\smallskip\noindent
\begin{tabularx}{\linewidth}{>{\ttfamily}l X}
    \toprule
    \normalfont{Operand} & Description                                                                                                                                             \\
    \midrule
    axes                 & Combination of orthogonal axes to use for the three successive rotations, can be X--Y--X, Y--Z--Y and Z--X--Z (support is \acrshort{qdevice}-dependent) \\
    ang\_1               & Rotation angle of the first gate, relative to the first axis in \texttt{axes} (granularity and range are \acrshort{qdevice}-dependent)                  \\
    ang\_2               & Rotation angle of the second gate, relative to the second axis in \texttt{axes} (granularity and range are \acrshort{qdevice}-dependent)                \\
    ang\_3               & Rotation angle of the third gate, relative to the third axis in \texttt{axes} (granularity
    and range are \acrshort{qdevice}-dependent)                                                                                                                                    \\
    \bottomrule
\end{tabularx}

\printbibliography[heading=subbibintoc,title={References}]

\chapter{Implementation of the Quantum Physical Layer}
\label{app:phys}

We provide here additional details concerning the implementation of the quantum physical layer used
for our experiments.

\paragraph{Single qubit gates}

At the physical layer, we implement real-time rotations around the X and Y axes of the qubit Bloch
sphere, using a resolution of $\pi/16=$\ang{11.25}. That is, the upper layer can request any
rotation that is a multiple of $\pi/16$ around either the X or Y axis. The different rotations are
performed using Hermite-shaped pulses (as described in Ref.~\cite{pompili_2021_multinode}) of
calibrated amplitude. The choice of X(Y) rotation axis is implemented using the I(Q) channel of the
microwave vector source.

While supported on \acrshort{qnodeos}, our physical layer currently does not implement Z-axis
rotations. Such rotations around the Z axis could be implemented by virtual rotations of the Bloch
sphere: a $\pi$ pulse around the Z axis is equivalent to multiplying future I and Q voltages by
$-1$. By keeping track of the accumulated Z rotations, and by adjusting I and Q mixing accordingly,
one can perform effective Z rotations with very high resolution and virtually no infidelity. The
\acrshortpl{awg} currently in use have the required capabilities, and the implementation of said Z
gates is planned for the near future.

\paragraph{Clock sharing and \acrshort{awg} triggering over longer distances}

One of the technical challenges of realizing a large scale quantum network is synchronizing
equipment at the physical layer across nodes. The synchronization is required to generate
entanglement --- the photons from the two nodes need to arrive at the same time at the heralding
station (compared to their duration, \qty{12}{\ns} for \acrshort{nv} centers in bulk diamond
samples); failing to do so would reduce (or even remove) their indistinguishability, which is
required to establish long-distance entanglement~\cite{pompili_2021_multinode}. Our two nodes are
located in a single laboratory, on the same optical table, approximately \qty{2}{\m} apart. This
allows for some simplifications, for the purpose of demonstrating entanglement delivery using a
network stack, which would not be possible over longer distances. Specifically:

\begin{enumerate}
    \item We use a single laser --- the client's --- to excite both nodes, as in
          Ref.~\cite{pompili_2021_multinode}. Over longer distances, one would need to phase-lock
          the excitation lasers at the two nodes to ensure phase-stability of the entangled states.
    \item The Device Controllers (ADwin Pro II microcontroller) are triggered every \qty{1}{\us} by
          the same signal generator, advancing the state machine algorithm that implements the
          physical layer. This ensures that the two microcontrollers have a common shared clock.
          Over longer distances, one could use existing protocols (and commercially-available
          hardware) to obtain a shared clock~\cite{whiterabbit}, and use that to trigger the
          microcontrollers.
    \item The two \acrshortpl{awg} need to be triggered to play entanglement attempts. In our
          implementation, one device controller --- the server's --- triggers both \acrshortpl{awg}.
          This ensures that the triggering delay between the two \acrshortpl{awg} is constant, and
          we can therefore calibrate it out. Triggering the \acrshortpl{awg} with two independent
          microcontrollers would result in jitter (realistically on the order of nanoseconds). Over
          larger distances, one could derive --- from the shared clock --- a periodic trigger signal
          that is gated by the microcontroller at each node. In this way the microcontroller can
          decide whether the \acrshort{awg} will be triggered on the next cycle, but the accuracy of
          the trigger's timing will be derived from the shared clock between the nodes, rather than
          from the microprocessor.
    \item The phase stabilization scheme we use, developed in Ref.~\cite{pompili_2021_multinode}, is
          designed to work at a single optical frequency (in our case, the \qty{637}{\nm} emission
          frequency of the \acrshort{nv} center). Over longer distances, conversion of the
          \acrshort{nv} center photons to the telecom band will be necessary to overcome photon
          loss. The phase stabilization scheme will therefore need to be adapted to new optical
          frequencies used.
\end{enumerate}

For reference, our client (server) is based on node Charlie (Bob) of the multi-node quantum network
presented in Ref.~\cite{pompili_2021_multinode}.

\paragraph{NV center resonance control}

The two quantum network nodes use different techniques to control the resonance of their
\acrshort{nv} centers (see Ref.~\cite{pompili_2021_multinode} for implementations details). The
server uses an off-resonant charge randomization strategy: when its \acrshort{nv} center is not on
resonance (it does not pass the charge and resonance check), it can apply an off-resonant (green,
\qty{515}{\nm}) laser pulse to shuffle the charge environment and probabilistically recover the
correct charge and resonance state. The server cannot get \emph{stuck} in a non-resonance state: in
a few tens of failed \acrshortpl{crcheck} and green laser pulses (overall less than \qty{1}{\ms})
the \acrshort{nv} center will be in resonance again.

The client, which needs to be tuned in resonance with the other node, uses a resonant strategy. When
in the wrong charge state (zero counts during \acrshort{crcheck}), it applies a resonant laser pulse
(yellow, \qty{575}{\nm}, \acrshort{nv}${}^0$ zero-phonon line) to go back to \acrshort{nv}${}^-$. To
bring \acrshort{nv}${}^-$ in resonance with the necessary lasers, it adjusts a biasing voltage
applied to the diamond sample, which shifts the resonance frequencies. This process is mostly
automated. However, occasional human intervention is still required when the resonance frequencies
of the \acrshort{nv} center shift too far --- for example due to a charge in the vicinity of the
\acrshort{nv} center changing position in the lattice --- for the automatic mechanism to find its
way back. Periods of inactivity in entanglement generation are due to the jumps in the client's
\acrshort{nv} optical transitions, which then require manual optimization of the laser frequencies
and/or the diamond biasing voltage --- depending on the magnitude of the frequency shift, it
requires tens of seconds to a few minutes to recover the optimal resonance condition.

\printbibliography[heading=subbibintoc,title={References}]

% \chapter{Introduction --- The Bot's Take}
\label{app:intro}

\begin{abstract}
Writing the actual introduction to this thesis (\cref{chp:intro}) took ...
\end{abstract}

% \chapter{Soundtrack}
\label{app:soundtrack}

As a musician (?) and music enthusiast, I always try to find a soundtrack that fits whatever I am
doing or thinking in a certain situation. Very much like in a movie, soundtracks can be mash-ups of
existing songs and organic contextual sounds, or sometimes more though-out compositions. Running
applications on an experimental quantum network using an experimental operating system can be very
frustrating at times: the hardware can break, the software can be buggy, the entities running the
simulated reality~\cite{bostrom_2003_simulation} can be pesky. Therefore, those scenes are typically
backed by impromptu rambling noises --- thudding, stomping, finger-snapping, thigh-smacking. On a
good day, the satisfaction of a successful experiment turns the percussive soundtracks into a more
melodic, climax-resolving tune --- think of something like \citetitle{gorillaz_feelgoodinc} by
\textcite{gorillaz_feelgoodinc}, or perhaps \citetitle{tool_lateralus} by \textcite{tool_lateralus}
if I am feeling more (bare) metal that day.

However, I felt that the experimental acts of this thesis needed their ad-hoc piece, something that
more adequately represented my feelings towards what in retrospect looks like a fantastic journey
through physics and computers. For that matter, I am offering an alternative interpretation of the
results obtained... post-processed through this Python script... and embellished with some
human-made percussion sounds. The outcome of this final experiment is available at...

\printbibliography[heading=subbibintoc,title={References}]


% Turn off thumb indices for unnumbered chapters.
\thumbfalse

% Print glossary.
% Notice the use of \printnoidxglossary instead of \printglossary to avoid running external tools
% (https://github.com/tectonic-typesetting/tectonic/issues/704).
\glsaddall
\printnoidxglossary[type=\acronymtype,title={Glossary}]
\addcontentsline{toc}{chapter}{Glossary}
\setheader{Glossary}

\chapter*{Acknowledgments}
\addcontentsline{toc}{chapter}{Acknowledgments}
\setheader{Acknowledgments}

Acks.

\begin{flushright}
{\itshape
Carlo \\
Delft, May 2023
}
\end{flushright}

\chapter*{Curriculum Vit\ae}
\addcontentsline{toc}{chapter}{Curriculum Vit\ae}
\setheader{Curriculum Vit\ae}

% Print the full name of the author.
\makeatletter
\authors{\@firstname\ {\titleshape\@lastname}}
\makeatother

\noindent
\begin{longtable}{p{.225\textwidth} p{.70\textwidth}}
    1994/05/02 & Date of birth in Potenza, Italy
\end{longtable}

\chapter*{List of Publications}
\addcontentsline{toc}{chapter}{List of Publications}
\setheader{List of Publications}
\label{publications}

% We use the 'etaremune' environment (the reverse of 'enumerate') to get a numbered list of
% publications in reverse chronological order. If the list of authors is long, it might be useful to
% emphasize your own name with \textbf.
\begin{etaremune}
% {\small
    \item[\faFileTextO~~1.] \textbf{Carlo Delle Donne}. Some paper title and related info.
    \item[2.] \textbf{Carlo Delle Donne}. Some other paper title and related info.
% }
\end{etaremune}

\vspace{0.5cm}
\noindent
\faFileTextO~~Included in this thesis.


\end{document}
