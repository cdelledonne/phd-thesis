\chapter*{Samenvatting}
\addcontentsline{toc}{chapter}{Samenvatting}
\setheader{Samenvatting}

{\selectlanguage{dutch}

Computer netwerken is een van de meest revolutionaire concepten and technologieën van de afgelopen
vijftig jaar. Tegenwoordig is het zo goed als onmogelijk om je een wereld zonder het internet voor
te stellen. En toch, slechts vijf decennia geleden, wist bijna niemand wat het ook maar betekende.
Nu wordt gewerkt aan de eerste kwantum computer netwerken, met daarbij een belofte op een toekomstig
kwantum internet. Kwantum netwerken maken gebruik van de fundamentele beginselen van de
kwantummechanica --- waarbij vooral het concept van \emph{verstrengeling} van belang is --- om een
nieuw paradigma van connectiviteit aan ge bieden, welke communicatie netwerken zal versterken en
nieuwe, spannende toepassingen zal doen opbrengen.

Kwantum netwerken worden al een aantal jaar bestudeerd. Desondanks is de status van kwantum
netwerken tegenwoordig ongeveer vergelijkbaar met dat van het klassieke internet aan het eind van de
jaren 60: veel interessante ideeën, een aantal experimentele demonstraties, en weinig betrouwbare
testbeds. Opschalen naar grotere netwerken met kwantum computers heeft op zijn minst een collectieve
inspanning van natuurkundigen, wiskundigen, elektrotechnici en informatici nodig. Deze disciplines
samen laten komen is een lastige taak, aangezien we nog geen standaard fysieke platformen hebben,
nog hebben we universele kaders en testbeds om onze hypothese mee te testen. Eén van de ontbrekende
schakels tussen complexe fysieke platformen en netwerken en de abstracte omschrijvingen van kwantum
netwerk toepassingen is een kader dat het gat daartussen overbrugt door platform-onafhankelijke
abstracties van de onderliggende natuurkunde aan te bieden aan programmeurs en gebruikers van het
kwantum netwerk.

Het doel van deze scriptie heeft drie hoofdzaken: de benodigdheden van een dergelijk kader van
abstracties voor kwantum netwerken --- welke we een besturingssysteem noemen --- bediscussiëren, een
ontwerp voor zo'n besturingssysteem aandragen, en dit ontwerpen implementeren en valideren op een
fysiek kwantum netwerk. Waar we wel geïnteresseerd zijn in de prestaties van het besturingssysteem,
noemen we ons ontwerp ``best-effort'', and richten we ons vooral op het neerzetten van een
uitgangspunt voor toekomstig onderzoek in dit onderzoeksgebied. Desondanks zijn we op zoek naar een
volledig functionerend product waarvan we hopen dat het de grenzen van kwantum netwerk demonstraties
kan verleggen, en om een beter begrip te krijgen van de uitdagingen in het efficiënt implementeren
van besturingssystemen voor kwantum netwerk nodes.

}
