\chapter*{Summary}
\addcontentsline{toc}{chapter}{Summary}
\setheader{Summary}

Computer networks have been one of the most revolutionary concepts and technologies of the last
fifty years. Currently, it is arguably impossible to imagine a world without the internet. And yet,
just five decades ago, hardly anybody knew what it even meant. Today, the first quantum computer
networks are starting to take shape, along with the promise of a future quantum internet. Quantum
networking exploits fundamental primitives of quantum mechanics --- most importantly
\emph{entanglement} --- to offer a new paradigm of connectivity, which will enhance communication
networks and bring some new exciting applications into the scene.

Quantum networking has been studied for a few years already. Nevertheless, the current state of the
art of quantum networks is somewhat comparable to that of the classical internet at the end of the
1960s: lots of interesting ideas, some experimental demonstrations, and very few reliable testbeds.
Scaling up to larger networks of quantum computers requires joint efforts of physics, mathematics,
electronics and computer science, at the very least. Bringing these disciplines together is a very
bumpy road, given that we do not yet have standard quantum physical platforms to work with, nor
universal frameworks and testbeds to validate our hypotheses against. One of the missing links
between the highly-complex physical platforms and networks and the high-level descriptions of
quantum networking applications is a framework that bridges that gap between these two, providing
platform-independent abstractions of the underlying physics to programmers and users of a quantum
network.

The goal of this thesis is threefold: discuss the requirements for such a framework of abstractions
--- which we refer to as an \acrlong{os} --- for quantum networks, propose a design for such an
\acrlong{os}, and implement and validate this design on a physical quantum network. Whilst we are
interested in measuring the performance of the \acrlong{os}, we consider our design to be
best-effort, and thus we are primarily aiming at establishing a baseline for future research in this
field. Nevertheless, we are after a fully-functional product that we hope can be used to push the
boundaries of quantum networking demonstrations, and to better understand the challenges of
designing and implementing efficient \acrlongpl{os} for quantum network nodes.
