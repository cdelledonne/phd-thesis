\chapter{Introduction}
\label{chp:intro}

\begin{refsection}

\begin{abstract}
Chapter abstract.
\end{abstract}

\newpage

\lettrine{Q}{uite} possibly, the word ``quantum'' is one of the most trending, and perhaps one of
the most abused, of the last decade, at least in the context of science and technology. Attaching it
to the name of tech products makes them sound more advanced. Even just a capital ``Q'' in a brand's
name suggests superiority. When I embarked on my PhD at the end of 2018, I started to comprehend how
challenging it is to develop actual quantum technology. Nevertheless, then next thing I learned is
that Mozilla had already dropped a quantum (?) browser~\cite{firefox_quantum}.

However, quantum physics is not just an otherworldly theory from science fiction books, nor just a
catchy name for 21st-century consumer electronics. Since the formulation of the theory of quantum
mechanics in (YEAR, and CITATION), researchers and enthusiasts have been looking for how to make use
of these physical properties, particularly in the fields of electronics, information processing and
telecommunications. Today, we know of a handful of applications that exploit the axioms of quantum
information theory to achieve something that was though to be very hard, sometimes even impossible.
The most well-known use cases include fast resolution of computational problems --- like integer
factorization (CITE) and database searching (CITE) --- and efficient and secure communication
schemes --- for instance quantum key distribution (CITE) and superdense coding (CITE).

\emph{Quantum networking}, a new paradigm of telecommunications, would enhance --- not replace ---
our current internet technology to provide new functionalities that are impossible to achieve
exclusively with classical communications. Novel applications include security-enhancing
communication schemes such as \acrfull{qkd} (CITE), advanced clock synchronization routines (CITE),
distributed consensus protocols (CITE), distributed sensing (CITE), distributed quantum computation
(CITE), and secure cloud quantum computing (CITE). Even though quantum communications are an
established reality, and their potential applications have garnered attention from industry and
research institutes, the general public is still rather puzzled, on average, when someone tries to
pitch their research on quantum networking. The above list of applications does not necessarily
appeal to the masses, which are still sometimes stuck with the hope that quantum teleportation will
instantly get them to the Bahamas~\cite{xkcd_teleportation}. Nonetheless, the community of quantum
networking researchers is not discouraged by this mismatch in expectations, as it hopes that more
applications will be devised once the technology becomes more widespread and available to more
``consumers''. After all, we were also not aware of all the possible uses of the classical internet
when it was first developed. Yet, today the internet means instantaneous access to low-cost clothes,
scenes of hilarious felines, and --- for some --- tapes of bare bodies engaging in intimate action
on camera.

One of the most frequently asked questions about quantum technology is: \emph{When can we use it?}
If more people had access to quantum networks, they would perhaps come up with more ideas for
quantum networking applications. Thus, what is missing before we can deploy quantum networks
consisting of a useful number of nodes? What are the main limitations we are facing? How can we
overcome them? Not surprisingly, the answers to these questions are complicated. There is a cauldron
of fundamental and theoretical limitations, technological hardware obstacles, and computer science
puzzles that hinder the success of quantum networking. In this thesis, we will focus on some of the
challenges of managing the activity and the resource of a quantum network node, from an operating
system's perspective. We thus aim to answer a tiny fraction of the research questions in the field
of quantum networking, and to lay another brick in the construction of \emph{scalable},
\emph{controllable}, and \emph{configurable} quantum networks.

The remainder of this chapter introduces the reader to basic networking concepts and quantum
networking challenges, lists the research questions we address, and provides an outline of the rest
of the thesis.

\section{Networking and Quantum Networking}

Give a layperson's overview of networking-related computer science and quantum
information/networking, and informally introduce the reader to the challenges of applying the former
to the latter.

Also, briefly list all main challenges that come with quantum information: decoherence, no-cloning.
But also advantages of no-cloning/monogamy of entanglement.

Initially, it was okay to have "slow" classical computers and classical networks. With "slow"
quantum computers and control algorithms, info is lost.

Drawing of the idea I have for the layperson talk.

\section{Research Goals and Methodology}

Succinctly list main research questions and goals. Also, clarify scope of thesis, stating that we
took an implementation- and experiment-driven approach, rather than a more fundamental one.

\section{Thesis Outline}

List and brief overview of chapters that follow.

\printbibliography[heading=subbibintoc,title={References}]

\end{refsection}
