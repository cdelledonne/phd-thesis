\chapter{Introduction}
\label{chp:intro}

\begin{refsection}

% \begin{abstract}
% Chapter abstract.
% \end{abstract}

% \newpage

\lettrine{Q}{uite} possibly, the word ``quantum'' is one of the most trending, and perhaps one of
the most abused, of the last decade, at least in the context of science and technology. Attaching it
to the name of tech products makes them sound more advanced. Even just a capital ``Q'' in a brand's
name suggests superiority. When I embarked on my PhD at the end of 2018, I started to comprehend how
challenging it is to develop actual quantum technology. Nevertheless, then next thing I learned is
that Mozilla had already dropped a quantum (?) browser~\cite{firefox_quantum}.

However, quantum physics is not just an otherworldly theory from science fiction books, nor just a
catchy name for 21st-century consumer electronics. Since the formulation of the theory of quantum
mechanics in (YEAR, and CITATION), researchers and enthusiasts have been looking for how to make use
of these physical properties, particularly in the fields of electronics, information processing and
telecommunications. Today, we know of a handful of applications that exploit the axioms of quantum
information theory to achieve something that was though to be very hard, sometimes even impossible.
Some of the most well-known use cases include fast resolution of computational problems --- like
integer factorization (CITE) and database searching (CITE) --- and efficient and secure
communication schemes --- for instance quantum key distribution (CITE) and superdense coding (CITE).

\emph{Quantum networking}, a new paradigm of telecommunications, seeks to enhance --- not replace
--- our current internet technology to provide new functionalities that are impossible to attain
with purely classical communications. Novel applications include security-enhancing communication
schemes such as \acrfull{qkd} (CITE), advanced clock synchronization routines (CITE), distributed
consensus protocols (CITE), distributed sensing (CITE), distributed quantum computation (CITE), and
secure cloud quantum computing (CITE). Even though quantum communications are an established
reality, and their potential applications have garnered attention from industry and research
institutes, the general public is still rather puzzled, on average, when someone tries to pitch
their research on quantum networking. The above list of applications does not necessarily appeal to
the masses, which are still sometimes stuck with the hope that quantum teleportation will instantly
get them to the Bahamas~\cite{xkcd_teleportation}. Nonetheless, the community of quantum networking
researchers is not discouraged by this mismatch in expectations, as it hopes that more applications
will be devised once the technology becomes more widespread and available to more ``consumers''.
After all, we were also not aware of all the possible uses of the classical internet when it was
first developed. Yet, today the internet means instantaneous access to low-cost clothes, scenes of
hilarious felines, and --- for some --- tapes of bare bodies engaging in intimate action on camera.

One of the most frequently asked questions about quantum technology is: \emph{When can we use it?}
If more people had access to quantum networks, they would perhaps come up with more ideas for useful
quantum networking applications. Thus, what is missing before we can deploy quantum networks
consisting of a useful number of nodes? What are the main limitations we are facing? How can we
overcome them? Not surprisingly, the answers to these questions are complicated. There is a cauldron
of fundamental and theoretical limitations, technological hardware obstacles, and computer science
puzzles that hinder the success of quantum networking. Fortunately, though, there is an increasing
community of passionate researchers trying to study these limitations, build better hardware, and
solve these puzzles. In this thesis, we will address some of the challenges of managing the activity
and the resources of a quantum network node, from an operating system's perspective. We thus aim to
answer a tiny fraction of the research questions in the field of quantum networking, and to lay
another brick in the construction of \emph{scalable}, \emph{controllable}, and \emph{configurable}
quantum networks.

The remainder of this chapter walks the reader through basic networking concepts and quantum
networking challenges, lists the research questions we aim to address, and provides an outline of
the rest of the thesis.

\section{Networking and Quantum Networking}

% - Drawing of the idea I have for the layperson talk
% - Or some other diagram 

Digital communication networks have come a long way from the early days of the \textsc{arpanet} ---
one of the most important precursors of today's internet. Back in 1969, one of the most advanced
networks of computers consisted of just \emph{four} nodes. As the global network grew larger,
networking researchers standardized, in 1981, the \acrfull{ipv4}, which allowed up to \num{4}
billion devices to have their own address on the public internet (CITE RFC791). Soon after, it
became clear that the pool of available \acrshort{ipv4} addresses was going to be depleted sooner
than later --- which happened in 2011~\cite{icann_2011}. In 2023, there are \num{3.6} devices
connected to the internet per capita, as estimated by \textcite{cisco_2020} in 2020.

Scaling up from a four-node experiment to the massive networks of the 21st century was made possible
by advancements in various fields, including networking hardware, traffic engineering, and network
programmability. You would most likely not be able to download a digital copy of this thesis in a
fraction of a second if it were not for \unit{\tera\bit\per\second} network
switches~\cite{broadcom_tomahawk, juniper_qfx5220}, a diverse spectrum of routing protocols (CITE),
and \acrlong{sdn} (CITE). Nevertheless, classical networking was already appealing in its infant
stages, for the simple reason that even a small network of nodes can accomplish tasks that would not
be attainable without it --- in the case of the \textsc{arpanet}, sending simple pieces of text over
large distances almost instantaneously. The applications of quantum networking are not dissimilar to
their classical networking counterpart, in that some of them can be useful and effective on small
quantum networks already, whilst other use cases require more powerful nodes and more complex
networks~\cite{wehner_stages_2018}.

Even though hardware and software advancements were essential to the betterment of the internet,
there are a few ingredients that have been there since the dawn of classical networking and that
have made these technological leaps even possible. Nailing down some basic design aspects right from
the start is often the key for a truly scalable system. For instance, it was immediately clear that
data was to be grouped and transmitted according to the \emph{packet switching} model, to maximize
network utilization and to allow for dynamic routing decisions. As another example, networking
protocols have been organized in \emph{abstraction layers} since the very beginning of computer
networks, so as to encapsulate network functionalities in services of increasing user-friendliness.
When designing quantum networks and networking services, one should draw inspiration from classical
networking principles and avoid monolithic designs and software architectures that would render the
system hard to scale.

Then, how similar are quantum networks and classical networks? Which are the challenges that they
have in common, and which ones are exclusive to quantum networking? How much can we capitalize on
the vast body of classical networking literature? In principle, quantum networks are just networks
with a special physical layer, which, albeit more technologically complex, could be abstracted away
and encapsulated into an ad-hoc networking protocol. This simplification would, however, disregard
some fundamental limitations that are inherent to quantum information, as well as the imperfect
nature of near-term quantum hardware. Standard networking routines like signal amplification,
classical error correction, and data retransmission would not work in quantum networks --- one
fundamental theorem of quantum mechanics states that \emph{an arbitrary quantum state cannot be
cloned}~\cite{wootters_nocloning_1982, dieks_communication_1982}. Moreover, quantum states are
subject to \emph{decoherence}, a physical process whereby the quality of the stored information
degrades due to time and external interferences, Thus, not only do computation and networking delays
affect throughput and latency, but they also exact a toll on the quality of the service, effectively
determining whether a certain application produced meaningful results or not. Whilst decoherence can
be worked around with more sophisticated quantum hardware and control algorithms, the no-cloning
theorem is a hard limit on what one can do with quantum information --- which, incidentally, has
also positive implications and is the foundation of many quantum networking applications.

Without being too speculative, we can argue that we cannot just encapsulate the requirements of
quantum networking into a specialized physical layer. Reusing classical networking techniques and
protocols as they are would fall short of the aforementioned challenges posed by quantum mechanics.
Nonetheless, many of the questions that drove the classical networking research community can be of
inspiration for analyzing requirements and limitations of quantum networks. Examples of such
questions are: \emph{Can we organize quantum information into packets?} \emph{Do we need to resort
to path reservation for quantum communication?} \emph{In which situations can we tolerate local and
network latency?} \emph{How do we organize and layer quantum networking protocols and services?}
\emph{What metrics do we look at when designing routing protocols?} \emph{Can we improve performance
and quality of service with the employment of a software-defined control plane?}

\section{Research Goals and Methodology}

% Succinctly list main research questions and goals. Also, clarify scope of thesis, stating that we
% took an implementation- and experiment-driven approach, rather than a more fundamental one.

Perhaps disappointingly, but unsurprisingly too, this thesis will not try to answer all the research
questions from the previous section.

\section{Thesis Outline}

List and brief overview of chapters that follow.

\printbibliography[heading=subbibintoc,title={References}]

\end{refsection}
