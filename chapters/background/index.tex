\chapter{Background: Abstraction Layers for Quantum Networking}
\label{chp:background}

\begin{refsection}

\begin{abstract}
Chapter abstract.
\end{abstract}

\newpage

% \lettrine{U}{...}

\noindent
Detailed analysis of background and challenges, to answer the question: \emph{How useful and
relevant is this thesis?} This would be an extended version of the sections ``Background'' and
``Design Considerations'' from the QNodeOS paper.

\section{Quantum Networking: Theory and Limitations}

Quantum networking basics, state of the art, ad-hoc experiments, and (lack of) sophisticated control
systems above the physical layer.

\section{Abstraction Layers for Classical Computer Networks}

Recap of what exists for classical networking, including communication protocols, network OSes,
SDNs, and what we can learn from these for quantum networking.

\section{Challenges and Design Considerations}

Limitations of quantum theory and hardware that need addressing when designing abstractions, and
generic requirements for such abstractions.

\printbibliography[heading=subbibintoc,title={References}]

\end{refsection}
