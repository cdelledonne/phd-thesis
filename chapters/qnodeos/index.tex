\chapter{Quantum Networking With an Elementary Operating System}
\label{chp:qnodeos}

\begin{refsection}

\begin{abstract}
Chapter abstract.
\end{abstract}

% \blfootnote{
%     This chapter is partly based on \emph{C. Delle Donne, Some Paper Title, Venue
%     and Year)}~\cite{somepaper}.
% }

\newpage

\noindent
This would be mostly based on the QNodeOS paper.

\section{Goals of the Operating System}

Recap of the generic requirements from \cref{chp:background}, and list of key components that such
an OS should have.

\section{Design and Implementation}

``Design'' and ``Implementation'' sections from the QNodeOS paper.

\section{Target Applications and Evaluation Goals}

First part of the ``Evaluation'' section from the QNodeOS paper, with a more extensive explanation
of the target applications (including multitasking), and an explanation of how each of the apps,
whilst conceptually simple, test most functionalities of the OS.

\section{Results and Takeaways}

Second part of the ``Evaluation'' section from the QNodeOS paper.

\section{Conclusion}

...

\printbibliography[heading=subbibintoc,title={References}]

\end{refsection}
