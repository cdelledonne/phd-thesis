\chapter{Soundtrack}
\label{app:soundtrack}

As a musician (?) and music enthusiast, I always try to find a soundtrack that fits whatever I am
doing or thinking in a certain situation. Very much like in a movie, soundtracks can be mash-ups of
existing songs and organic contextual sounds, or sometimes more though-out compositions. Running
applications on an experimental quantum network using an experimental operating system can be very
frustrating at times: the hardware can break, the software can be buggy, the entities running the
simulated reality~\cite{bostrom_2003_simulation} can be pesky. Therefore, those scenes are typically
backed by impromptu rambling noises --- thudding, stomping, finger-snapping, thigh-smacking. On a
good day, the satisfaction of a successful experiment turns the percussive soundtracks into a more
melodic, climax-resolving tune --- think of something like \citetitle{gorillaz_feelgoodinc} by
\textcite{gorillaz_feelgoodinc}, or perhaps \citetitle{tool_lateralus} by \textcite{tool_lateralus}
if I am feeling more (bare) metal that day.

However, I felt that the experimental acts of this thesis needed their ad-hoc piece, something that
more adequately represented my feelings towards what in retrospect looks like a fantastic journey
through physics and computers. For that matter, I am offering an alternative interpretation of the
results obtained... post-processed through this Python script... and embellished with some
human-made percussion sounds. The outcome of this final experiment is available at...

\begin{xstretch}
\printbibliography[heading=subbibintoc,title={References},notcategory=noprint]
\end{xstretch}
