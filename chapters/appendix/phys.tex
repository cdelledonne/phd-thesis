\chapter{Implementation of the Quantum Physical Layer}
\label{app:phys}

We provide additional details concerning the implementation of the quantum physical layer used for
our experiments.

\paragraph{Single qubit gates}

At the physical layer, we implement real-time rotations around the X and Y axes of the qubit Bloch
sphere, using a resolution of $\pi/16=$\ang{11.25}. That is, the upper layer can request any
rotation that is a multiple of $\pi/16$ around either the X or Y axis. The different rotations are
performed using Hermite-shaped pulses (as described in Ref.~\cite{pompili_2021_multinode}) of
calibrated amplitude. The choice of X(Y) rotation axis is implemented using the I(Q) channel of the
microwave vector source.

While supported on \acrshort{qnodeos}, our physical layer currently does not implement Z-axis
rotations. Such rotations around the Z axis could be implemented by virtual rotations of the Bloch
sphere: a $\pi$ pulse around the Z axis is equivalent to multiplying future I and Q voltages by
$-1$. By keeping track of the accumulated Z rotations, and by adjusting I and Q mixing accordingly,
one can perform effective Z rotations with very high resolution and virtually no infidelity. The
\acrshortpl{awg} currently in use have the required capabilities, and the implementation of said Z
gates is planned for the near future.

\paragraph{Clock sharing and \acrshort{awg} triggering over longer distances}

One of the technical challenges of realizing a large scale quantum network is synchronizing
equipment at the physical layer across nodes. The synchronization is required to generate
entanglement --- the photons from the two nodes need to arrive at the same time at the heralding
station (compared to their duration, \qty{12}{\ns} for \acrshort{nv} centers in bulk diamond
samples); failing to do so would reduce (or even remove) their indistinguishability, which is
required to establish long-distance entanglement~\cite{pompili_2021_multinode}. Our two nodes are
located in a single laboratory, on the same optical table, approximately \qty{2}{\m} apart. This
allows for some simplifications, for the purpose of demonstrating entanglement delivery using a
network stack, which would not be possible over longer distances. Specifically:

\begin{enumerate}
    \item We use a single laser --- the client's --- to excite both nodes, as in
          Ref.~\cite{pompili_2021_multinode}. Over longer distances, one would need to phase-lock
          the excitation lasers at the two nodes to ensure phase-stability of the entangled states.
    \item The Device Controllers (ADwin Pro II microcontroller~\cite{adwin}) are triggered every
          \qty{1}{\us} by the same signal generator, advancing the state machine algorithm that
          implements the physical layer. This ensures that the two microcontrollers have a common
          shared clock. Over longer distances, one could use existing protocols (and
          commercially-available hardware) to obtain a shared clock~\cite{whiterabbit}, and use that
          to trigger the microcontrollers.
    \item The two \acrshortpl{awg} need to be triggered to play entanglement attempts. In our
          implementation, one device controller --- the server's --- triggers both \acrshortpl{awg}.
          This ensures that the triggering delay between the two \acrshortpl{awg} is constant, and
          we can therefore calibrate it out. Triggering the \acrshortpl{awg} with two independent
          microcontrollers would result in jitter (realistically on the order of nanoseconds). Over
          larger distances, one could derive --- from the shared clock --- a periodic trigger signal
          that is gated by the microcontroller at each node. In this way the microcontroller can
          decide whether the \acrshort{awg} will be triggered on the next cycle, but the accuracy of
          the trigger's timing will be derived from the shared clock between the nodes, rather than
          from the microprocessor.
    \item The phase stabilization scheme we use, developed in Ref.~\cite{pompili_2021_multinode}, is
          designed to work at a single optical frequency (in our case, the \qty{637}{\nm} emission
          frequency of the \acrshort{nv} center). Over longer distances, conversion of the
          \acrshort{nv} center photons to the telecom band will be necessary to overcome photon
          loss. The phase stabilization scheme will therefore need to be adapted to new optical
          frequencies used.
\end{enumerate}

For reference, our client (server) is based on node Charlie (Bob) of the multi-node quantum network
presented in Ref.~\cite{pompili_2021_multinode}.

\paragraph{NV center resonance control}

The two quantum network nodes use different techniques to control the resonance of their
\acrshort{nv} centers (see Ref.~\cite{pompili_2021_multinode} for implementations details). The
server uses an off-resonant charge randomization strategy: when its \acrshort{nv} center is not on
resonance (it does not pass the charge and resonance check), it can apply an off-resonant (green,
\qty{515}{\nm}) laser pulse to shuffle the charge environment and probabilistically recover the
correct charge and resonance state. The server cannot get \emph{stuck} in a non-resonance state: in
a few tens of failed \acrshortpl{crcheck} and green laser pulses (overall less than \qty{1}{\ms})
the \acrshort{nv} center will be in resonance again.

The client, which needs to be tuned in resonance with the other node, uses a resonant strategy. When
in the wrong charge state (zero counts during \acrshort{crcheck}), it applies a resonant laser pulse
(yellow, \qty{575}{\nm}, \acrshort{nv}${}^0$ zero-phonon line) to go back to \acrshort{nv}${}^-$. To
bring \acrshort{nv}${}^-$ in resonance with the necessary lasers, it adjusts a biasing voltage
applied to the diamond sample, which shifts the resonance frequencies. This process is mostly
automated. However, occasional human intervention is still required when the resonance frequencies
of the \acrshort{nv} center shift too far --- for example due to a charge in the vicinity of the
\acrshort{nv} center changing position in the lattice --- for the automatic mechanism to find its
way back. Periods of inactivity in entanglement generation are due to the jumps in the client's
\acrshort{nv} optical transitions, which then require manual optimization of the laser frequencies
and/or the diamond biasing voltage --- depending on the magnitude of the frequency shift, it
requires tens of seconds to a few minutes to recover the optimal resonance condition.

\begin{xstretch}
\printbibliography[heading=subbibintoc,title={References},notcategory=noprint]
\end{xstretch}
