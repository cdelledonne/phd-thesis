\chapter{\acrshort{qdevice} Interface} \label{app:qdevice}

The implementation of a \acrshort{qdevice} depends on a number of factors. Most importantly, the
physical signals that are fed to the quantum processing and networking device, and those that are
output from the device, are specific to the nature of the device itself. Different qubit
realizations require different digital and analog control. For instance, manipulating the state of a
spin-based qubit (e.g., in a nitrogen-vacancy center processor) and that of an ultracold atom qubit
(e.g., in a trapped ion processor) are two physical processes that vastly differ in a number of
complicated ways.

For \acrshort{qnodeos} to be portable to a diverse set of quantum physical platforms, there needs to
be a common \emph{\acrshort{qdevice} interface} that \acrshort{qnodeos} can rely on, and that each
\acrshort{qdevice} instance can implement as it is most convenient for the underlying quantum
device. This interface need be quite general, to be able to express all quantum operations that
different quantum devices might be capable of performing, and rather abstract, so that two different
implementations of a well-defined qubit manipulation operation can be expressed with the same
instruction on \acrshort{qnodeos}. Nevertheless, an interface that is too general could result in a
high implementation complexity on the \acrshort{qdevice}, as it might have to transform high-level
instructions in a series of native operations on the fly. Other than complexity of implementation, a
very high-level set of \acrshort{qdevice} instructions might compromise the compiler's ability to
optimize an application for a certain physical platform, as reported by
\textcite{murali_2019_fullstack}.

Defining a set of instructions to express abstract quantum operations as close as possible to what
different quantum physical platforms can natively perform is, to some extent, an open problem. While
this is outside the scope of this work, we have made an effort to specify an interface which is a
good compromise between generality and expressiveness. The \acrshort{qdevice} interface is
essentially a set of instructions that \acrshort{qnodeos} expects a \acrshort{qdevice} to implement.
To be precise, a \acrshort{qdevice} might implement a subset of the interface, according to what
native physical operations it can perform. The Host compiler must then have knowledge about the set
of instructions implemented by the underlying \acrshort{qdevice}, so that it can decompose
instructions that are not natively supported.

Even though this interface does not impose any formal timing constraints, it is important to note
that a \acrshort{qdevice} implementation that tries to guarantee more or less deterministic
instruction processing latencies can prove more beneficial to the real-time requirements of
\acrshort{qnodeos}. Particularly, it would be advisable to time-bound the processing time of
operations whose duration is by nature probabilistic --- most notably, those involving entanglement
generation. Creating an entangled pair may involve a varying number of attempts. Sometimes, if the
remote node becomes unresponsive for a period of time, the number of necessary attempts can increase
by a large amount. Capping the number of attempts could, for instance, provide a more deterministic
maximum processing latency for entanglement instructions, which in turn might help
\acrshort{qnodeos} react more timely to temporary failures or downtime periods of remote nodes. Not
to mention that unbounded entanglement attempts affect the state of other qubits in memory, because
of both passive decoherence and cross-qubit noise.

\begin{table}[t]
    \centering
    \begin{tabularx}{0.75\linewidth}{>{\ttfamily}l l}
        \toprule
        \normalfont{Instruction} & Description                                  \\
        \midrule
        INI                      & Initialize a qubit to default state          \\
        SQG                      & Perform a single-qubit gate                  \\
        TQG                      & Perform a two-qubit gate                     \\
        AQG                      & Perform a gate on all qubits                 \\
        MSR                      & Measure a qubit in a specified basis         \\
        ENT                      & Attempt entanglement generation              \\
        ENM                      & Attempt entanglement and measure qubit       \\
        MOV                      & Move qubit state to another qubit            \\
        SWP                      & Swap the state of two qubits                 \\
        ESW                      & Swap qubits belonging to two entangled pairs \\
        PMG                      & Set pre-measurement gates                    \\
        \bottomrule
    \end{tabularx}
    \caption{
        Summary of \acrshort{qdevice} instructions defined in the \acrshort{qdevice} interface. A
        specific \acrshort{qdevice} might implement a subset of these, depending on the underlying
        quantum physical device and on other design constraints.
    }
    \label{tab:qdevice-instructions}
\end{table}

\Cref{tab:qdevice-instructions} lists the complete set of instructions defined in the
\acrshort{qdevice} interface. Instructions can have operands, whose range of valid values depends on
the underlying \acrshort{qdevice}. For instance, an operand that specifies which qubit to apply an
operation to can only have as many valid values as there are physical qubits in memory. Details for
each instruction and its operands are given below.

\paragraph{Qubit initialization (\texttt{INI})}

The \texttt{INI} instruction brings a qubit to the $\ket{0}$ state. On some physical platforms,
single-qubit initialization is not possible, thus this instruction initializes all qubits to the
$\ket{0}$ state.

\smallskip\noindent
\begin{tabularx}{\linewidth}{>{\ttfamily}l X}
    \toprule
    \normalfont{Operand} & Description                                                                                                         \\
    \midrule
    qubit                & Physical address of the qubit to initialize, ignored on platforms where single-qubit initialization is not possible \\
    \bottomrule
\end{tabularx}
\medskip

\paragraph{Single-qubit gate (\texttt{SQG})}

The \texttt{SQG} instruction manipulates the state of one qubit. The gate is expressed as a rotation
in the Bloch sphere.

\smallskip\noindent
\begin{tabularx}{\linewidth}{>{\ttfamily}l X}
    \toprule
    \normalfont{Operand} & Description                                                                  \\
    \midrule
    qubit                & Physical address of the qubit to manipulate                                  \\
    axis                 & Rotation axis, can be X, Y, Z or H (support is \acrshort{qdevice}-dependent) \\
    angle                & Rotation angle (granularity and range are \acrshort{qdevice}-dependent)      \\
    \bottomrule
\end{tabularx}
\medskip

\paragraph{Two-qubit gate (\texttt{TQG})}

The \texttt{TQG} instruction manipulates the state of two qubits. The gate is expressed as a
controlled rotation, with one qubit being the control and the other one being the target.

\smallskip\noindent
\begin{tabularx}{\linewidth}{>{\ttfamily}l X}
    \toprule
    \normalfont{Operand} & Description                                                                  \\
    \midrule
    qub\_c               & Physical address of the control qubit                                        \\
    qub\_t               & Physical address of the target qubit                                         \\
    axis                 & Rotation axis, can be X, Y, Z or H (support is \acrshort{qdevice}-dependent) \\
    angle                & Rotation angle (granularity and range are \acrshort{qdevice}-dependent)      \\
    \bottomrule
\end{tabularx}
\medskip

\paragraph{All-qubit gate (\texttt{AQG})}

The \texttt{AQG} instruction manipulates the state of all available qubits. The gate is expressed as
a rotation in the Bloch sphere.

\smallskip\noindent
\begin{tabularx}{\linewidth}{>{\ttfamily}l X}
    \toprule
    \normalfont{Operand} & Description                                                                  \\
    \midrule
    axis                 & Rotation axis, can be X, Y, Z or H (support is \acrshort{qdevice}-dependent) \\
    angle                & Rotation angle (granularity and range are \acrshort{qdevice}-dependent)      \\
    \bottomrule
\end{tabularx}
\medskip

\paragraph{Qubit measurement (\texttt{MSR})}

The \texttt{MSR} instruction measures the state of one qubit in a specified basis. A qubit
measurement is destructive --- that is --- the qubit has to be reinitialized before it can be used
again.

\smallskip\noindent
\begin{tabularx}{\linewidth}{>{\ttfamily}l X}
    \toprule
    \normalfont{Operand} & Description                                                                    \\
    \midrule
    qubit                & Physical address of the qubit to measure                                       \\
    basis                & Measurement basis, can be X, Y, Z, H (support is \acrshort{qdevice}-dependent) \\
    \bottomrule
\end{tabularx}
\medskip

\paragraph{Entanglement generation (\texttt{ENT})}

The \texttt{ENT} instruction performs a series of entanglement generation attempts, until one
succeeds, or until a maximum number of attempts is reached (the behavior is
\acrshort{qdevice}-dependent).

\smallskip\noindent
\begin{tabularx}{\linewidth}{>{\ttfamily}l X}
    \toprule
    \normalfont{Operand} & Description                                                                                               \\
    \midrule
    nghbr                & Neighboring node to attempt entanglement with, if the local \acrshort{qdevice} has multiple quantum links \\
    fid                  & Target entanglement fidelity (granularity and range are \acrshort{qdevice}-dependent)                     \\
    \bottomrule
\end{tabularx}
\medskip

\paragraph{Entanglement generation with qubit measurement (\texttt{ENM})}

The \texttt{ENM} instruction performs a series of entanglement generation attempts followed by an
immediate measurement of the local qubit, until one succeeds, or until a maximum number of attempts
is reached (the behavior is \acrshort{qdevice}-dependent).

\smallskip\noindent
\begin{tabularx}{\linewidth}{>{\ttfamily}l X}
    \toprule
    \normalfont{Operand} & Description                                                                                               \\
    \midrule
    nghbr                & Neighboring node to attempt entanglement with, if the local \acrshort{qdevice} has multiple quantum links \\
    fid                  & Target entanglement fidelity (granularity and range are \acrshort{qdevice}-dependent)                     \\
    basis                & Measurement basis, can be X, Y, Z, H (support is \acrshort{qdevice}-dependent)                            \\
    \bottomrule
\end{tabularx}
\medskip

\paragraph{Qubit move (\texttt{MOV})}

The \texttt{MOV} instruction moves the state of one qubit to another qubit. A qubit move renders the
state of the source qubit undefined, and the qubit has to be reinitialized before it can be used
again.

\smallskip\noindent
\begin{tabularx}{\linewidth}{>{\ttfamily}l X}
    \toprule
    \normalfont{Operand} & Description                               \\
    \midrule
    qub\_s               & Physical address of the source qubit      \\
    qub\_d               & Physical address of the destination qubit \\
    \bottomrule
\end{tabularx}
\medskip

\paragraph{Qubit swap (\texttt{SWP})}

The \texttt{SWP} instruction swaps the state of two qubits.

\smallskip\noindent
\begin{tabularx}{\linewidth}{>{\ttfamily}l X}
    \toprule
    \normalfont{Operand} & Description                          \\
    \midrule
    qub\_1               & Physical address of the first qubit  \\
    qub\_2               & Physical address of the second qubit \\
    \bottomrule
\end{tabularx}
\medskip

\paragraph{Entanglement swap (\texttt{ESW})}

The \texttt{ESW} instruction results in two qubits belonging to two entangled pairs to have their
roles swapped.

\smallskip\noindent
\begin{tabularx}{\linewidth}{>{\ttfamily}l X}
    \toprule
    \normalfont{Operand} & Description                          \\
    \midrule
    qub\_1               & Physical address of the first qubit  \\
    qub\_2               & Physical address of the second qubit \\
    \bottomrule
\end{tabularx}
\medskip

\paragraph{Pre-measurement gates setting (\texttt{PMG})}

The \texttt{PMG} instruction allows for a set of (up to) 3 rotations to be performed before a qubit
measurement (\texttt{MSR} or \texttt{ENM}). If the axis of the second rotation is orthogonal to the
axis of the first and the third rotation, these gates can be used to perform a qubit measurement in
an arbitrary basis, given that most likely a \acrshort{qdevice} can natively measure in a limited
set of bases.

\smallskip\noindent
\begin{tabularx}{\linewidth}{>{\ttfamily}l X}
    \toprule
    \normalfont{Operand} & Description                                                                                                                                             \\
    \midrule
    axes                 & Combination of orthogonal axes to use for the three successive rotations, can be X--Y--X, Y--Z--Y and Z--X--Z (support is \acrshort{qdevice}-dependent) \\
    ang\_1               & Rotation angle of the first gate, relative to the first axis in \texttt{axes} (granularity and range are \acrshort{qdevice}-dependent)                  \\
    ang\_2               & Rotation angle of the second gate, relative to the second axis in \texttt{axes} (granularity and range are \acrshort{qdevice}-dependent)                \\
    ang\_3               & Rotation angle of the third gate, relative to the third axis in \texttt{axes} (granularity
    and range are \acrshort{qdevice}-dependent)                                                                                                                                    \\
    \bottomrule
\end{tabularx}

\printbibliography[heading=subbibintoc,title={References}]
