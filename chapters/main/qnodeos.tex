\chapter{Quantum Networking With an Elementary Operating System}
\label{chp:qnodeos}

\begin{abstract}
An \acrfull{os} for quantum network nodes should provide more than just networking functionalities.
Ultimately, it should enable quantum networking applications to be written in high-level,
platform independent software, and should be able to manage the resources of the underlying
device when deployed in a multi-node and multi-user quantum network. This chapter briefly
discusses our implementation of \acrshort{qnodeos}, an \acrshort{os} for quantum network nodes,
which includes a quantum network stack for entanglement generation, as well as resource
management and scheduling features that allow the concurrent execution of multiple applications.
We also validate our implementation on state-of-the-art quantum network hardware based on
\acrlong{nv} centers in diamond, and discuss general design considerations for any such quantum
network \acrshort{os}. Our work sets a baseline for running applications on quantum networks
of the future, and serves as a hands-on study to understand and define computer-science
challenges in building quantum network \acrshortpl{os}.
\end{abstract}

\noindent
\note{To be precise, this chapter is extracted from sections 5 (Implementation), 6 (Evaluation), 7
(Related Work) and 8 (Discussion) from the \acrshort{qnodeos} paper. There aren't any major
additions.}

\blfootnote{
    This chapter is based on the preprint \fullcite{delledonne_2023_qnodeos_noprint}. \note{Add
    proper link to arXiv when submitted}
}

\newpage

\lettrine{T}{he} preliminary experiments conducted in \cref{chp:netstack} showcased elementary
quantum networking functionalities through a platform-independent control system --- mainly, the
quantum networking stack embedded in \acrshort{qnodeos}. Nevertheless, entanglement generation is
just one of the blocks constituting fully-fledged quantum communications applications, which also
include local quantum processing and classical communication and processing, as shown in
\cref{fig:app-struct}. With \acrshort{qnodeos}, we aim to take the state of the art of quantum
networking experiments one step further, and demonstrate the execution of complete applications,
some of which comprise quantum and classical processing and communication. We also include one case
study that serves as a proof-of-concept demonstration of the usefulness of a multitasking-ready
\acrshort{os}, to be expanded on when investigating multi-user quantum networks more in depth. In
this chapter we report on and discuss the results of these experiments. Prior to that, we also give
a brief overview of the implementation of \acrshort{qnodeos} and the underlying \acrshort{qdevice}.
Refer to \cref{app:qnodeos} for additional details on the implementation of the components of
\acrshort{qnodeos} and their interfaces, and to \cref{app:qdevice} for the specification of the
interface to the \acrshort{qdevice}.

\section{Implementation}
\label{sec:qnodeos:implementation}

\Cref{fig:node-deployment} outlines software and hardware implementation of \acrshort{qnodeos} and
the whole node system. \acrshort{qnodeos} is implemented in C++ on top of FreeRTOS, a tiny operating
system for microcontrollers~\cite{freertos}. The stack runs on a dedicated MicroZed~\cite{microzed}
--- an off-the-shelf platform based on the Zynq-7000 SoC, which hosts two ARM Cortex-A9 processing
cores, of which only one is used, clocked at \qty{667}{\MHz}. \acrshort{qnodeos} connects to peer
\acrshort{qnodeos} systems via \acrshort{tcp} over a Gigabit Ethernet interface. For the
\acrshort{qdevice}, we replicated the setup used for \cref{chp:netstack}, which mainly consists of:
%
\begin{enumerate*}[label=(\arabic*)]
    \item an ADwin-Pro II~\cite{adwin} acting as the main orchestrator of the setup;
    \item a series of subordinate devices responsible for qubit control, including laser pulse
          generators and optical readout circuits;
    \item the quantum physical device, based on \acrshort{nv} centers, counting one single
          (communication) qubit for each node.
\end{enumerate*}
The \acrshort{qdevice} is where the time-critical qubit control lies. \acrshort{qnodeos} interfaces
with the \acrshort{qdevice}'s ADwin-Pro II through a \qty{12.5}{\MHz} \acrshort{spi} interface, used
to exchange 4-byte control messages at a rate of \qty{50}{kHz}. Finally, the host layer is a Python
runtime executing on a general-purpose 4-core desktop machine running Linux. The host machine
connects to \acrshort{qnodeos} via \acrshort{tcp} over the same Gigabit Ethernet interface that
\acrshort{qnodeos} uses to connect to its peers (average ping \acrshort{rtt} of \qty{0.1}{\ms}), and
sends application registration requests and quantum code blocks over this interface (\num{10} to
\num{1000} bytes, depending on the length of the block).

We implemented \acrshort{qnodeos} on top of FreeRTOS to avoid re-implementing standard \acrshort{os}
primitives like threads and network communication. FreeRTOS provides basic \acrshort{os}
abstractions like tasks, inter-task message passing, and the \acrshort{tcpip} stack. The FreeRTOS
kernel --- like any other standard \acrshort{os} --- cannot however directly manage the quantum
resources (qubits, entanglement requests and entangled pairs), and hence its task scheduler cannot
take decisions based on such resources. \acrshort{qnodeos} adds these capabilities and takes care of
the scheduling of quantum code blocks based on the status of the quantum resources.

\begin{figure}[t]
    \centering
    \includegraphics[width=0.6\linewidth]{figures/node-deployment.pdf}
    \caption{
        Node deployment overview. Our quantum network node consists of a desktop machine for the
        host runtime, a Zynq-7000 SoC for \acrshort{qnodeos}, and a series of digital and analog
        controllers for the \acrshort{qdevice}.
    }
    \label{fig:node-deployment}
\end{figure}

\section{Target Applications and Evaluation Goals}

First part of the ``Evaluation'' section from the \acrshort{qnodeos} paper, with a more extensive
explanation of the target applications (including multitasking), and an explanation of how each of
the apps, whilst conceptually simple, tests most functionalities of the \acrshort{os}.

\section{Results and Takeaways}

Second part of the ``Evaluation'' section from the \acrshort{qnodeos} paper.

\section{Related Work}
\label{sec:qnodeos:relwork}

Relative to quantum networking, a substantial amount of software and systems work happens in the
field of quantum computing. For example, operating systems for quantum computers (without networking
functionality) are under active development in research and industry~\cite{kong_2021_origin,
deltaflow_os}. Furthermore, extensive work exists on developing quantum computing
architectures~\cite{fu_2017_microarch, murali_2019_fullstack, bourassa_2021_blueprint}. In this
work, we have addressed new problems that arise specifically from the inclusion of quantum
networking which has not been considered at all in the aforementioned \acrshortpl{os} and
publications.

Nevertheless, systems research in quantum networking has been growing as a field as well. In
particular, over the past several years there have been multiple proposals for quantum network
protocol stacks~\cite{van_meter_2013_repeaters, pirker_2019_quantum, dahlberg_2019_egp,
illiano_2022_quantum} and quantum network architectures~\cite{matsuo_2019_bootstrapping,
aguado_2020_enabling, li_2022_connectionless, diadamo_2022_packet, pouryousef_2022_overlay,
gu_2023_fendi, mandil_2023_packet}. One of the proposed network stacks has even been demonstrated
experimentally on a state-of-the-art two-node network in a lab~\cite{pompili_2022_experimental} ---
as detailed in \cref{chp:netstack} --- while the rest has only been validated in simulation.
However, these works heavily focus on the network protocol aspects and whilst some of them
acknowledge that the stacks will exist as a component in a bigger system, they do not tackle any of
the related issues, such as resource management or task scheduling.

Quantum network applications themselves have also been demonstrated on small networks in
laboratories~\cite{barz_2012_demonstration}. However, such demonstrations have always been \emph{ad
hoc}, and scripted through low-level experimental controls as their purpose was to demonstrate
hardware technology milestones rather than develop general systems for multiple users.

\section{Discussion}

\noindent
\note{Copy over from \acrshort{qnodeos} paper after experiments have been performed}

\printbibliography[heading=subbibintoc,title={References},notcategory=noprint]
