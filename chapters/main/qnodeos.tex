\chapter{Quantum Networking With an Elementary Operating System}
\label{chp:qnodeos}

\begin{abstract}
Chapter abstract.
\end{abstract}

\blfootnote{
    This chapter is based on the preprint \fullcite{delledonne_2023_qnodeos_noprint}. \note{Add
    proper link to arXiv when submitted}
}

\newpage

% \lettrine{T}{...}

\noindent
This would be mostly based on the \acrshort{qnodeos} paper.

\section{Goals of the Operating System}

Recap of the generic requirements from \cref{chp:background}, and list of key components that such
an OS should have.

\section{Design and Implementation}

``Design'' and ``Implementation'' sections from the \acrshort{qnodeos} paper.

\section{Target Applications and Evaluation Goals}

First part of the ``Evaluation'' section from the \acrshort{qnodeos} paper, with a more extensive
explanation of the target applications (including multitasking), and an explanation of how each of
the apps, whilst conceptually simple, test most functionalities of the OS.

\section{Results and Takeaways}

Second part of the ``Evaluation'' section from the \acrshort{qnodeos} paper.

\section{Discussion}

...

\printbibliography[heading=subbibintoc,title={References},notcategory=noprint]
