\chapter{Quantum Networking With an Elementary Operating System}
\label{chp:qnodeos}

\begin{abstract}
An \acrfull{os} for quantum network nodes should provide more than just networking functionalities.
Ultimately, it should enable quantum networking applications to be written in high-level,
platform-independent software, and should be able to manage the resources of the underlying
device when deployed in a multi-node and multi-user quantum network. This chapter discusses our
implementation of \acrshort{qnodeos}, an \acrshort{os} for quantum network nodes, which includes a
quantum network stack for entanglement generation, as well as resource management and scheduling
features that allow the concurrent execution of multiple applications. We also design and propose a
set of benchmarks which will be used to quantify --- in upcoming work --- the performance of the
\acrshort{os} on state-of-the-art quantum network hardware based on \acrlong{nv} centers in diamond.
\end{abstract}

\blfootnote{
This chapter is extracted from the article in preparation:
\fullcite{delledonne_2023_qnodeos_noprint}.

Contributions: as indicated in \cref{chp:background}.
}

\newpage

\lettrine{T}{he} preliminary experiments conducted in \cref{chp:netstack} showcased elementary
quantum networking functionalities through a platform-independent control system --- mainly, the
quantum networking stack embedded in \acrshort{qnodeos}. Nevertheless, entanglement generation is
just one of the blocks constituting fully-fledged quantum communications applications, which also
include local quantum processing and classical communication and processing, as shown in
\cref{fig:app-struct}. With \acrshort{qnodeos}, we aim to take the state of the art of quantum
networking experiments one step further, and demonstrate the execution of complete applications,
some of which comprise quantum and classical processing and communication. We also include one case
study that serves as a proof-of-concept demonstration of the usefulness of a multitasking-ready
\acrshort{os}, to be expanded on when investigating multi-user quantum networks more in depth. In
this chapter we describe the proposed test applications, which will serve as case studies for the
upcoming evaluation of \acrshort{qnodeos} on \acrshort{nv} center devices. Prior to that, we also
give an overview of the implementation of \acrshort{qnodeos} and the underlying \acrshort{qdevice}.
Refer to \cref{app:qnodeos} for additional details on the implementation of the components of
\acrshort{qnodeos} and their interfaces, and to \cref{app:qdevice} for the specification of the
interface to the \acrshort{qdevice}.

\section{Implementation}
\label{sec:qnodeos:implementation}

\begin{figure}[b]
    \centering
    \includegraphics[width=0.6\linewidth]{figures/node-deployment.pdf}
    \caption{
        Node deployment overview. Our quantum network node consists of a desktop machine for the
        host runtime, a Zynq-7000 SoC for \acrshort{qnodeos}, and a series of digital and analog
        controllers for the \acrshort{qdevice}.
    }
    \label{fig:node-deployment}
\end{figure}

\Cref{fig:node-deployment} outlines software and hardware implementation of \acrshort{qnodeos} and
the whole node system. \acrshort{qnodeos} is implemented in C++ on top of FreeRTOS~\cite{freertos},
a tiny operating system for microcontrollers. The stack runs on a dedicated MicroZed~\cite{microzed}
--- an off-the-shelf platform based on the Zynq-7000 SoC, which hosts two ARM Cortex-A9 processing
cores, of which only one is used, clocked at \qty{667}{\MHz}. \acrshort{qnodeos} connects to peer
\acrshort{qnodeos} systems via \acrshort{tcp} over a Gigabit Ethernet interface. We opted for a
device like the Zynq-7000 SoC for its advantageous trade-off between high flexibility and moderate
cost (around €~100 in the Netherlands at the time of writing). Whilst more concrete device
requirements may arise from our future benchmarking of \acrshort{qnodeos}, we believe that the
selected SoC provides enough computational bandwidth for the envisioned tasks, and it also offers
the possibility to implement optimized hardware modules on its \acrshort{fpga} fabric. We however
remark that the design of \acrshort{qnodeos} is in no way tied to a certain computing architecture.
For the \acrshort{qdevice}, we replicated the setup used for \cref{chp:netstack}, which mainly
consists of:
%
\begin{inlinelist}
    \item an ADwin-Pro II~\cite{adwin} acting as the main orchestrator of the setup;
    \item a series of subordinate devices responsible for qubit control, including laser pulse
          generators and optical readout circuits;
    \item the quantum physical device, based on \acrshort{nv} centers, counting one single
          (communication) qubit for each node.
\end{inlinelist}
The \acrshort{qdevice} is where the time-critical qubit control lies. \acrshort{qnodeos} interfaces
with the \acrshort{qdevice}'s ADwin-Pro II through a \qty{12.5}{\MHz} \acrshort{spi} interface, used
to exchange 4-byte control messages at a rate of \qty{50}{kHz}. Finally, the host layer is a Python
runtime running on a general-purpose 4-core desktop machine running Linux. The host machine connects
to \acrshort{qnodeos} via \acrshort{tcp} over the same Gigabit Ethernet interface that
\acrshort{qnodeos} uses to connect to its peers (average ping \acrshort{rtt} of \qty{0.1}{\ms}), and
sends application registration requests and quantum code blocks over this interface (\num{10} to
\num{1000} bytes, depending on the length of the block).

\begin{table}[t]
    \centering
    \begin{tabularx}{\linewidth}{llrYYYY}
        \toprule
                                   &                    &                & \multicolumn{4}{c}{\textbf{Lines of code}}                           \\
        \cmidrule(l){4-7}
        \textbf{Component}         & \textbf{File type} & \textbf{Files} & \textbf{Total} & \textbf{Blanks} & \textbf{Comments} & \textbf{Code} \\
        \midrule
        \multirow{4}{*}{Core code} & C/C++              & 85             & 16273          & 2779            & 1953              & 11541         \\
                                   & C/C++ header       & 121            & 13281          & 2418            & 4188              & 6675          \\
                                   & CMake              & 25             & 486            & 99              & 43                & 344           \\
                                   & Assembly           & 1              & 141            & 15              & 0                 & 126           \\
        \midrule
        \multirow{4}{*}{Test code} & C/C++              & 57             & 21849          & 3842            & 2195              & 15812         \\
                                   & C/C++ header       & 17             & 1725           & 288             & 177               & 1260          \\
                                   & Python             & 14             & 2577           & 527             & 427               & 1623          \\
                                   & CMake              & 25             & 483            & 97              & 22                & 364           \\
        \midrule
        \multicolumn{3}{l}{Estimated schedule effort (COCOMO)} & \multicolumn{4}{l}{\num{15.47} months} \\
        \multicolumn{3}{l}{Estimated people required (COCOMO)} & \multicolumn{4}{l}{\num{11.81}}        \\
        \bottomrule
    \end{tabularx}
    \caption{
        Code metrics for \acrshort{qnodeos} core code and testing code, including number of files
        per language, lines of code, and \emph{constructive cost model} (COCOMO) effort
        estimates~\cite{boehm_2009}, generated using \texttt{scc}~\cite{scc}. The COCOMO metrics
        were generated using the ``Semi-detached'' model, meaning that the estimated are computed
        assuming that the project requires a certain level of expertise and creativity and the
        problem is not well understood (i.e. there is research involved).
    }
    \label{tab:code-metrics}
\end{table}

\acrshort{qnodeos} is a complex project, developed by multiple researchers and engineers over the
course of around three years and counting. Its test infrastructure is also relatively large, with a
continuous-integration pipeline consisting of an extensive set of unit tests for each of the
\acrshort{os} core components and some system-level application tests. \Cref{tab:code-metrics}
reports code metrics for \acrshort{qnodeos} code code and testing code, including number of files
per language, lines of code, and \emph{constructive cost model} (COCOMO) effort
estimates~\cite{boehm_2009}, generated using \textit{scc}~\cite{scc}. Although these metrics do not
fully capture the research effort put into \acrshort{qnodeos}, they are an indicator of the amount
of engineering work involved. We also point out that \acrshort{qnodeos} builds on top of existing
real-time software frameworks --- namely, FreeRTOS. We implemented \acrshort{qnodeos} on top of
FreeRTOS to avoid re-implementing standard \acrshort{os} primitives like threads and network
communication. FreeRTOS provides basic \acrshort{os} abstractions like tasks, inter-task message
passing, and the \acrshort{tcpip} stack. The FreeRTOS kernel --- like any other standard
\acrshort{os} --- cannot however directly manage the quantum resources (qubits, entanglement
requests and entangled pairs), and hence its task scheduler cannot take decisions based on such
resources. \acrshort{qnodeos} adds these capabilities and takes care of the scheduling of quantum
code blocks based on the status of the quantum resources.

\section{Test Cases}
\label{sec:qnodeos:evaluation}

We propose a set of benchmarks that are to be used to verify the functioning of \acrshort{qnodeos}.
These four case studies are aimed at validating
\begin{inlinelist}
    \item single-node execution, including qubit initialization, gates, and measurements,
    \item entanglement generation,
    \item delegated quantum computation, and
    \item multitasking.
\end{inlinelist}

\subsection{Single-Qubit Gate Tomography}

Our first case study is a simple local application where a single gate is applied to a qubit
initialized in the $\ket{0}$ state, and then the qubit is measured in a number of bases. This
translates to one or more single-qubit gates and one qubit measurement. The application is run
several times to assess the quality of the prepared state, and various qubit states are analyzed.
This single-qubit gate tomography is the simplest application \acrshort{qnodeos} can run --- there
is a single user process running, and the network process is not even activated, given that
entanglement is never requested.

\paragraph{Configurations and expected results}

This application is configured to apply six different gates in separate runs: $R_x(\pi)$,
$R_x(\pi/2)$, $R_x(\pi/4)$, $R_y(\pi)$, $R_y(\pi/2)$, $R_y(\pi/4)$. Each resulting state is measured
in all six cardinal bases. This application is run \num{1000} times for each combination of gate and
readout basis. We expect the measured state fidelity to be in line with what the quantum hardware is
capable of delivering, demonstrating that the overhead incurred by \acrshort{qnodeos} is negligible,
at least when running local applications.

% The measured state fidelity, reported in \cref{fig:local-tomo}, is in line with what the quantum
% hardware is capable of delivering, showing that basic interactions among the components of
% \acrshort{qnodeos} and with the \acrshort{qdevice} work.
%
% \begin{figure}[t]
%     \centering
%     \includegraphics[width=0.6\linewidth]{figures/local-tomography.pdf}
%     \caption{
%         Average fidelity of the prepared cardinal states $\ket{+X}$, $\ket{+Y}$, $\ket{+Z}$,
%         $\ket{-X}$, $\ket{-Y}$, and $\ket{-Z}$ in the local qubit state tomography application.
%     }
%     \label{fig:local-tomo}
% \end{figure}

\subsection{Entanglement Generation}

The second test case is an application that generates an entangled pair between two nodes and then
measures the generated state. This is a distributed application, where both nodes are active ---
they engage in entanglement generation, and they both measure their end of the entangled pair. As
the user can specify the requested fidelity of the entangled pairs, this application is to be run
for various target fidelities. This time, all \acrshort{qnodeos} components are at work. Since
entanglement is requested, the quantum network stack is triggered, and thus the network process
becomes active, competing for resources with the user process. The \acrshort{qmmu} is also invoked
by the network process to transfer ownership of the entangled qubit to the user process (the
inter-process communication primitive of \acrshort{qnodeos}).

\paragraph{Configurations and expected results}

Entanglement generation is run for a range or target fidelities: \numlist{0.50; 0.55; 0.60; 0.65;
0.70; 0.75; 0.80}. Entangled pairs are read out in various bases to measure their correlators
$\braket{\mathrm{XX}}$, $\braket{\mathrm{YY}}$ and $\braket{\mathrm{ZZ}}$ (and their $\pm$
variations, for a total of $12$ correlators). The application is run \num{125} times for each
combination of target fidelity and correlator, for a total of \num{10500} entangled pairs. We expect
the measured fidelity to be at least matching --- within measurement uncertainty --- the requested
minimum fidelity, if the quantum hardware is capable of delivering such requested fidelities.

% The results for measured fidelity versus requested fidelity are shown in \cref{fig:ent-gen}. The
% measured fidelities are --- within measurement uncertainty --- always matching or exceeding the
% requested minimum ones (the dashed gray line in \cref{fig:ent-gen} is the $y=x$ diagonal). It is to
% be noted that measurement outcomes are post-processed to eliminate tomography errors and events in
% which the physical devices were in the incorrect charge state.
%
% \begin{figure}[t]
%     \centering
%     \includegraphics[width=0.6\linewidth]{figures/ent-gen.pdf}
%     \caption{
%         Measured fidelity of the entangled states generated and read out via \acrshort{qnodeos}.
%         Measurements are corrected to eliminate tomography errors and events in which the physical
%         devices were in the incorrect charge state. Error bars represent 1 s.d. The dashed gray line
%         is the $y=x$ diagonal.
%     }
%     \label{fig:ent-gen}
% \end{figure}

\subsection{Delegated Computation}

With this case study we aim to showcase a more complex quantum network application, schematically
depicted in \cref{fig:del-comp}. Here, one node acts as the client, and the other as the server. The
client's goal is to delegate a certain quantum computation on some data qubit to the server, while
keeping the server agnostic to the computation. To perform the desired computation, described by a
parameter $\alpha$, the two nodes follow these steps:
%
\begin{inlinelist}
    \item the two node establish an entangled pair,
    \item the client ``encodes'' its qubit by means of a series of local gates, described by a
          parameter $\theta$,
    \item the client measures its end of the entangled pair and stores the classical outcome
          $m_\text{c}$
    \item the client communicates the delegated computation parameter, which is a function of
          $\alpha$, $\theta$ and $m_\text{c}$,
    \item the server performs the computation,
    \item the server measures its end of the entangled pair and sends the classical outcome
          $m_\text{s}$ back to the client.
\end{inlinelist}
In this scheme, the client application consists of a single quantum code block and an additional
classical code block that communicates the computation parameter to the server. More interestingly,
the server application comprises two quantum code blocks --- the first is the establishment of the
entangled pair, and the second is the delegated computation --- interleaved by a classical code
block that receives the computation parameter from the client. This is the first example of an
application with inter-block and inter-node data dependencies, where the execution (on the server)
spans more than one quantum code block, and thus the quantum state generated in one block has to
persist and remain valid for the other block too.

\begin{figure}[t]
    \centering
    \includegraphics[width=0.6\linewidth]{figures/del-comp.pdf}
    \caption{
        Schematic of delegated computation application. The client wishes to have the server perform
        a quantum computation on a certain data qubit. To do so, the two nodes establish
        entanglement, then the client processes and measures its end of the entangled pair, sends
        the computation parameter to the server, which finally executes the delegated computation.
        Blue boxes represent quantum code blocks. The client's application is composed of a single
        block, while the server's consists of one block for entanglement and one block for the
        quantum computation, interleaved by a classical code block (the reception of the computation
        parameter).
    }
    \label{fig:del-comp}
\end{figure}

\paragraph{Configurations and expected results}

The delegated computation is run for various values of $\alpha$ ($\pi$, $\pi/2$) and $\theta$
($\pi$, $\pi/2$, $\pi/4$). The application is run \num{500} times for each combination of $\alpha$
and $\theta$. The metrics of interest are:
%
\begin{inlinelist}
    \item the measured fidelity of the qubit state after the delegated computation for each of the
          computation values,
    \item a breakdown of the average application latency, to give an indication of where time is
          spent during execution.
\end{inlinelist}
We expect the fidelity to be somewhat lower than the best-case performance due to the communication
latency incurred at the application level --- the ``Wait for $\theta, \alpha, m_c$'' block in
\cref{fig:del-comp}. In the latency breakdown, we also expect this classical communication step at
the application level to be the dominating factor. These delays are expected to manifest in
distributed applications. The resulting idle times can be allocated to other pending applications.

% We report the measured fidelity of the qubit state after the delegated computation for each of the
% computation values. We also report a breakdown of the average application latency, to give an
% indication of where time is spent during execution. \note{Plot and discuss data} The abstractions
% provided by \acrshort{qnodeos} allow for the execution of more complex quantum network applications.
% Distributed applications may result in idle time on some nodes, which \acrshort{qnodeos} can
% allocate to other pending applications.

\subsection{Multitasking}

One of the core features of modern \acrshortpl{os} is the ability to run several applications
concurrently, a key aspect in multi-user nodes and networks. \acrshort{qnodeos} is designed with
multitasking capabilities --- not only can it multiplex a user process and the network process, but
it also allows for multiple user processes to run at the same time. This means that multiple users
can submit their applications simultaneously, and \acrshort{qnodeos} will service all pending user
processes based on resource availability, in order to increase the utilization of the
\acrshort{qdevice} and to limit idle time and average application latency. One possible shortcoming
of multitasking on a quantum network node is the trade-off between concurrency and fidelity:
applications that have active data in the quantum memory, and that are waiting to be scheduled while
other applications are in progress, may experience lower-quality qubit states, given that such
quality degrades due to the passing of time and to the noise induced by operations on other qubits.

We aim to demonstrate the multitasking capabilities of \acrshort{qnodeos} by having multiple users
run independent applications concurrently. In our case study, a pool of users runs the delegated
computation application, while the rest of the users runs the local single-qubit gate tomography
application. Multitasking is evaluated on the client node, while the server just runs its part of
the delegated computation application. The idle time resulting from running the delegated
computation application on the client is a perfect candidate for scheduling other pending
applications. The multitasking performance of \acrshort{qnodeos} is assessed under various system
load conditions, which essentially depend on the number of users submitting applications
\acrshort{qnodeos} at the same time. We note that, even though a higher degree of concurrency should
in principle results in better device utilization, this is limited by the scarce physical resources
available on the underlying \acrshort{qdevice}.

\paragraph{Configurations and expected results}

Device utilization and average application latency are measured on the client, for various
configurations of users and applications: $N$ users, with $N \in \{2, 3, 5, 10\}$, half (rounded up)
of which running the local single-gate tomography application, and the remaining half (rounded down)
running the delegated computation application. To measure the performance benefit of multitasking,
we also run the same set of applications with multitasking disabled --- on \acrshort{qnodeos}, this
means that a user process can only be scheduled if no other user processes are either running or
waiting for entanglement generation. We expect the fidelity of the states measured at the end of
both applications to be somewhat comparable across the configurations with and without multitasking
enabled, given that the underlying \acrshort{qdevice} has a one-qubit memory at the moment, and thus
the level of concurrency is fairly limited. On the other hand, we expect device utilization and
average application latency to be affected by the multitasking capabilities of \acrshort{qnodeos}.
Whilst this is a relatively simple benchmark and the degree of concurrency is highly limited by the
available quantum memory, this case study should exemplify why it is important for an \acrshort{os}
for quantum network nodes to be multitasking-capable, and how such an \acrshort{os} can take
advantage of idle times.

% We report device utilization and average application latency, for the various usage patters,
% resulting from a continuous execution of each usage pattern over \qty{\approx 30}{min}. \note{Plot
% and discuss data} The scheduler of \acrshort{qnodeos}, in combination with the \acrshort{qmmu}, can
% dynamically re-prioritize outstanding processes based on resource availability. \acrshort{qnodeos}
% can thus support multi-user quantum network nodes, and can take advantage of network idle times to
% improve device utilization and average application latency. Impact of multitasking on fidelity not
% visible with a single qubit.

\section{Discussion}

The test cases discussed in this chapter complement those of \cref{chp:netstack}, and represent a
milestone for the evaluation of \acrlongpl{os} for quantum network nodes. These experiments are
planned for \acrshort{qnodeos} for the near future. The results of this experimental effort will
primarily showcase certain aspects of the importance of software abstractions for quantum
networking. Additionally, the outcome will provide insights into design and implementation strengths
an pitfalls on \acrshort{qnodeos}, and establish a baseline performance for similar studies
involving future versions of our system or alternative designs.

In the next and final chapter of this thesis we analyze the impact of \acrlong{doa} of classical
messages exchanged at the quantum network stack level. The results of this investigation will serve
as guidelines for incorporating \acrlong{doa} mechanisms into more advanced designs of
\acrshort{qnodeos}.

\begin{xstretch}
\printbibliography[heading=subbibintoc,title={References},notcategory=noprint]
\end{xstretch}
