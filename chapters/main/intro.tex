\chapter{Introduction}
\label{chp:intro}

\lettrine{Q}{uite} possibly, the word ``quantum'' is one of the most trending, and perhaps one of
the most abused, of the last decade, at least in the context of science and technology. Attaching it
to the name of tech products makes them sound more advanced. Even just a capital ``Q'' in a brand's
name suggests superiority. When I embarked on my Ph.D. at the end of 2018, I started to comprehend
how challenging it is to develop actual quantum technology. Nevertheless, then next thing I learned
is that Mozilla had already dropped a quantum (?) browser~\cite{firefox_quantum}.

However, quantum physics is not just an otherworldly theory from science fiction books, nor just a
catchy name for 21st-century consumer electronics. Since the formulation of the theory of quantum
mechanics in the early decades of the 20th century, researchers and enthusiasts have been looking
for how to make use of these physical properties, particularly in the fields of electronics,
information processing and telecommunications. Today, we know of a handful of applications that
exploit the axioms of quantum information theory to achieve something that was though to be very
hard, sometimes even impossible. Some of the most well-known use cases include fast resolution of
computational problems --- like integer factorization~\cite{shor_1994_algorithms} and database
searching~\cite{grover_1996_search} --- and efficient and secure communication schemes --- for
instance quantum key distribution~\cite{bennett_2014_bb84, ekert_1991_e91} and superdense
coding~\cite{bennett_1992_communication}.

\emph{Quantum networking}, a new paradigm of telecommunications, seeks to enhance --- not replace
--- our current internet technology to provide new functionalities that are impossible to attain
with purely classical communications. Novel applications include security-enhancing communication
schemes such as \acrfull{qkd}~\cite{bennett_2014_bb84, ekert_1991_e91}, advanced clock
synchronization routines~\cite{komar_2014_clocks}, distributed consensus
protocols~\cite{benor_2005_byzantine}, distributed sensing~\cite{gottesman_2012_telescope}, and
secure cloud quantum computing~\cite{broadbent_2009_ubqc, childs_2005_secure_qc}. Even though
quantum communications are an established reality, and their potential applications have garnered
attention from industry and research institutes, the general public is still rather puzzled, on
average, when someone tries to pitch their research on quantum networking. The above list of
applications does not necessarily appeal to the masses, which are still sometimes stuck with the
hope that quantum teleportation will instantly get them to the Bahamas~\cite{xkcd_teleportation}.
Nonetheless, the community of quantum networking researchers is not discouraged by this mismatch in
expectations, as it hopes that more applications will be devised once the technology becomes more
widespread and available to more ``consumers''. After all, we were also not aware of all the
possible uses of the classical internet when it was first developed. Yet, today the internet means
instantaneous access to low-cost clothes, scenes of hilarious felines, and --- for some --- tapes of
bare bodies engaging in intimate action on camera.

One of the most frequently asked questions about quantum technology is: \emph{When can we use it?}
If more people had access to quantum networks, they would perhaps come up with more ideas for useful
quantum networking applications. Thus, what is missing before we can deploy quantum networks
consisting of a useful number of nodes? What are the main limitations we are facing? How can we
overcome them? Not surprisingly, the answers to these questions are complicated. There is a cauldron
of fundamental and theoretical limitations, technological hardware obstacles, and computer science
puzzles that hinder the success of quantum networking. Fortunately, though, there is an increasing
community of passionate researchers trying to study these limitations, build better hardware, and
solve these puzzles. In this thesis, we will address some of the challenges of managing the activity
and the resources of a quantum network node, from an operating system's perspective. We thus aim to
answer a tiny fraction of the research questions in the field of quantum networking, and to lay
another brick in the construction of \emph{scalable}, \emph{controllable}, and \emph{configurable}
quantum networks.

The remainder of this chapter walks the reader through basic networking concepts and quantum
networking challenges, lists the research questions we aim to address, and provides an outline of
the rest of the thesis.

\section{Networking and Quantum Networking}

Digital communication networks have come a long way from the early days of the \textsc{arpanet} ---
one of the most important precursors of today's internet. Back in 1969, one of the most advanced
networks of computers consisted of just \emph{four} nodes. In 1981, as the global network grew
larger, networking researchers standardized the \acrfull{ipv4}, which allowed up to \num{4} billion
devices to have their own address on the public internet~\cite{rfc_791}. Soon after, it became clear
that the pool of available \acrshort{ipv4} addresses was going to be depleted sooner than later ---
which happened in 2011~\cite{icann_2011}. In 2023, there are \num{3.6} devices connected to the
internet per capita, as estimated by \citeauthor{cisco_2020} in 2020~\cite{cisco_2020}.

Scaling up from a four-node experiment to the massive networks of the 21st century was no easy feat
of course. This was made possible by advancements in various fields, including networking hardware,
traffic engineering, and network programmability. You would most likely not be able to download a
digital copy of this thesis in a fraction of a second if it were not for \unit{\tera\bit\per\second}
network switches, a diverse spectrum of routing protocols, and \acrlong{sdn}. Nevertheless,
classical networking was already appealing in its infant stages, for the simple reason that even a
small network of nodes can accomplish tasks that would not be attainable without it --- in the case
of the \textsc{arpanet}, sending simple pieces of text over large distances almost instantaneously.
The applications of quantum networking are not dissimilar to their classical networking counterpart,
in that some of them can be useful and effective on small quantum networks already, whilst other use
cases require more powerful nodes and more complex networks~\cite{wehner_2018_stages}.

Even though hardware and software advancements were essential to the betterment of the internet,
there are a few basic design ingredients that have been there since the dawn of classical networking
and that have made these technological leaps even possible. For instance, it was immediately clear
that data was to be grouped and transmitted according to the \emph{packet switching} model, to
maximize network utilization and to allow for dynamic routing decisions. As another example,
networking protocols have been organized in \emph{abstraction layers} since the very beginning of
computer networks, so as to encapsulate network functionalities in services of increasing
user-friendliness. When designing quantum networks and networking services, one should draw
inspiration from classical networking principles and avoid monolithic designs and software
architectures that would render the system hard to scale.

Then, how similar are quantum networks and classical networks? Which are the challenges that they
have in common, and which ones are exclusive to quantum networking? How much can we capitalize on
the vast body of classical networking literature? In principle, quantum networks are just networks
with a special physical layer, which, albeit more technologically complex, could be abstracted away
and encapsulated into an ad-hoc networking protocol. This simplification would, however, disregard
some fundamental limitations that are inherent to quantum information, as well as the imperfect
nature of near-term quantum hardware. Standard networking routines like signal amplification,
classical error correction, and data retransmission would not work in quantum networks --- one
fundamental theorem of quantum mechanics states that \emph{an arbitrary quantum state cannot be
cloned}~\cite{wootters_1982_nocloning, dieks_1982_communication}. Moreover, quantum states are
subject to \emph{decoherence}, a physical process whereby the quality of the stored information
degrades over time and due to external interferences. Thus, not only do computation and networking
delays affect throughput and latency, but they also exact a toll on the quality of the service,
effectively determining whether a certain application produced meaningful results or not. Whilst
decoherence can be worked around with more sophisticated quantum hardware and control algorithms,
the no-cloning theorem is a hard limit on what one can do with quantum information, and thus we
cannot design quantum networking protocols assuming quantum information can be freely copied and
stored indefinitely.

Without being too speculative, we can argue that we cannot just encapsulate the requirements of
quantum networking into a specialized physical layer. Reusing classical networking techniques and
protocols as they are would fall short of the aforementioned challenges posed by quantum mechanics.
Nonetheless, many of the questions that drove the classical networking research community can be of
inspiration for analyzing requirements and limitations of quantum networks. Examples of such
questions are: \emph{Can we organize quantum information into packets? Do we need to resort to path
reservation for quantum communications? In which situations can we tolerate local and network
latency? How do we organize and layer quantum networking protocols and services? What metrics do we
look at when designing routing protocols? Can we improve performance and quality of service with the
employment of a software-defined control plane?}

\section{Research Goals}

Perhaps disappointingly, but unsurprisingly too, this thesis will not try to answer all the research
questions from the previous section. The good news is that there are many ongoing efforts from
various scientists that are looking into these questions. Most of the times, however, researchers
have to validate their designs on a simulated quantum network, given that we do not yet have access
to mid-scale testbeds consisting of more than a handful of nodes --- the most advanced of which
features three interconnected devices~\cite{pompili_2021_multinode}. On the other hand, we need a
framework to evaluate quantum networking protocols and applications on real quantum networks, to
help us verify our assumptions and simulation results even in the early stages of this research
field. In this thesis, we will discuss design considerations for such a framework, design and
implement a rudimentary instance of it, and evaluate our design and implementation on a quantum
network.

The goal of this thesis is to provide a framework that facilitates the experimental investigation of
quantum networking-related research questions, and that can help researchers learn about the
behavior of quantum communication applications without having to delve into the complexity of
running and managing the underlying network and the interaction of the nodes with it. We refer to
this framework as an \emph{\acrlong{os}} (\acrshort{os}), as its goal is to abstract and manage
quantum physical processes and resources to provide a user-friendly interface to the application.
More specifically, we will address the following questions:

\begin{enumerate}[label={Q\arabic*.}]
    \item \emph{What goals should an \acrshort{os} for quantum network nodes achieve?} We explore
          challenges, requirements and goals that one should consider when designing such an
          \acrshort{os}, whose overarching objective is to bridge the gap between high-level user
          applications and low-level quantum networking hardware.
    \item \emph{What does an architecture for such an \acrshort{os} look like?} We propose the first
          proof-of-principle architecture for an \acrshort{os} for quantum network nodes. The
          architecture ensures the \acrshort{os} can be deployed on various quantum platforms,
          provides means to manage resources and schedule operations, and allows running multiple
          quantum networking applications concurrently.
    \item \emph{What is the performance of the \acrshort{os}'s quantum network stack?} We revise and
          implement state-of-the-art quantum network protocols to be integrated in the proposed
          \acrshort{os}, and evaluate their performance for a basic networking feature: entanglement
          delivery.
    \item \emph{What is the performance of the whole \acrshort{os}?} We implement the proposed
          architecture, and evaluate its functioning and performance for some basic quantum
          networking applications, including scenarios of concurrent execution of multiple
          applications.
    \item \emph{How would \acrlong{doa} affect the performance of a quantum link?} We evaluate, this
          time in simulation, what penalty would be incurred in the performance of entanglement
          generation if the quantum network stack would communicate over an authenticated classical
          channels, as opposed to exchanging non-authenticated classical messages.
\end{enumerate}

This work is aimed at designing, implementing and evaluating a ``product'' that, although
experimental and not production-ready, is a fully-functional research tool, and as such is ready to
be reused and adapted with little effort by anyone who is interested in experimenting with quantum
networking protocols and applications on real networks. Our implementation-driven research does not
intend to produce the best protocols and algorithms for the control of quantum network nodes ---
rather, it wishes to establish a baseline for such a system, and a framework to study and test more
advanced versions of its components. To demonstrate the applicability of our tool, we evaluate the
\acrshort{os} on a small state-of-the-art quantum network based on \acrlong{nv} centers in
diamond~\cite{pompili_2021_multinode}, deployed in a laboratory environment.

\begin{table}[t]
    \centering
    \begin{tabularx}{\linewidth}{Xl}
        \toprule
        \textbf{Object}                                  & \textbf{Location}                                                                            \\
        \midrule
        Datasets for \cref{chp:netstack}                 & \href{https://doi.org/10.4121/16912522}{\textsc{doi}: \texttt{10.4121/16912522}}             \\
        Datasets for \cref{chp:qnodeos}                  & \href{https://doi.org/xx.xxxx/xxxxxxxx}{\textsc{doi}: \texttt{TBD}}                          \\
        Datasets for \cref{chp:doa}                      & \href{https://doi.org/yy.yyyy/yyyyyyyy}{\textsc{doi}: \texttt{TBD}}                          \\
        Data analysis for \cref{chp:netstack}            & With data (\href{https://doi.org/10.4121/16912522}{\textsc{doi}: \texttt{10.4121/16912522}}) \\
        Data analysis for \cref{chp:qnodeos}             & With data (\href{https://doi.org/xx.xxxx/xxxxxxxx}{\textsc{doi}: \texttt{TBD}})              \\
        Data analysis for \cref{chp:doa}                 & With data (\href{https://doi.org/yy.yyyy/yyyyyyyy}{\textsc{doi}: \texttt{TBD}})              \\
        Application \acrshort{sdk} (NetQASM)             & \url{https://github.com/QuTech-Delft/netqasm}                                                \\
        Applications for \cref{chp:netstack,chp:qnodeos} & \url{https://gitlab.tudelft.nl/TBD/}                                                         \\
        \bottomrule
    \end{tabularx}
    \caption{
        Location of experimental data and software supporting this thesis.
        \note{Add missing references}
    }
    \label{tab:data-and-soft}
\end{table}

\section{Data and Software Availability}

The datasets that support this thesis are made public and available in the online data repository
4TU.ResearchData. The software to analyze such data is also made public. The application
\acrfull{sdk} is open-sourced on GitHub. The application scripts themselves are also open-sourced.
Finally, the software for the \acrlong{os} implemented in this thesis is part of a larger project
that is still under development, and this not currently available for external use. Refer to
\cref{tab:data-and-soft} for pointers to data and software.

\section{Thesis Outline}

The rest of this thesis is organized as follows. \Cref{chp:background} offers some background on
quantum information, quantum networking and classical networking, and recaps the main challenges
involved. \Cref{chp:arch} outlines the most important design considerations that serve as the basis
for the design of our \acrshort{os} (question Q1), and describes our proposal for the architecture
of a quantum network node's \acrshort{os} (question Q2). \Cref{chp:netstack} evaluates the
performance of the quantum network stack integrated in the \acrshort{os} for entanglement generation
(question Q3). \Cref{chp:qnodeos} validates and benchmarks the full \acrshort{os} against a set of
simple quantum networking applications, also when running multiple of these concurrently (question
Q4). \Cref{chp:doa} quantifies, this time in simulation, the effect of authenticating classical
messages in the quantum network stack on the performance of a quantum link (question Q5). Finally,
\cref{chp:conclusion} concludes this thesis and reflects upon future steps.

\printbibliography[heading=subbibintoc,title={References},notcategory=noprint]
