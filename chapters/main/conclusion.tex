\chapter{Conclusion}
\label{chp:conclusion}

\lettrine{M}{ost} of the results obtained in this thesis were aimed at laying the foundations for a
quantum internet where nodes are programmable, and can be interacted with by anyone with basic
knowledge of quantum networking, as opposed to only being operable by quantum physics experts.
Whilst this is merely a single step towards more scalable quantum networks, we have demonstrated the
importance of abstractions in this emerging field, and we hope this work can be of inspiration to
future research seeking to continue on this path.

\section{Summary of Results}

The primary focus of this work was to design, build and experimentally validate \emph{software
abstractions} for \emph{programmable} quantum network nodes, as the title of this dissertation reads
out. Here we recap the main results of our investigation.

\paragraph{Design considerations}

To begin with, we have explored the fundamental reasons that make quantum networking challenging,
and why classical networking and \acrfull{os} research fall short of such challenges. We have thus
laid out general design considerations to be addressed when thinking about abstractions for quantum
network nodes. With respect to that, we have also put forward a shortlist of the main constituent
parts of an \acrshort{os} for such nodes --- which include a network stack for quantum
communications, a process scheduler, and a \acrlong{qmmu}. We have done so in \cref{chp:arch}.

\paragraph{Design of an operating system for quantum network nodes}

Following our design considerations, we have produced a first design of an \acrshort{os} for quantum
network nodes. Our system, which we call \acrshort{qnodeos}, includes all the components as per the
design considerations, and represents a fully-functional proof of concept to be used for further
research on the topic. Key design points of \acrshort{qnodeos} are:
\begin{inlinelist}
    \item the integration of a state-of-the-art quantum networking stack, to coordinate entanglement
          generation in a platform-independent manner;
    \item the separation of quantum networking and local operations into distinct processes, to
          allow for better scheduling decisions and higher concurrency;
    \item the isolation of quantum platform-specific abstractions into a device driver, called
          \acrshort{qdevice} driver, to enable deploying \acrshort{qnodeos} onto various quantum
          platform with minimal effort.
\end{inlinelist}
Our design was presented in \cref{chp:arch}.

\paragraph{Entanglement generation}

As a first case study of \acrshort{qnodeos}, we tested its quantum networking capabilities by means
of three simple applications centered around entanglement generation. We have implemented the
quantum networking stack proposed in recent publications~\cite{dahlberg_2019_egp,
kozlowski_2020_qnp}, integrated it into \acrshort{qnodeos}, and demonstrated its operation on a
two-node state-of-the-art quantum network based on \acrlong{nv} center technology. With these
preliminary tests, we showed how \acrshort{os} abstractions can simplify the programming of quantum
network nodes, at a minimal, almost negligible performance cost. We in fact found that, while the
quantum networking stack introduces latency as compared to ``bare-metal'' entanglement generation at
the physical layer, the application-level fidelity of the delivered states is not much affected by
this overhead. These results were presented in \cref{chp:netstack}.

\paragraph{Quantum networking applications and multitasking}

To further showcase the service offered by \acrshort{qnodeos} and to establish a performance
baseline for an \acrshort{os} for quantum network nodes, we took the experiments to another level by
running fully-fledged quantum networking applications through \acrshort{qnodeos}. Primarily, we
demonstrated the complete run of a delegated quantum computation protocol, as well as the concurrent
execution of multiple applications, leveraging the multitasking capabilities of \acrshort{qnodeos}.
The results, again obtained on a two-node quantum network based on \acrlong{nv} centers, asserted
the importance of a robust stack of abstraction layers for quantum networking. \note{Refine after
\acrshort{qnodeos} experiments are done.} We presented these results in \cref{chp:qnodeos}.

\paragraph{Data origin authentication in quantum networking}

In perspective of a large-scale, fully-operational quantum network, where classical communications
ought to be more secure than in a laboratory, we have also analyzed the performance penalty that
would be incurred if classical messages exchanged at the quantum networking stack were to be
authenticated so as to ensure integrity. Using a combination of experimentally-measured
authentication delays and the simulation of a quantum link, we have shown that the additional delay
incurred by \acrlong{doa} would not be detrimental to the fidelity of the delivered entangled
states. This topic was investigated in \cref{chp:doa}.

\section{Future Work}

We hope that our work and results open the door to further research on the topic of \acrshortpl{os}
for quantum networking. On that regard, we have identified a few areas of this field that would
benefit from a deeper investigation, which we summarize here.

\paragraph{Quantum network node architecture}

In our design, we have defined a certain architecture of a quantum network node the separation, and
we have not investigated possible alternatives. In particular, the separation between host,
\acrshort{qnodeos} and \acrshort{qdevice} determines which component is responsible for which task,
and limits certain scheduling decisions to certain components. In the future, one might want to
explore other solutions and examine scenarios in which, for instance, resource allocation and
scheduling decisions are not only taken on \acrshort{qnodeos}.

\paragraph{Scheduling of processes and applications}

The rudimentary process scheduler embedded in \acrshort{qnodeos} is just good enough to guarantee
concurrency and to prioritize what in general is deemed to have higher priority in quantum
networking applications --- that is, entanglement generation. Nevertheless, an optimal scheduler
might need to factor in more run-time data to take better scheduling decisions that are
context-dependent. For instance, one could well imagine a scheduler which is aware of the
approximate duration of the tasks to be scheduled, as well as of the current level of decoherence of
the quantum states in memory, and thus re-compute task priorities at run-time when needed.
Additionally, as the responsibilities of \acrshort{qnodeos} and host might be redefined (see
previous paragraph), one could design the host such that it takes part in making scheduling
decisions, or at least supporting \acrshort{qnodeos} with application- and device-specific hints.

\paragraph{Integration with control plane}

At the moment, the interactions of \acrshort{qnodeos} with the control plane of a quantum network
are mostly one-sided --- the control plane installs network schedules on \acrshort{qnodeos}, mostly.
In a future iteration of \acrshort{qnodeos}, and in a full-blown implementation of a control plane,
the former could feed a set of run-time statistics back into the control plane scheduler, so as to
provide an up-to-date picture to the latter, which can in turn refine the computing of the network
schedule in real-time.

\paragraph{Smart quantum memory management unit}

The \acrfull{qmmu} implemented on the first version of \acrshort{qnodeos} is rather basic. As
discussed in \cref{chp:arch}, a more advanced iteration of the \acrshort{qmmu} might feature smart
mechanisms for tracking (estimating) qubit decoherence to offer better run-time scheduling support,
as well as device-specific descriptions of the underlying quantum memory to improve allocation
strategies that better cater to memory lifetime requirements and to the limitations of near-term
quantum devices.

\paragraph{Larger-scale experimental validation and benchmarking}

As quantum networks become larger in scale, and the nodes therein begin to offer a larger set of
quantum resources, a natural experimental follow-up of this thesis would see \acrshort{qnodeos}, or
any future \acrshort{os} for quantum networks, be tested again more complex quantum networking
applications, especially those requiring more qubits, and those involving more parties within a
network. Along the same lines, a great addition to this work would be a set of experimental tests
where \acrshort{qnodeos} is deployed on a larger variety of quantum physical platforms, eventually
even on a heterogeneous network of several different devices.

\printbibliography[heading=subbibintoc,title={References},notcategory=noprint]
